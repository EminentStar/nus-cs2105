\documentclass[a4paper,12pt,answers]{exam}
\usepackage{hyperref}
\ifx\pdftexversion\undefined
\usepackage{epsfig}
\else
\usepackage[pdftex]{graphics}
\fi
\usepackage{epsfig}

\usepackage{fancyvrb}

\rhead{\textsf{CS2105 Assignment 3}}
\lhead{\textsf{Matriculation Number:
}}
\headrule

\begin{document}
\extraheadheight{.5in}
\firstpageheader{\large\sf CS2105}%
{\large\sf National University of Singapore\\ School of Computing \\
\LARGE\sf Assignment 3}%
{\large\sf Semester 2 12/13}
\firstpageheadrule
\pagestyle{headandfoot}

\begin{center}
{\Large{\sf MATRICULATION NUMBER:}}
\begin{tabular}{| c | c | c | c | c | c | c | c | c |}
\hline
\hspace{.3cm} &
\hspace{.3cm} &
\hspace{.3cm} &
\hspace{.3cm} &
\hspace{.3cm} &
\hspace{.3cm} &
\hspace{.3cm} &
\hspace{.3cm} &
\hspace{.3cm} \\[11pt]
\hline
\end{tabular}
\end{center}

\textbf{Deadline: 15 April, 6:00pm}

This assignment can be done on your local computer.  We assume that you are familiar with basic user interfaces of Wireshark and have installed it on your computer.  Otherwise, please take a look at past DIY exercises for guidance.

The goal of this assignment, like the DIY exercises, is to relate what you have learnt from CS2105 to networking practice in the real world.  This exercise also introduces a few new concepts that are not in the book/lecture.  You will need to be resourceful (e.g., using Google, Wikipedia) to answer these questions.

Answer all questions in the space provided.  Write your matriculation number on top of this page, and hand in a hardcopy of this assignment to Wei Tsang's office (AS6 05-14) before 6pm, 15 April 2013.

\section*{Your Task}

In this assignment, you are given a file, named a3.packets, that contains a set of packets captured using Wireshark on a host with IP address 172.26.187.16.  The host was originally not connected to the Internet and has its Ethernet cable unplugged.  Trace collection starts (at Time 0) as soon as the Ethernet cable is plugged into the host.  

You can download this file from \url{http://www.comp.nus.edu.sg/~ooiwt/cs2105/1011s2/a3.packets}.  Open this file using Wireshark.  Examined the packets carefully, and try to reconstruct the sequence of events when a host is plugged into the network.

\begin{questions}

\question \textbf{ARP and DHCP}
\begin{parts}
\uplevel{We start by looking at Packet 1.}
	\part[1] What is the MAC address of 172.26.187.16?  How did you find this out?
	%\fillwithdottedlines{.5in}
	\begin{solution}
		64:b9:e8:c9:07:b4.  Source of Packet 1.
	\end{solution}
	\part[2] What is the destination MAC address of Packet 1?  Why isn't the destination MAC address FF:FF:FF:FF:FF:FF as expected?
	%\fillwithdottedlines{.5in}
	\begin{solution}
The host looks up the ARP table and found that, previously, the MAC address of 172.26.186.1 is 00:00:0c:07:ac:02.  So it sends it directly to that MAC address instead of broadcasting to everyone.
	\end{solution}
	\part[1] What is the purpose of Packet 1?
	%\fillwithdottedlines{.5in}
	\begin{solution}
		To reconfirm the MAC address of 172.26.186.1.
	\end{solution}
	\part[1] Is Packet 1 sent using UDP, TCP, or neither?
	%\fillwithdottedlines{.5in}
	\begin{solution}
		Neither.  ARP is a protocol between link and network layer.
	\end{solution}
	\part[1] Which packet is the response to Packet 1? (indicate the packet number).
	%\fillwithdottedlines{.5in}
	\begin{solution}
		Packet 11.
	\end{solution}
\uplevel{Now, let's look at Packet 2.  Note that Packet 2 is a DHCP Request packet, instead of a DHCP Discover packet as explained in the book/lecture.}
	\part[1] Why is the source IP address 0.0.0.0?  
	%\fillwithdottedlines{.5in}
	\begin{solution}
		Because the host does not have an IP address yet.
	\end{solution}
	\part[1] Which packet is the response to this DHCP request?
	%\fillwithdottedlines{.5in}
	\begin{solution}
		Packet 16 (or 17).
	\end{solution}
\uplevel{Now, take a look at Packets 5 - 8.}
	\part[1] These packets does not concern 172.26.187.16.  Why do these packets still get delivered to 172.26.187.16?
	%\fillwithdottedlines{.5in}
	\begin{solution}
		They are broadcast packets.	
	\end{solution}
	\part[3] Now, right click on ``Source'' column of Packet 5, and choose ``Apply as Filter'', ``Selected''.  Take note of the ARP packets sent from the same Ethernet address, and the timing interval between the packets.  What do you think the purpose of these packets is?
	%\fillwithdottedlines{.5in}
	\begin{solution}
		They are likely used to probe if the IP addresses are in used.  (To find out more, google for periodic ARP probes).
	\end{solution}
	\part[1] The same MAC address is also the link-layer source for two other IP addresses.  What are they?  What is the associated role played by the host with each of these IP addresses?
	%\fillwithdottedlines{.5in}
	\begin{solution}
		137.132.82.7 is the DHCP Server; 137.132.87.2 is the DNS server.
	\end{solution}
	\uplevel{Now, click ``Clear'' to remove the filter.  Click on Packet 20, and filter packets based on its source MAC address.  Suppose the MAC address is X, filter the packets using the expression "eth.addr == X" by entering and applying it in the display filter field.}
	\part[1] The same MAC address is also the link-layer source for two IP addresses under 137.132.80/20.  What are they?  What is the associated role played by the host with each of these IP addresses?
	%\fillwithdottedlines{.5in}
	\begin{solution}
		137.132.82.6 is the DHCP Server; 137.132.85.2 is the DNS server.
	\end{solution}
	\part[2] Does 137.132.82.6 belong to the same LAN (not separated by a router) as 172.26.187.16?  How can you tell?
	%\fillwithdottedlines{.5in}
	\begin{solution}
		Yes, because 172.26.186.16 can hear broadcast packet from 137.132.82.6; and no, because they have different subnet mask.
	\end{solution}

\uplevel{Click ``Clear'' to remove the filter, and let's move on to Packet 12}
	\part[1] Packet 12 is labelled as a Gratuitous ARP packet.  What is the source and destination IP address of this packet?
	%\fillwithdottedlines{.5in}
	\begin{solution}
		They are both 172.26.187.16
	\end{solution}
	\part[3] What is the purpose of sending a Gratuitous ARP request packet?
	%\fillwithdottedlines{.5in}
	\begin{solution}
		To check if anyone else have the same IP address. 
	\end{solution}
\end{parts}

\question \textbf{SSL}.  We now turn our focus to Packet 36, which starts a TCP connection to a remote server.  Right click on Packet 36 and choose ``Conversation Filter'', ``TCP''.
You can observe the establishment of a TCP connection, followed by a SSL connection.  Click on Packet 42, where the certificate of the server is handed over to 172.26.187.16.  
Look into the content of the packet, paying attention of the certificates.  
\begin{parts}
	\part[2] Why are there two certificates?  What is their relation?
%	\fillwithdottedlines{1in}
	\begin{solution}
		The first certificate certifies the public key of Google, signed by Google Internet Authority.  The other certificate certifies the public key of Google Internet Authority, signed by Equifax Secure Certificate Authority.
	\end{solution}
	\part[2] What is the purpose of Packet 44?  What is it encrypting, and what is the message encrypted with?  (Note: the answer comes from the textbook/lecture, not from the payload of the packet).
	%\fillwithdottedlines{.5in}
	\begin{solution}
		Packet 44 contains the master key encrypted with the public key of google.com.
	\end{solution}
\end{parts}
\end{questions}
\vfill
\begin{center}
    \textsf{\Huge THE END}
\end{center}
\end{document}
