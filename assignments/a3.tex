\documentclass[a4paper,12pt,answers]{exam}
\usepackage{hyperref}
\ifx\pdftexversion\undefined
\usepackage{epsfig}
\else
\usepackage[pdftex]{graphics}
\fi
\usepackage{epsfig}
 
\usepackage{fancyvrb}
 
\begin{document}
\extraheadheight{.5in}
\firstpageheader{\large\sf CS2105}%
{\large\sf National University of Singapore\\ School of Computing \\
\LARGE\sf Assignment 3}%
{\large\sf Semester 2 13/14}
\firstpageheadrule
\pagestyle{headandfoot} 


\textbf{Deadline: 18 April, 6:00pm}

This assignment can be done on your local computer. We assume that you are
familiar with basic user interfaces of Wireshark and have installed it on your
computer. Otherwise, please take a look at past DIY exercises for guidance.

The goal of this assignment, like the DIY exercises, is to relate what you have
learnt from CS2105 to networking practice in the real world. This exercise also
introduces a few new concepts that are not in the book/lecture. You will need to
be resourceful (e.g., using Google, Wikipedia) to answer these questions.

Answer all questions in the answer sheet provided, and hand in a hardcopy of the
answer sheet to Wei Tsang's office (AS6 05-14) before 6pm, 18 April 2014 (Friday).

Submission of answers in any other form (softcopy, foolscap paper etc.) will result 
in an automatic deduction of 1 point.  

WRITE YOUR MATRICULATION NUMBER OF THE TOP LEFT HAND CORNER OF EVERY PAGE IN THE ANSWER SHEET.  Failure to do so could result in 0 marks for the assignment (since we do not know who you are).

\section*{Your Task}

In this assignment, you are given a file, named a3.packets, that contains a set
of packets captured using Wireshark on a host with IP address 172.26.191.153.
We call this computer the capturing host in the questions. The host was
originally not connected to the Internet and has its Ethernet cable unplugged.
Trace collection starts (at Time 0) as soon as the Ethernet cable is plugged into the host.

You can download this file from
\url{http://www.comp.nus.edu.sg/~ooiwt/cs2105/1314s2/a3.packets}. Open this file
using Wireshark. Examined the packets carefully, and try to reconstruct the
sequence of events when a host is plugged into the network.

\begin{questions}
\question[10] \textbf{ARP}
\begin{parts} 
\part[1] What is the purpose of Packet 9?
%\fillwithdottedlines{.25in}
% \begin{solution} 
% A. To obtain the mac address of 172.26.190.1.
% \end{solution}
\part[1] Who sends Packet 9?
%\fillwithdottedlines{.25in}
% \begin{solution}
% A. 172.26.191.183
%\end{solution}
\part[2] Does the response to Packet 9 exist in the trace? Why?
%\fillwithdottedlines{.5in}
% \begin{solution}
% A. No. Because 172.26.190.1 responds within a standard frame to the
%  172.26.191.183 not in the broadcast manner.
% \end{solution}
%\uplevel{Consider Packet 126:}
\part[1] What is the purpose of Packet 126?
%\fillwithdottedlines{.25in}
% \begin{solution}
% A. To query for the mac address of the IP 172.26.190.1.
% \end{solution}
\part[1] Which packet is the response to Packet 126?  Write down the most important
information brought to the capturing host by the response of Packet 126.
%\fillwithdottedlines{.25in}
% \begin{solution}
% A. 127. The Mac address of IP 172.26.190.1 which is 00:00:0c:07:ac:02.
% \end{solution}
\part[3] What is the purpose of Packet 18?
%\fillwithdottedlines{.75in}
% \begin{solution}
% A. Gratuitous ARP message is useful for updating other hosts' mapping of a
% hardware address when the sender's IP address or MAC address has changed.
% \end{solution}
\part[1] What is the MAC address of capturing host?
%\fillwithdottedlines{.25in}
\end{parts}

\question[8] \textbf{DHCP}
\begin{parts} 
\part[1] Find the IP addresses of two DHCP servers that are accessible from
the subnet of capturing host.
%\fillwithdottedlines{.25in}
% \begin{solution}
% A.137.132.83.7 and 137.132.83.6
% \end{solution}
\part[3] Are the DHCP servers in the same subnet as the capturing host?  Justify your answer.
%\fillwithdottedlines{.75in}
% \begin{solution}
% a. By finding the DHCP ack packets, we can find 137.132.83.7 and
% 137.132.83.6 as the source IP address. By finding the subnet mask for
% 172.26.191.153 which is 255.255.254.0, you can find that they are not in the same subnet.
% \end{solution}
\part[2] How many distinct clients (including the capturing host) request for
the DHCP IP address in the subnet of capturing host during the trace time?
%\fillwithdottedlines{.25in}
% \begin{solution}
% Two. Answer Description: We can obtain this by list all the DHCP Discover
% messages and then count the number of different mac address.
% \end{solution}
\part[2] Why is the first DHCP packet a DHCP Request packet, rather than a DHCP Discover packet?
%\fillwithdottedlines{.5in}
% \begin{solution}
% A. Because the capturing host wants to keep its previous IP address if it is possible.
% \end{solution}
\end{parts}

%\question \textbf{General}
%\begin{parts}
% \begin{solution}
% d4:be:d9:9d:b4:ba.
% \end{solution}
%\end{parts}

\question[5] \textbf{DNS}
\begin{parts}
\part[2] What are the IP addresses of the three DNS servers which are contacted
by capturing host?
%\fillwithdottedlines{.25in}
% \begin{solution}
% 137.132.87.2, 137.132.85.2 and 137.132.94.2
% \end{solution}

\part[2] What could be the reason for contacting three DNS servers instead of one?
%\fillwithdottedlines{.75in}
% \begin{solution}
% The client may contacts several DNS server in parallel to save the time in
% the case of failure of one or when an address may not be cached in one DNS
% server but may be cached in another.
% \end{solution}

\part[1] What is the longest TTL among the DNS records corresponding to the IP
address of daisy.ubuntu.com (starts from packet 128)? Which DNS server does it
come from? (You can use ctrl+F to look for daisy.ubuntu.com)
%\fillwithdottedlines{.25in}
% \begin{solution}
% 9 min and 31 second comes form 137.132.85.2.
% \end{solution}
\end{parts}

\question[7] \textbf{SSL}
\begin{parts}
\uplevel{We now turn our focus to Packet 145, which starts a TCP connection to a
remote server. Right click on Packet 145 and choose ``Conversation Filter'',
``TCP''. You can observe the establishment of a TCP connection, followed by a SSL
connection. Click on Packet 211, where the certificate of the server is handed
over to 172.26.191.153. Look into the content of the packet, paying attention of
the certificates.} 
\part[3] Why are there four certificates? What is their relation?
%\fillwithdottedlines{.75in}
% \begin{solution}
% The first certificate certifies the public key of diasy.ubuntu.com, signed by
% "Go Daddy Secure Certificate Authority". The second certificate certifies the
% public key of "Go Daddy Secure Certificate Authority", signed by "Go Daddy
% Root Certificate Authority". The third certificate certifies the public key
% of "Go Daddy Root Certificate Authority", signed by "Go Daddy Class 2
% Certification Aut".
% \end{solution}

\part[2] Is it trustable that the forth certificate is self-signed? Explain your answer.
%\fillwithdottedlines{.75in}
% \begin{solution}
% \end{solution}

\part[2] What is the purpose of Packet 211? What is it encrypting, and what is
the message encrypted with? (Note: the answer comes from the textbook/lecture, not
from the payload of the packet).
%\fillwithdottedlines{.75in}
% \begin{solution}
%Packet 211 contains the master key encrypted with the public key of
% diasy.ubuntu.com
% \end{solution}

\end{parts}
\end{questions}
\vfill
\begin{center}
    \textsf{\Huge THE END}
\end{center}
\end{document}