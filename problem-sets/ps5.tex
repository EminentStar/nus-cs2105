\documentclass[a4paper,11pt]{exam}
    \usepackage{hyperref}
    \ifx\pdftexversion\undefined
       \usepackage{epsfig}
    \else
       \usepackage[pdftex]{graphics}
    \fi
    \usepackage{epsfig}
\begin{document}
    \extraheadheight{.5in}
    \firstpageheader{\large\sf CS2105}%
    {\large\sf National University of Singapore\\ School of Computing \\
    \LARGE\sf Problem Set 5}%
    {\large\sf Semester 2 10/11}
    \firstpageheadrule
    \pagestyle{headandfoot}

    \begin{questions}
	\question \textbf{(KR, Chapter 4, Problem 13)}
	Consider a subnet with prefix 101.101.101.64/26.  Give the range
	of IP addresses (of form xxx.xxx.xxx.xxx) that can be
	assigned to this network. 
	
	Suppose an ISP owns the block of
	addresses of the form 101.101.128/17.  Suppose it wants to
	create four subnets from this bock, with each block having the
	same number of IP addresses.  What are the prefixes (of form
	a.b.c.d/x) for the four subnets?

	\question \textbf{(KR, Chapter 4, Problem 16)}
	Consider the topology shown in Figure 4.17.  Denote the three subnets
	with hosts (starting clockwise at 12:00) as Networks A, B, and C.  Denote
	the subnets without hosts as Networks D, E, and F.
	\begin{figure}[h!]
		\begin{center}
	\includegraphics[scale=0.18]{f417.pdf}
		\end{center}
	\end{figure}
	\begin{parts}
	\part Assign network addresses to each of these six
	subnets, with the following constraints: All
	addresses must be allocated from 214.97.254/23;
	Subnet A should have enough addresses to support 250
	interfaces; Subnet B should have enough addresses to
	support 120 interfaces; and Subnet C should have
	enough addresses to support 120 interfaces.  Of
	course, subnets D, E, and F should each be able to
	support two interfaces.  For each subnet, the
	assignment should take the form a.b.c.d/x or
	a.b.c.d/x - e.f.g.h/y.  
	\part Using your answer to
	part (a), provide the forwarding tables (using
	longest prefix matching) for each of the three
	routers.
	\end{parts}

	\question \textbf{(KR, Chapter 4, Problem 18)}
	In this problem we will explore the impact of NATs on P2P applications\footnote{Knowledge of
	Section 2.6 not required to solve this problem}.  Suppose a peer with
	user name Arnold discovers through querying that a peer with user name Bernard has a file it
	wants to download.  Also suppose that Bernard is behind a NAT whereas Arnold isn't.  Let
	138.76.29.7 be the WAN-side address of the NAT and let 10.0.0.1 be the internal IP address
	for Bernard.  Assume that the NAT is not specifically configured for the P2P application.
    \begin{parts}
	\part Discuss why Arnold's peer cannot initiate a TCP connection to Bernard's peer, even if
	Arnold knows the WAN's side address of the NAT, 138.76.29.7.
	\part Now suppose that Bernard has established an ongoing TCP connection to another peer,
	Cindy, which is not behind a NAT.  Also suppose that Arnold learned from Cindy that Bernard
	has the desired file and that Arnold can establish (or already has established) a TCP
	connection with Cindy.  Describe how Arnold can use these two TCP connections (one from
	Bernard to Cindy, and the other from Arnold to Cindy) to instruct Bernard to initiate a
	direct TCP connection (that is, not passing through Cindy) back to Arnold.  This technique
	is sometimes called \textit{connection reversal}.  Note that even though Bernard is behind
	a NAT, Arnold can use this direct TCP connection to request the file, and Bernard can use
	the connection to deliver the file.
    \end{parts}


    \end{questions}
\end{document}
