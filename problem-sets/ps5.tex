\documentclass[a4paper,11pt]{exam}
    \usepackage{hyperref}
    \ifx\pdftexversion\undefined
       \usepackage{epsfig}
    \else
       \usepackage[pdftex]{graphics}
    \fi
    \usepackage{epsfig}
\begin{document}
    \extraheadheight{.5in}
    \firstpageheader{\large\sf CS2105}%
    {\large\sf National University of Singapore\\ School of Computing \\
    \LARGE\sf Problem Set 5}%
    {\large\sf Semester 2 13/14}
    \firstpageheadrule
    \pagestyle{headandfoot}

    \begin{questions}
	\question \textbf{(KR, Chapter 4, P12)}
	Consider a datagram newtork using 8-bit host addresses. Suppose a 
	router uses longest prefix matching and has the follwoing forwarding
	table:

	\begin{center}
	\begin{tabular}{cc}
			Prefix Match & Interface\\
			\hline
			11 & 0\\
			101 & 1\\
			100 & 2\\
			otherwise & 3\\
			\hline
	\end{tabular}
	\end{center}

	For each of the four interfaces, give the associated range of 
	destination host addresses and the number of addresses in the range.

	\begin{solution}

			0: 1100 0000 - 1111 1111\\
			1: 1010 0000 - 1011 1111\\
			2: 1000 0000 - 1001 1111\\
			3: 0000 0000 - 0111 1111\\

	\end{solution}

	\question \textbf{(KR, Chapter 4, P16)}
	Consider a subnet with prefix 192.168.56.128/26.  Give the
	range of IP addresses (of form xxx.xxx.xxx.xxx) that can
	be assigned to this network.

	Suppose an ISP owns the block of addresses of the form
	192.168.56.128/26.  Suppose it wants to create four subnets
	from this block, with each block having the same number of
	IP addresses.  What are the prefixes (of form a.b.c.d/x)
	for the four subnets?

	\begin{solution}
		192.168.56.128 - 192.168.56.191.

		192.168.56.128/28, 192.168.56.134/28, 192.168.56.140/28, 192.168.56.156/28
	\end{solution}

	\question 
	Two hosts $A$ and $B$ participate in a peer-to-peer file
	sharing application and need to connect to each other.
	Both $A$ and $B$, however, are behind NATs. 

	Devise a technique
	that will allow $A$ to establish a TCP connection with
	$B$ without application-specific NAT configuration, if  
	\begin{parts}
	\part
	the NAT router uses a simple, predictable, algorithm 
	to allocate a public port number for mapping to the local/private
	port number.  
	\part
	there is a third host, $C$, participating in the
	same file sharing session, that is not behind NAT.  
	\end{parts}

	\begin{solution}
		First, make sure students understand the concept of NAT
		and why it is non-trivial for $A$ And $B$ to connect to
		each other.

		Then, introduces students to the concept of 
		``TCP hole punching.''  If the mapping is predictable
		(e.g., use the private port as public port), then an
		external client can still connect to the server, after
		the server initiates a connection outside.  Alternatively,
		if there is a third part, both $A$ and $B$ can connects
		to the third party.  This third party can then exchange
		port information of $A$ and $B$ for them to connect to
		each other.

	\end{solution}


	\question \textbf{(CS2105 Final Exam, April 2006)}
	The following diagram shows a simple network topology with 4 nodes.
	The links in the diagram are labeled with the cost of each link. 
	The nodes run distance vector routing protocol.  The protocol has
	terminated, and each node knows the cost of the minimum cost path
	to every other node.

		\vspace{2mm}
    \centerline{
        \includegraphics[scale=.55]{dv}
    }

\begin{parts}

\part The following table shows an incomplete routing table at node $x$.
Fill in the missing distance vector for $x$ and $z$.

\vspace{0.5cm}
\centerline{
\begin{tabular}{|r|c|c|c|c|}
\hline
& cost to $w$ & cost to $x$ & cost to $y$ & cost to $z$ \\
\hline
from $x$ &  & 0 &  &  \\
\hline
from $y$ &  &   & 0 &  \\
\hline
from $z$ &  &   &  &  0 \\
\hline
\end{tabular}
}
\vspace{0.5cm}

\part Now, suppose the cost of the link between $x$ and $w$ 
increases from 3 to 20.  Node $x$ detects the changes in the 
cost.  Before $x$ receives any new distance vector 
from its neighbors, triggered by these changes, $x$ recomputes its 
new minimum-cost path to $w$, $y$, and $z$, and updates its distance
vector.

Suppose that poisoned reverse is NOT used.  
What is the new computed cost from $x$ to $w$?


\part Suppose that poisoned reverse is used.
What is the new computed cost from $x$ to $w$?

\end{parts}

\begin{solution}
(a) from x to w:3, y:3, z:5\\
from y to w:6, x:3, z:2\\
from z to w:8, x:5, y:2\\

(note: the distance vector of w is not shown in table table, but exists in x)

(b) 9 (to y)\\

(c) 20 (to w)\\

Although the exam question ends here, you can continue to show what happen next if time allows.  Once x informs y of its new distance vector, y will update its least cost path to w as 7.  Next time the distance vector of y is sent to x, x will update its least cost path to w as 10 (the correct answer).

\end{solution}

\end{questions}
\end{document}
