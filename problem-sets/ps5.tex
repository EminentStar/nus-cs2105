\documentclass[a4paper,11pt]{exam}
    \usepackage{hyperref}
    \ifx\pdftexversion\undefined
       \usepackage{epsfig}
    \else
       \usepackage[pdftex]{graphics}
    \fi
    \usepackage{epsfig}
\begin{document}
    \extraheadheight{.5in}
    \firstpageheader{\large\sf CS2105}%
    {\large\sf National University of Singapore\\ School of Computing \\
    \LARGE\sf Problem Set 5}%
    {\large\sf Semester 2 12/13}
    \firstpageheadrule
    \pagestyle{headandfoot}

    \begin{questions}
	\question \textbf{(KR, Chapter 4, P12)}
	Consider a datagram newtork using 8-bit host addresses. Suppose a 
	router uses longest prefix matching and has the follwoing forwarding
	table:

	\begin{center}
	\begin{tabular}{cc}
			Prefix Match & Interface\\
			\hline
			11 & 0\\
			101 & 1\\
			100 & 2\\
			otherwise & 3\\
			\hline
	\end{tabular}
	\end{center}

	For each of the four interfaces, give the assocaited range of 
	destination host addresses and the number of addresses in the range.

	\question \textbf{(KR, Chapter 4, P16)}
	Consider a subnet with prefix 192.168.56.128/26.  Give the
	range of IP addresses (of form xxx.xxx.xxx.xxx) that can
	be assigned to this network.

	Suppose an ISP owns the block of addresses of the form
	192.168.56.128/26.  Suppose it wants to create four subnets
	from this block, with each block having the same number of
	IP addresses.  What are the prefixes (of form a.b.c.d/x)
	for the four subnets?

	\question \textbf{(KR, Chapter 4, P23)}
	In this problem we will explore the impact of NATs on P2P
	application.
	Suppose a peer with username Arnold
	discovers through querying that a peer with username Bernard
	has a file it wants to download.  Also suppose that both
	Arnold and Bernard are behind NATs.  Try to devise a technique
	that will allow Arnold to establish a TCP connection with
	Bernard without application-specific NAT configuration.  If
	you have difficulty devising such a technique, discuss why.

	\question \textbf{(CS2105 Final Exam, April 2006)}
	The following diagram shows a simple network topology with 4 nodes.
	The links in the diagram are labeled with the cost of each link. 
	The nodes run distance vector routing protocol.  The protocol has
	terminated, and each node knows the cost of the minimum cost path
	to every other node.

    \centerline{
        \includegraphics[scale=.6]{dv}
    }

\begin{parts}

\part The following table shows an incomplete routing table at node $x$.
Fill in the missing distance vector for $x$ and $z$.

\vspace{0.5cm}
\centerline{
\begin{tabular}{|r|c|c|c|c|}
\hline
& cost to $w$ & cost to $x$ & cost to $y$ & cost to $z$ \\
\hline
from $x$ &  & 0 &  &  \\
\hline
from $y$ &  &   & 0 &  \\
\hline
from $z$ &  &   &  &  0 \\
\hline
\end{tabular}
}
\vspace{0.5cm}

\part Now, suppose the cost of the link between $x$ and $w$ 
increases from 3 to 20.  Node $x$ detects the changes in the 
cost.  Before $x$ receives any new distance vector 
from its neighbors, triggered by these changes, $x$ recomputes its 
new minimum-cost path to $w$, $y$, and $z$, and updates its distance
vector.

Suppose that poisoned reverse is NOT used.  
What is the new computed cost from $x$ to $w$?


\part Suppose that poisoned reverse is used.
What is the new computed cost from $x$ to $w$?

\end{parts}

\end{questions}
\end{document}
