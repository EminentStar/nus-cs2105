\documentclass[a4paper,11pt]{exam}
    \usepackage{hyperref}
    \ifx\pdftexversion\undefined
       \usepackage{epsfig}
    \else
       \usepackage[pdftex]{graphics}
    \fi
    \usepackage{epsfig}
\begin{document}
    \extraheadheight{.5in}
    \firstpageheader{\large\sf CS2105}%
    {\large\sf National University of Singapore\\ School of Computing \\
    \LARGE\sf Problem Set 1}%
    {\large\sf Semester 2 13/14}
    \firstpageheadrule
    \pagestyle{headandfoot}

	Note: We use b as the notation for bits, and B as the notation for bytes.  We define 1Kb as 1000 bits, 1 Mb as 1000 Kb, and 1 Gb as 1000 Mb (similarly for bytes).

\begin{questions}
\question \textbf{(KR, Chapter 1, Problem 6)} Consider two hosts, A and B,
connected by a single link of rate $R$ bps.  Suppose that the two hosts
are separated by $m$ meters, and suppose the propagation speed along the
link is $s$ meters/sec.  Host A is to send a packet of size $L$ bits to
Host B.
\begin{parts}
\part Express the propagation delay, $d_{prop}$, in terms of $m$ and $s$.
\part Determine the transmission time of the packet, $d_{trans}$, in
	terms of $L$ and $R$.
\part Ignoring processing and queueing delays, obtain an expression for
	the end-to-end delay.
\part Suppose Host A begins to transmit the packet at time $t$ = 0.  At
	time $t = d_{trans}$, where is the last bit of the packet?
\part Suppose $d_{prop}$ is greater than $d_{trans}$.  At time $t =
	d_{trans}$, where is the first bit of the packet?
\part Suppose $d_{prop}$ is less than $d_{trans}$.  At time $t =
	d_{trans}$, where is the first bit of the packet?
\part Suppose $s = 2.5 \times 10^8$, $L$ = 100 bits, and $R = 28$ kbps.
Find the distance $m$ so that $d_{prop}$ equals $d_{trans}$.
    \end{parts}

\question \textbf{(KR, Chapter 1, Problem 12)} A packet switch receives a packet $P$ and determines the outbound link to which the packet should be forwarded.  When $P$ arrives, one other packet is halfway done being transmitted on this outbound link and four other packets are waiting to be transmitted. Packets are transmitted in order of arrival.  Suppose all packets are 1,500 bytes and the link rate is 2 Mbps.  What is the queueing delay for the packet $P$?  

More generally, what is the queueing delay when all packets have length $L$, the transmission rate is $R$, $x$ bits of the currently-being-transmitted packet have been transmitted, and $n$ packets are already in the queue?
  

\question \textbf{(KR, Chapter 1, Problem 25)} 
Suppose two hosts, A and B, are separated by 100,000 kilometers
and are connected by a direct link of $R = 1$ Mbps.  Suppose the
propagation speed over the link is $2.5 \times 10^8$ meters/sec.
\begin{parts}
\part Calculate the bandwidth-delay product, $R \times d_{prop}$.
\part Consider sending a file of 400,000 bits from Host A to Host B.
Suppose the file is sent continuously as one big message.  What is the
maximum number of bits that will be in the link at any given time?
\part Provide an interpretation of the bandwidth-delay product.
\part What is the width (in meters) of a bit in the link?  Is it longer
than a soccer field?  (A standard soccer field is between 90 to 120 m long)
\part Derive a general expression for the width of a bit in terms of the
propagation speed $s$, the transmission rate $R$, and the length of the link $m$.
\end{parts}

\pagebreak

\question \textbf{(KR, Chapter 1, Problem 31)} 
In modern packet-switched networks, the source host segments
long, application-layer messages (for example, an image or a music file)
into smaller packets and sends the packets into the network.  The
receiver then reassembles the packets back  into the original message.
We refer to this process as \textit{message segmentation.}  (See 
Figure 1.27 in KR, page 103).  

Consider a message that is $7.5\times 10^6$ bits long that is to be sent
from a source to destination, through two packet switches.  Suppose each 
link in the figure is 1.5 Mbps.  Ignore propagation, queuing, and
processing delays.

\begin{parts}
\part Consider sending the message from source to destination
\textit{without} message segmentation.  How long does it take to move the
message from the source host to the first packet switch?  Keeping in mind
that each switch uses store-and-forward packet switching, what is the
total time to move the message from source host to destination host?

\part Now suppose that the message is segmented into 5,000 packets, with
each packet being 1,500 bits long.  How long does it take to move the
first packet from source host to the first switch?  When the first packet
is being sent from the first switch to the second switch, the second
packet is being sent from the source host to the second switch.  At what
time will the second packet be fully received at the first switch?

\part How long does it take to move the file from source host to
destination host when message segmentation is used?  Compare this result
with your answer in part (a) and comment.

\part Discuss the drawbacks of message segmentation.
\end{parts}
\end{questions}
\end{document}
