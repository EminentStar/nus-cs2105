\documentclass[a4paper,11pt]{exam}
    \usepackage{hyperref}
    \ifx\pdftexversion\undefined
       \usepackage{epsfig}
    \else
       \usepackage[pdftex]{graphics}
    \fi
    \usepackage{epsfig}
\begin{document}
    \extraheadheight{.5in}
    \firstpageheader{\large\sf CS2105}%
    {\large\sf National University of Singapore\\ School of Computing \\
    \LARGE\sf Problem Set 3}%
    {\large\sf Semester 2 13/14}
    \firstpageheadrule
    \pagestyle{headandfoot}

    \begin{questions}
	\question \textbf{(KR, Chapter 3, P8)} 
	Draw the FSM for the receiver side of protocol \texttt{rdt3.0}.

	\question \textbf{(KR, Chapter 3, P12)} 
	The sender side of \texttt{rdt3.0} simply ignores (that 
	is, take no action on) all received packets that are either 
	in error or have the wrong value in the \texttt{acknum} 
	field of an acknowledgement packet.  Suppose that in such 
	circumstances, \texttt{rdt3.0} were simply to retransmit 
	the current data packet.  Would the protocol still work?  
	(\textit{Hint}: Consider what would happen if there were 
	only bit errors; there are no packet losses but premature 
	timeouts can occur.  Consider how many times the $n$th packet 
	is sent).

	\question \textbf{(Modified from KR, Chapter 3, P13)} 
	Show that if the network connection between the sender and 
	receiver can reorder messages (that is, that two messages 
	propagating in the medium between the sender and receiver 
	can be reordered), then \texttt{rdt3.0} will not work correctly.  

%	\question \textbf{(KR, Chapter 3, Problem 13)} 
%	Consider a reliable data transfer protocol that uses \textit{only}
%	negative acknowledgements.  Suppose the sender sends data only
%	infrequently.  Would a NAK-only protocol be preferable to 
%	a protocol that uses ACKs?  Why?  
%	
%	Now suppose the sender has a lot of data to send and the 
%	end-to-end connection experiences few losses.  In this
%	second case, would a NAK-only protocol be preferable to a 
%	protocol that uses ACKs?  Why?

	\question \textbf{(KR, Chapter 3, P22)} 
	Consider the GBN (Go-Back-N) protocol with a sender window size of 4 and a sequence
	number range of 1,024. Suppose that at time $t$, the next in-order packet
	that the receiver is expecting has a sequence number of $k$. Assume that the
	medium does not reorder messages. Answer the following questions:
    \begin{parts}
	\part What are the possible sets of sequence numbers inside the sender's window
	at time $t$? Justify your answer.
	\part What are the possible values of the ACK field in all possible messages
	currently propagating back to the sender at time $t$? Justify your answer.
    \end{parts}

	\question \textbf{(KR, Chapter 3, P23)} 
	Consider the GBN and SR protocols. Suppose the sequence
	number space is of size $k$. What is the largest allowable sender window that
	will avoid the occurrence of problems such as that in Figure 3.27 for each of
	these protocols?

    \end{questions}
\end{document}
