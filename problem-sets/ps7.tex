\documentclass[a4paper,11pt,answers]{exam}
    \usepackage{hyperref}
    \ifx\pdftexversion\undefined
       \usepackage{epsfig}
    \else
       \usepackage[pdftex]{graphics}
    \fi
    \usepackage{epsfig}
\begin{document}
    \extraheadheight{.5in}
    \firstpageheader{\large\sf CS2105}%
    {\large\sf National University of Singapore\\ School of Computing \\
    \LARGE\sf Problem Set 7}%
    {\large\sf Semester 2 12/13}
    \firstpageheadrule
    \pagestyle{headandfoot}

    \begin{questions}
	\question 
	Compute the CRC for the bits 10101010 using the generator $G = 1001$.
	
	(Answer: 101)

	\question \textbf{(Modified from KR, Chapter 5, P7)}
	In this problem, we explore some of the properties of CRC.  
	For the generator $G = 1001$,
	\begin{parts}
			\part why can it detect any single bit error in data $D$? 
			\part can the above $G$ detect any odd number of bit errors? why? (Hint: any number with odd number of ones cannot be divisible by $11$).
	\end{parts}
	\begin{solution}
		\input{solutions/ps7q1}
	\end{solution}


	\question 
	Nodes $A$ and $B$ are accessing the same shared medium using CSMA/CD, 
	with a proagation delay of 245 bit times between them.  The minimum 
	frame size is 64 bytes.  Suppose node A begins transmitting a frame 
	and, before it finishes, node B begins transmitting a frame.  

	\begin{parts}
	\part What is the minimum possible time taken by A to finish transmission?
	\part When is the latest time, by which B can begin its transmission?
	\part Can A finish transmitting before it detects that B has transmitted? 
	\end{parts}

	Express all your answers above in the unit of bit time.
	
	\begin{solution}
	\input{solutions/ps7q2.tex}
	\end{solution}

	\question \textbf{(KR, Chapter 5, P19)}
	Suppose nodes A and B are on the same 10 Mbps Ethernet
	segment, and the propagation delay between two nodes is 245
	bit times.  Suppose A and B send frames at the same time,
	the frames collide, and then A and B choose different values
	of $K$ in the CSMA/CD algorithm.  Assuming no other nodes
	are active, can the retransmission from A and B collide?

	Work out the following example.  Suppose A and B begin
	transmission at $t = 0$ bit times.  They both detect
	collisions at $t = 245$ bit times.  Suppose $K_A =
	0$ and $K_B = 1$.  At what time does B schedule its
	retransmission?  At what time does A begin transmission?
	(Note that a node must wait for an idle channel after
	returning to Step 2 -- see protocol.)  At what time does
	A's signal reach B? Does B refrain from transmitting at its
	scheduled time?
    \end{questions}
	\begin{solution}
		\input{solutions/ps7q3}
	\end{solution}
\end{document}
