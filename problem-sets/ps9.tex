\documentclass[a4paper,11pt]{exam}
    \usepackage{hyperref}
    \ifx\pdftexversion\undefined
       \usepackage{epsfig}
    \else
       \usepackage[pdftex]{graphics}
    \fi
    \usepackage{epsfig}
\begin{document}
    \extraheadheight{.5in}
    \firstpageheader{\large\sf CS2105}%
    {\large\sf National University of Singapore\\ School of Computing \\
    \LARGE\sf Problem Set 9}%
    {\large\sf Semester 2 13/14}
    \firstpageheadrule
    \pagestyle{headandfoot}

\begin{questions}
\question 
For each encoding method below, show how the bit sequence \texttt{01011001} is encoded: RZ, NRZ-I, NRZ-L, Manchester, Differential Manchester.  
	
Assume that the signal for the first bit (i.e., bit 0) starts at positive value.

\question 
The following signal is received.  The first bit is 0.  Decode the bit sequence if the encoding scheme used is (i) NRZ-I; (ii) Manchester coding; and (iii) Differential Manchester coding.  

\begin{center}
	\includegraphics[scale=0.2]{signal-crop.pdf}
\end{center}

\question A given transmission medium has a SNR of 127 and supports frequency ranging from 1Mhz to 3MHz.  A signal is transmitted using the following modulation scheme:
	\[
	s(t) = \left\{\begin{array}{lr}
		5 \cos(2\pi ft + 45^\circ)  & 000\\
		5 \cos(2\pi ft + 135^\circ)  & 001\\
		5 \cos(2\pi ft + 225^\circ)  & 010\\
		5 \cos(2\pi ft + 315^\circ)  & 011\\
		10 \cos(2\pi ft + 45^\circ)  & 100\\
		10 \cos(2\pi ft + 135^\circ)  & 101\\
		10 \cos(2\pi ft + 225^\circ)  & 110\\
		10 \cos(2\pi ft + 315^\circ)  & 111\\
	\end{array}
		\right.
		\]
\begin{parts}
\part Draw the constellation for the modulation scheme above.
\part What is the theoretical maximum bit rate that can transmitted through the medium?  
\part What is the maximum baud rate achieved?
\part If the transmission medium is noiseless, what is the achievable bitrate?
\end{parts}
\question 

\end{questions}
\end{document}
