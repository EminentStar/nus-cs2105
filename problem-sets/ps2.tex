\documentclass[a4paper,11pt]{exam}
    \usepackage{hyperref}
    \ifx\pdftexversion\undefined
       \usepackage{epsfig}
    \else
       \usepackage[pdftex]{graphics}
    \fi
    \usepackage{epsfig}
\begin{document}
\extraheadheight{.5in}
\firstpageheader{\large\sf CS2105}%
{\large\sf National University of Singapore\\ School of Computing \\
\LARGE\sf Problem Set 2}%
{\large\sf Semester 2 10/11}
\firstpageheadrule
\pagestyle{headandfoot}

\begin{questions}

\question \textbf{(KR, Chapter 2, Problem 8)}
Suppose with your Web browser you click on a link to obtain a
Web page.  The IP address for the associated URL is not cached
in your local host, so a DNS look-up is necessary to obtain the
IP address.  Suppose that $n$ DNS servers are visited before your
host receives the IP address from DNS; the successive visits
incur an RTT of RTT$_1$, ..., RTT$_n$.  Further suppose that
the Web page associated with the link contains exactly one
object, consisting of a small amount of HTML text.  Let RTT$_0$
denote the RTT between the local host and the server containing
the object.  Assuming zero transmission time of the object, how
much time elapses from when the client clicks on the link until
the client receives the object?

\question \textbf{(KR, Chapter 2, Problem 9)}
Referring to the previous question, suppose the HTML file
references three very small objects on the same server.
Neglecting transmission time, how much time elapses with (a)
Non-persistent HTTP with no parallel TCP connections? (b)
Non-persistent HTTP with parallel connections? (c) Persistent
HTTP with pipelining?

\question 
Suppose a malicious party can intercept your HTTP GET requests (but not necessary your HTTP responses).  Explain how this could lead to undesirable circumstances.

\question
Suppose a malicious party can insert fake DNS records into DNS servers.  Suggest three different ways how this can be abused.

\end{questions}
\end{document}
