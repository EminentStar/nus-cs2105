\documentclass[a4paper,11pt]{exam}
    \usepackage{hyperref}
    \ifx\pdftexversion\undefined
       \usepackage{epsfig}
    \else
       \usepackage[pdftex]{graphics}
    \fi
    \usepackage{epsfig}
\begin{document}
    \extraheadheight{.5in}
    \firstpageheader{\large\sf CS2105}%
    {\large\sf National University of Singapore\\ School of Computing \\
    \LARGE\sf Midterm Review}%
    {\large\sf Semester 2 10/11}
    \firstpageheadrule
    \pagestyle{headandfoot}

\begin{questions}
\question
Consider the following Java code fragment.
\begin{verbatim}
new Socket("www.example.com", 80)
\end{verbatim}
Which of the following statements about the above code fragment are FALSE?
\renewcommand{\labelenumi}{(\roman{enumi})}
\begin{enumerate}
\item The code causes a DNS query for domain name www.example.com to be issued.
\item The code causes a SYN packet to be sent to host www.example.com.
\item The code causes a HTTP request to be sent to host www.example.com.
\item The code causes host www.example.com to open port 80 and listen for      
incoming connections.
\end{enumerate}

\begin{choices}
\choice (i) and (ii) only
\choice (ii) and (iii) only
\choice (iii) and (iv) only
\choice (i),(ii) and (iv) only
\choice (i),(iii) and (iv) only      
\end{choices}

\question
Consider the sender's view of the sequence numbers in \textbf{Go-Back-N}         protocol.
Suppose the first sequence number in the sender's window is $k$, and the last
sequence number in the sender's window is $k+3$.  Let a packet with sequence
number $i$ be $p_i$.  Which of the following MUST be TRUE?
\begin{choices}
\choice $p_k$ is sent and acknowledge.
\choice $p_{k+3}$ is not sent but is usable.
\choice If $p_{k+2}$ is sent, then $p_{k+1}$ must have been sent.
\choice If $p_{k+2}$ is not sent, then $p_{k+1}$ must not have been sent.
\choice The receiver is currently expecting $p_k$.
\end{choices}

\question
Consider a sender and a receiver, communicating using \textbf{Selective-Repeat}
protocol.  The sender just sent a packet with sequence number 10.  The window
size is unknown.  Which of the following CANNOT possibly be the sequence number
of the next packet transmitted by the sender?
\begin{choices}
\choice 0
\choice 9
\choice 10
\choice 11
\choice 12
\end{choices}

\question
Consider the checksum algorithm used to compute the checksum field in UDP   header and TCP header.  Although UDP and TCP uses 16-bits words in computing the checksum, in this  question you are asked to consider 8-bits summands.

Suppose a receiver received the following four 8-bit bytes over a           communication channel that might introduce errors.
\begin{verbatim}
1101 1011
1101 0001
1001 1011
1011 0110
\end{verbatim}
One of the given byte is the checksum (but I am not telling you which one), and the other three bytes is the data used to compute the checksum.  Which of the         following MUST BE TRUE?
								          \begin{choices} 
\choice There are no errors in the received bytes.
\choice There is a one bit error in the received bytes.
\choice There are two bits error in the received bytes.
\choice The last byte \texttt{1011 0110} must be the checksum.
\choice The first byte \texttt{1101 1011} must be the checksum.
\end{choices}

\question
Consider the following abbreviated output from the command
\begin{verbatim}
    dig a3.nstld.com +trace
\end{verbatim}
    issued from within NUS. 
\begin{verbatim}
; <<>> DiG 9.2.2 <<>> a3.nstld.com +trace
;; global options:  printcmd
.                       162410  IN      NS      M.ROOT-SERVERS.NET.
.                       162410  IN      NS      A.ROOT-SERVERS.NET.
.                       162410  IN      NS      B.ROOT-SERVERS.NET.
.                       162410  IN      NS      C.ROOT-SERVERS.NET.
;; Received 436 bytes from 137.132.90.2#53(137.132.90.2) in 27 ms

com.                    172800  IN      NS      C.GTLD-SERVERS.NET.
com.                    172800  IN      NS      D.GTLD-SERVERS.NET.
com.                    172800  IN      NS      E.GTLD-SERVERS.NET.
com.                    172800  IN      NS      F.GTLD-SERVERS.NET.
;; Received 490 bytes from 202.12.27.33#53(M.ROOT-SERVERS.NET) in 296 ms

a3.nstld.com.           172800  IN      A       192.5.6.32
nstld.com.              172800  IN      NS      a2.nstld.com.
nstld.com.              172800  IN      NS      c2.nstld.com.
nstld.com.              172800  IN      NS      d2.nstld.com.
;; Received 277 bytes from 192.26.92.30#53(C.GTLD-SERVERS.NET) in 246 ms
\end{verbatim}
	
Give the IP address of
\begin{parts}
\part a root DNS server\\
\part a top-level domain DNS server \\
\part local DNS server\\
\part host a3.nstld.com\\
\end{parts}

\question
Suppose we want to design a stop-and-wait, reliable protocol for communication 
between a sender S and a receiver R, over a channel with the following 
characteristics.
\begin{itemize}
	\item Packets can be lost from S to R.
	\item The channel from R to S is reliable.
	\item Neither channel will corrupt a packet (it a packet is received, it
	      must be correct)
	\item Neither channel will reorder packets.
	\item The maximum RTT between S and R is known and is guaranteed to be $D$.
	\end{itemize}

\begin{parts}
\part For each of the following techniques you learnt in class for building
reliable protocol, comment on whether or not, it is needed in the design
of the protocol.  Justify your answer.
(i) Acknowledgement
(ii) Negative Acknowledgement
(iii) Timeout
(iv) Sequence Number

\part Using \textit{only} the necessary techniques above, design
a stop-and-wait protocol for reliable communication between S and R 
through the channel with the specified characteristics.  Show your 
design by drawing the FSM for the sender side and receiver side of 
the protocol.  You can use either the C-like notation in the textbook
or the pseudocode notation shown in lecture to describe the events and
actions in your FSM. 
	
Note that for full credit, your protocol should not only be correct, 
but should also be as simple as possible.
\end{parts}

\question
The following figure shows two hosts X and Y communicating over 
a channel using TCP.  X and Y are sending data to each other
(recall that TCP supports duplex communications).  Each segments
contain 100 bytes of data.  None of the segments shown in the 
figure are retransmitted packets, and the second segment send by 
X is lost.

The sequence numbers ($S$) and acknowledgement numbers ($A$) for 
some segments (indicated by thicker line) are missing.  Complete 
the figure below by filling in the missing sequence numbers and 
acknowledgement numbers.

\centerline{
        \includegraphics[scale=0.7]{tcp.eps}
}


\question
Two hosts A and B are connected by a link with bandwidth-delay 
product of $P$ bits and transmission rate of $R$ bps.  A sends 
a packet of size $L$ bits to B.

\begin{parts}
\part 
Let $T$ be the time between the first bit of the packet leaving A and the 
last bit of the packet arriving at B.  Express $T$ in terms of $P$, $R$ and $L$.

\part
Suppose A and B are using alternating-bit protocol.  Assuming no packets are 
lost, express the link utilization in terms of $P$, $R$, and $L$.

\end{parts}
\end{questions}
\end{document}
