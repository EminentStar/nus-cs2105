\documentclass[a4paper,11pt]{exam}
\usepackage{hyperref}
\ifx\pdftexversion\undefined
\usepackage{epsfig}
\else
\usepackage[pdftex]{graphics}
\fi
\usepackage{epsfig}

\usepackage{fancyvrb}

%\rhead{\textsf{CS2106 Lab 1}}
%\lhead{\textsf{Matriculation Number: \underline{\hspace{1.2in}}}}
%\headrule

\begin{document}
\extraheadheight{.5in}
\firstpageheader{\large\sf CS2105}%
{\large\sf National University of Singapore\\ School of Computing \\
\LARGE\sf DIY Exercise 1}%
{\large\sf Semester 2 10/11}
\firstpageheadrule
\pagestyle{headandfoot}

You may run the following tools on Solaris (e.g., SunFire), Linux, Mac OS X, or Microsoft Windows.  The behaviour may be slightly different (but should not impact your learning).

To access the SunFire server, use any ssh client and ssh into:
\texttt{sunfire.comp.nus.edu.sg}.   Use your SoC UNIX username and password to login.  If you do not have an account on SunFire, apply for one here: \url{http://mysoc.comp.nus.edu.sg/~newacct}.

\begin{center}
	\textbf\textsf{HAVE FUN, AND BE CURIOUS}
\end{center}

\begin{questions}
\question{netstat}

\texttt{netstat} is a command line tool that displays various information about your computer's network stack.

Find out what are some open sockets on your computer.  Can you recorgnize some of the establish connections?  Do you see some sockets that are in listening state?
\begin{itemize}
\item If you are running Windows or Linux, type the command \texttt{netstat -a}.
\item If you are running Mac OS X, type the command \texttt{netstat -a -f inet}.
\end{itemize}


\question{traceroute}

\texttt{traceroute} is a command line tool that prints the route a packet takes to go from the current host to a destination.

On your computer, run traceroute to \texttt{mail.google.com}.  Repeat several times at different time of the day.  Do you see the same results?

(Note: the equivalent of \texttt{traceroute} utility on Windows is \texttt{tracert}.)

(Note: the utility \texttt{traceroute} is located under \texttt{/usr/sbin} on SunFire.  To run \texttt{traceroute}, you need to either add \texttt{/usr/sbin} to your \texttt{PATH} environment variable, or type the full path everytime.)


\question{dig}

Use \texttt{dig} to answer the following questions.  You may have to download and install \texttt{dig} if it is not already available.  Dig for Windows is available here \url{http://members.shaw.ca/nicholas.fong/dig/}.

\begin{parts}
	\part Find the IP addresses of \texttt{mail.google.com}.  Repeat several time and observe the changes.  Why do you think different answers are given out at different time?

	\part Use the dig server at URL \url{http://www.kloth.net/services/dig.php} (using your Web browser) and find the IP addresses of \texttt{mail.google.com}. Do you get the same results as part (b)?  Why do you think that the same hostnme provides a different answer when \texttt{dig} is run from different places?

	\part Find the TTL value for the A-type DNS record of the following hosts at their authoritative DNS server: (a) \texttt{sentosa.comp.nus.edu.sg}, (b) \texttt{delco0.ddns.comp.nus.edu.sg}.  What do you think there are differences in the TTL of these records?  (Hint: ddns stands for Dynamic DNS.)

\part Observe the TTL values for DNS record of \texttt{www.metacafe.com}.  Why do you think the TTL values are so small? (Hint: the DNS servers are provided by Akamai. Google to find out how Akamai works)

\part Wikileaks was ``shut down'' in December 2010 when its DNS service provider removed its mapping from its server.  As a result, users can only access the Wikileaks website through its IP address and not through hostname.  The site has since been up with the domain name \texttt{wikileaks.ch}.  Find out who is providing the DNS server (i.e., authoritative DNS servers) for the domain \texttt{wikileaks.ch} now.

\part The site Torrent-Finder was ``shut down'' by U.S. goverment in November 2010 by forcing all lookups to \texttt{www.torrent-finder.com} to map to an IP address of a U.S. goverment website.  Do an interative DNS query (with \texttt{+trace} option to \texttt{dig}) to find out who are the authorative DNS severs for the domain \texttt{torrent-finder.com}.
\end{parts}

\question{whois}

The tool \texttt{whois} queries domain name databases and provides information about the registrar of a domain name, contact of the organization/person responsible, domain name servers, etc.  The tool should be available on UNIX and Mac OS X.  You may need to download Windows version of whois here: \url{http://technet.microsoft.com/en-us/sysinternals/bb897435}.

Find out who is the domain registrar for the domain \texttt{nus.edu.sg} and \texttt{google.com} by typing: \texttt{whois <domain name>}.  For the latter, observe how hackers exploit the WHOIS database for advertisement.

\end{questions}

\vfill
\begin{center}
    \textsf{\Huge THE END}
\end{center}
\end{document}
