\begin{cf}{
CS2105 Lecture 11\\
\large
\textbf{Physical Layer\\[10pt]}
\normalsize
7 April, 2014
\note{
\tiny
	After this class, you are expected to be able to understand:
	\begin{itemize}
        \setlength{\itemsep}{1pt}
	\item how NRZ, RZ, Manchester, and DM, are used to encode 0s and 1s, and their pros and cons.
	\item how $A$, $f$ and $\phi$ can be used to encode 0s and 1s, and their pros and cons.
	\item the term bandwidth and the theoretical capacabity of a medium using Nyquist's and Shannon's formula.
	\item how a signal can be viewed in frequency domain and how frequency can be shifted to multiplex multiple signals.
	\item how QAM works and its representation as a constellation.
	\end{itemize}
}
}
\end{cf}

\begin{cf}{
	\begin{tikzpicture}[scale=2, start chain=going below]
	\foreach \color / \label in {
		AppBlue/Application,
		TransportBlue/Transport,
		NetworkGreen/Network,
		LinkBrown/Link,
		PhysicalRed/Physical
	}
	\node[draw,rectangle,
	minimum height=1.5 cm, 
	minimum width=5 cm, 
	text=white, 
	node distance=0cm,
	on chain, fill=\color]{\label};
	\end{tikzpicture}
}
\end{cf}

\begin{cf}{\small
	\begin{itemize}
	\item 0s and 1s can be transmitted either as digital signal or analog signal over a medium.
	\item WiFi uses analog signal.  Ethernet uses digital signal.
	\end{itemize}
}
\end{cf}
	
\begin{cf}{
	\begin{tikzpicture}
	\path[draw,->] (0,0) -- (7, 0);
	\path[draw,->] (0.5,-1) -- (0.5, 1);
	\path[draw,->] (0,-4) -- (7, -4);
	\path[draw,->] (0.5,-5) -- (0.5, -3);
	\end{tikzpicture}

}
\end{cf}

\begin{cf}{
	Digital Encoding
}
\end{cf}

\begin{cf}{\small
	\begin{itemize}
	\item A digital signal has a limited number of defined values (e.g., -1, 0, and 1 only)
	\item The value of a digital signal is determined by the voltage sent over the wire.
	\item Polar encoding uses two levels: -1 and 1.
	\item Bipolar encoding uses three levels: -1, 0, and 1.
	\end{itemize}
}
\end{cf}

\begin{cf}{\small
	\begin{itemize}
	\item 0s and 1s can be encoded into digital signals in different ways.
	\item \textbf{NRZ} encoding is a polar encoding scheme.
	\item Two variants: \textbf{NRZ-L} encodes the bit value using the level of the signal;
	\textbf{NRZ-I} inverts the signal if bit 1 is encountered.
	\end{itemize}
}
\end{cf}

\begin{cf}{
	NRZ-L
	\vspace{1cm}

	\begin{tikzpicture}
	\path[draw,->] (0,0) -- (7, 0);
	\path[draw,->] (0.5,-1) -- (0.5, 1);
	\foreach \x in {1.5, 2.5, 3.5, 4.5, 5.5, 6.5} 
		\path[draw,-,color=gray!50] (\x,-1) -- (\x, 1);
	\end{tikzpicture}
}
\end{cf}

\begin{cf}{
	NRZ-I
	\vspace{1cm}

	\begin{tikzpicture}
	\path[draw,->] (0,0) -- (7, 0);
	\path[draw,->] (0.5,-1) -- (0.5, 1);
	\foreach \x in {1.5, 2.5, 3.5, 4.5, 5.5, 6.5} 
		\path[draw,-,color=gray!50] (\x,-1) -- (\x, 1);
	\end{tikzpicture}
}
\end{cf}

\begin{cf}{\small
	\begin{itemize}
	\item \textbf{RZ} is a bi-polar encoding scheme, always returning the signal to zero halfway through the bit interval.
	\item It allows synchronization of clock at the sender and receiver, without which clock drift could lead to bit errors.
	\item E.g., if the sender is sending 111111..., the receiver may lost track of how many 1s have been received.  This is known as a \textbf{bit slip}.
	\end{itemize}
}
\end{cf}

\begin{cf}{
	RZ
	\vspace{1cm}

	\begin{tikzpicture}
	\path[draw,->] (0,0) -- (7, 0);
	\path[draw,->] (0.5,-1) -- (0.5, 1);
	\foreach \x in {1.5, 2.5, 3.5, 4.5, 5.5, 6.5} 
		\path[draw,-,color=gray!50] (\x,-1) -- (\x, 1);
	\end{tikzpicture}
}
\end{cf}

\begin{cf}{\small
	\begin{itemize}
	\item \textbf{Machester} coding inverts the signal in the middle of a bit.  A -ve to +ve trasition represents 1.  A +ve to -ve trasition represents 0.
	\item \textbf{Differential Manchester (DM)} uses the presence of absence of a transition at the beginning of the interval to identify a bit.  A transition means 0.  No transition means 1.
	\end{itemize}
}
\end{cf}

\begin{cf}{
	Manchester
	\vspace{1cm}

	\begin{tikzpicture}
	\path[draw,->] (0,0) -- (7, 0);
	\path[draw,->] (0.5,-1) -- (0.5, 1);
	\foreach \x in {1.5, 2.5, 3.5, 4.5, 5.5, 6.5} 
		\path[draw,-,color=gray!50] (\x,-1) -- (\x, 1);
	\end{tikzpicture}
}
\end{cf}

\begin{cf}{
	Differential Manchester
	\vspace{1cm}

	\begin{tikzpicture}
	\path[draw,->] (0,0) -- (7, 0);
	\path[draw,->] (0.5,-1) -- (0.5, 1);
	\foreach \x in {1.5, 2.5, 3.5, 4.5, 5.5, 6.5} 
		\path[draw,-,color=gray!50] (\x,-1) -- (\x, 1);
	\end{tikzpicture}
}
\end{cf}

\begin{cf}{\small
	\begin{itemize}
	\item Detecting transition is less error prone than comparing against a threshold.
	\item DM works even if the signals are swapped.
	\end{itemize}
}
\end{cf}

%\begin{cf}{
%		\textbf{Final Exam}\\
%		\vspace{1cm}
%		\textbf{10} MCQs (A,B,C,D,E,X)\\
%		\textbf{5} Short Questions
%}
%\end{cf}
%
%\begin{cf}{
%		\textbf{Scope}\\
%		\vspace{1cm}
%		Lecture 1-12\\
%		Problem Set 1-9\\
%		DIY Exercises 1-4\\
%		Assignment 1-3\\
%		\tiny{(less emphasis on the topics covered in midterm)}
%}
%\end{cf}

%\begin{cf}{
%	One\\
%	A4-size\\
%	double-sided\\
%	crib sheet\\
%	allowed
%}
%\end{cf}

\begin{cf}{
	Digital-to-Analog Modulation
}
\end{cf}

\begin{cf}{\small
	\begin{itemize}
	\item The most basic analog signal is a sine wave:
	\[
		A sin(2\pi ft + \phi)
	\]
	where $A$ is the peak amplitude, $f$ is the frequency and $\phi$ is the phase.
	\end{itemize}
}
\end{cf}

\begin{cf}{
	\begin{tikzpicture}
	\path[draw,->] (0,0) -- (7, 0);
	\path[draw,->] (0.5,-1) -- (0.5, 1);
	\path[draw,->] (0,-4) -- (7, -4);
	\path[draw,->] (0.5,-5) -- (0.5, -3);
	\end{tikzpicture}

}
\end{cf}

\begin{cf}{\small
	\begin{itemize}
	\item We can combine sine waves to form composite signals.
	\item Composite signal can be decomposed to multiple sine waves.
	\item It is useful to visualize a composite signal using the frequency of its composition (i.e., in \textbf{frequency domain} instead of time domain).
	\end{itemize}
}
\end{cf}

\begin{cf}{
	\begin{tikzpicture}
	\path[draw,->] (0,0) -- (7, 0);
	\path[draw,->] (0.5,-1) -- (0.5, 1);
	\end{tikzpicture}

}
\end{cf}

\begin{cf}{\small
	\begin{itemize}
	\item A transmission medium only allows a frequency range to pass through.
	\item The difference in highest and lowest frequency that can pass through a medium is kown as the \textbf{bandwidth of the medium}.
	\item The difference in the highest and lowest frequency that represent a signal is known as the \textbf{bandwidth of the signal}.
	\end{itemize}
}
\end{cf}

\begin{cf}{\small
	\begin{itemize}
	\item A transmission medium also introduces noise, which distorts the signal, limiting the number of bits that can go through.
	\item For an ideal, noiseless channel, the \textbf{Nyquist bit rate formula} gives the theorectical maximum bit rate:
	\[
	2 B \times log_2 L
	\]
	where $B$ is the channel bandwidth (in Hz) and $L$ is the number of signal levels.
	\end{itemize}
}
\end{cf}

\begin{cf}[t]{\small
	If we use Machester coding on a noiseless 1 MHz channel, the maximum data rate is 2 Mbps.
}
\end{cf}

\begin{cf}{\small
	\begin{itemize}
	\item For a noisy channel, characterized by a signal-to-noise ratio $SNR$, the theoretical maximum is given by \textbf{Shannon Capacity}
	\[
	B \times log_2 ( 1 + SNR )
	\]
	\end{itemize}
}
\end{cf}

\begin{cf}[t]{\small
	Phone line has a bandwidth of 3000 Hz and SNR of 3162.  The capacity of the channel is 34860 bps.
}
\end{cf}

\begin{cf}{\small
	\begin{itemize}
	\item Let's revisit how FDM works on a medium with large enough bandwidth to support multiple signals.
	\item The signals' frequencies are shifted, added together, and transmitted.
	\item At the receiver, we filter out different frequency range, and shift the signal back to their original frequency.
	\end{itemize}
}
\end{cf}

\begin{cf}{
	Frequency-Division Multiplexing\\
	\vspace{1cm}
	\begin{tikzpicture}[start chain=going below,node distance=0cm,minimum height=1cm, minimum width=6cm]
	\node[draw,rectangle,on chain,fill=red!30!white](){4-8kHz};
	\node[draw,rectangle,on chain,fill=yellow!30!white](){8-12kHz};
	\node[draw,rectangle,on chain,fill=green!30!white](){12-16kHz};
	\node[draw,rectangle,on chain,fill=blue!30!white](){16-20kHz};
	\end{tikzpicture}
}
\end{cf}

\begin{cf}[t]{
	FDM
}
\end{cf}

\begin{cf}{\small
	\begin{itemize}
	\item To transmit 0s and 1s with an analog signal, we can change either $A$, $f$, or $\phi$.
	\item Amplitude Shift Keying (ASK) changes peak amplitude to represent 0s and 1s.
	\item Frequency Shift Keying (FSK) changes frequency to represent 0s and 1s.
	\item Phase Shift Keying (PSK) changes phase to represent 0s and 1s.
	\end{itemize}
}
\end{cf}

\begin{cf}{
	ASK
	\vspace{1cm}

	\begin{tikzpicture}
	\path[draw,->] (0,0) -- (7, 0);
	\path[draw,->] (0.5,-1) -- (0.5, 1);
	\foreach \x in {2.5, 4.5, 6.5} 
		\path[draw,-,color=gray!50] (\x,-1) -- (\x, 1);
	\end{tikzpicture}
}
\end{cf}

\begin{cf}{
	FSK
	\vspace{1cm}

	\begin{tikzpicture}
	\path[draw,->] (0,0) -- (7, 0);
	\path[draw,->] (0.5,-1) -- (0.5, 1);
	\foreach \x in {2.5, 4.5, 6.5} 
		\path[draw,-,color=gray!50] (\x,-1) -- (\x, 1);
	\end{tikzpicture}
}
\end{cf}

\begin{cf}{\small
	\begin{itemize}
	\item ASK is vulnerable to noise.
	\item FSK is limited by bandwidth
	\end{itemize}
}
\end{cf}

\begin{cf}{
	PSK
	\vspace{1cm}

	\begin{tikzpicture}
	\path[draw,->] (0,0) -- (7, 0);
	\path[draw,->] (0.5,-1) -- (0.5, 1);
	\foreach \x in {2.5, 4.5, 6.5} 
		\path[draw,-,color=gray!50] (\x,-1) -- (\x, 1);
	\end{tikzpicture}
}
\end{cf}

\begin{cf}{
	PSK constellation
	\vspace{1cm}

	\begin{tikzpicture}
	\path[draw,->] (-2,0) -- (2, 0);
	\path[draw,->] (0,-2) -- (0,2);
	\end{tikzpicture}
}
\end{cf}

\begin{cf}{
	QPSK constellation
	\vspace{1cm}

	\begin{tikzpicture}
	\path[draw,->] (-2,0) -- (2, 0);
	\path[draw,->] (0,-2) -- (0,2);
	\end{tikzpicture}
}
\end{cf}

\begin{cf}{\small
	\begin{itemize}
	\item Quadrature Amplitude Modulation (QAM) combines ASK and PSK.  Many combinations are possible.
	\item A signal unit in a $2^k$-QAM scheme is a combination of amplitude and phase that represents $k$ bits.
	\item Baud rate is the number of signal units per second (unit is Bd).
	\end{itemize}
}
\end{cf}


\begin{cf}{
	4-QAM
	\vspace{1cm}

	\begin{tikzpicture}
	\path[draw,->] (-2,0) -- (2, 0);
	\path[draw,->] (0,-2) -- (0,2);
	\end{tikzpicture}
}
\end{cf}

\begin{cf}{
	8-QAM
	\vspace{1cm}


	\begin{tikzpicture}
	\path[draw,->] (-2,0) -- (2, 0);
	\path[draw,->] (0,-2) -- (0,2);
	\end{tikzpicture}
}
\end{cf}

\begin{cf}{
	8-QAM
	\vspace{1cm}

	\begin{tikzpicture}
	\path[draw,->] (0,0) -- (7, 0);
	\path[draw,->] (0.5,-1) -- (0.5, 1);
	\foreach \x in {2.5, 4.5, 6.5} 
		\path[draw,-,color=gray!50] (\x,-1) -- (\x, 1);
	\end{tikzpicture}
}
\end{cf}

\begin{cf}{\small
	\begin{itemize}
	\item Singapore TV broadcast uses DVB-T, which uses QPSK, 16-QAM, or 64-QAM.
	\item 802.11a uses BPSK, QPSK, 16-QAM or 64-QAM.
	\item Ethernet, RFID, and NFC use Machester coding.
	\item USB uses NRZ-I.
	\end{itemize}
}
\end{cf}
