\begin{cf}{
CS2105 Lecture 9\\
\large
\textbf{Link Layer\\[10pt]}
\normalsize
	25 March, 2013
\note{
\tiny
	After this class, you are expected to be able to understand:
	\begin{itemize}
	\item how the role of link layer and the services it could provide.
	\item how parity and CRC schemes work
	\item different methods for accessing shared medium
	\item how ALOHA, Slotted ALOHA, CSMA, and CSMA/CD works
	\item how the DOCSIS standard incorperate different medium access techniques
	\item the framing of an Ethernet frame 
	\end{itemize}
}
}
\end{cf}

\begin{frame}
\begin{center}
\tikzstyle{layer}=[draw,rectangle,minimum height=1.5 cm, minimum width=5 cm, text=white]
	\begin{tikzpicture}[scale=2]
	\node[layer](A)[fill=AppBlue]{Application};
	\node[layer](T)[below = 0cm of A,fill=TransportBlue]{Transport};
	\node[layer](N)[below = 0cm of T,fill=NetworkGreen]{Network};
	\node[layer](L)[below = 0cm of N,fill=LinkBrown]{Link};
	\node[layer](P)[below = 0cm of L,fill=PhysicalRed]{Physical};
	\end{tikzpicture}
\end{center}
\end{frame}

\begin{cf}{
\tikzstyle{packet}=[draw,rectangle,minimum height=.7cm]
\tikzstyle{layer}=[draw,rectangle,minimum height=2.5 cm, minimum width=9 cm, text=white]
\tikzstyle{interface}=[draw,single arrow,fill=white, minimum width=0.9cm, minimum height=1.2cm]
\tikzstyle{message}=[packet,minimum width=1cm,fill=white!90!yellow]
\tikzstyle{datagram payload}=[packet,minimum width=1cm,fill=white!90!yellow]
\tikzstyle{datagram header}=[packet,minimum width=.5cm,fill=white!90!blue]
	\begin{tikzpicture}[scale=2]
	\node[layer](A)[
		fill=NetworkGreen,
		]{};
	\node[layer](T)[
		below = 0cm of A,
		yshift = -0.1cm, 
		fill=LinkBrown]{};
    \node[message](M1)[below=0.5cm of A.north]{};
    \node[message](M2)[left=0.5cm of M1]{};
    \node[message](M3)[right=0.5cm of M1]{};
	\node[layer](N)[
		below = 0cm of T,
		yshift = -0.1cm, 
		fill=PhysicalRed]{\small 101011001...};
	\node[interface,
		below=of A.south, 
		xshift=-0.5cm, 
		yshift=1cm, 
		rotate=270](FirstArrow){};
	\node[interface,
		below=of A.south, 
		xshift=0.5cm, 
		yshift=1cm, 
		rotate=90]{};
    \node[datagram header](D1)[below=1cm of FirstArrow, xshift=-2cm]{};
    \node[datagram payload](D2)[right=0cm of D1]{}; 
    \node[datagram header](D3)[right=0.5cm of D2]{};
    \node[datagram payload](D4)[right=0cm of D3]{};
    \node[datagram header](D5)[right=0.5cm of D4]{};
    \node[datagram payload](D6)[right=0cm of D5]{};
	\node[interface,
		below=of T.south, 
		xshift=-0.5cm, 
		yshift=1cm, 
		rotate=270](FirstArrow){};
	\node[interface,
		below=of T.south, 
		xshift=0.5cm, 
		yshift=1cm, 
		rotate=90]{};
	\end{tikzpicture}
	}
\end{cf}

\begin{cf}
	Link layer provides \textbf{node-to-node} communication services of \textbf{frames}.
\end{cf}
	
\begin{cf}
	Possible services:\\
	\textbf{Framing}
\end{cf}

\begin{cf}{
	Possible services:\\
	\textbf{Link Access}
}
\end{cf}

\begin{cf}{
	Possible services:\\
	\textbf{Reliable Delivery}
}
\end{cf}

\begin{cf}{
	Possible services:\\
	\textbf{Error Detection and Correction}
}
\end{cf}

\begin{cf}[t]
	Parity Bit
\end{cf}

\begin{cf}[t]
	2D Parity
\end{cf}

\begin{cf}[t]
	Cyclic Redundancy Check\\[10pt]

	$D 2^r + R = nG$ \\
	$R$ is remainder of $D2^r / G$
\end{cf}

\begin{cf}[t]
	CRC calculation is done in base-2 arithmatic without carry or borrow.
\end{cf}

\begin{cf}[t]
	Example: G = 1001, D = 101110
\end{cf}
\begin{cf}{
	\textbf{Multiple Access Protocol}\\
	\small (for a shared medium)
}
\end{cf}

\begin{cf}{
	1. Partition the Channel \\
	2. Take Turns\\
	3. Randomly Access\\
}
\end{cf}

\begin{cf}{
	Time-Division Multiplexing\\
	\vspace{1cm}
	\begin{tikzpicture}[start chain=going right,node distance=0cm,minimum height=4cm]
	\node[draw,rectangle,on chain,fill=red!30!white](){1};
	\node[draw,rectangle,on chain,fill=yellow!30!white](){2};
	\node[draw,rectangle,on chain,fill=green!30!white](){3};
	\node[draw,rectangle,on chain,fill=blue!30!white](){4};
	\node[draw,rectangle,on chain,fill=red!30!white](){1};
	\node[draw,rectangle,on chain,fill=yellow!30!white](){2};
	\node[draw,rectangle,on chain,fill=green!30!white](){3};
	\node[draw,rectangle,on chain,fill=blue!30!white](){4};
	\end{tikzpicture}
}
\end{cf}

\begin{cf}{
	Frequency-Division Multiplexing\\
	\vspace{1cm}
	\begin{tikzpicture}[start chain=going below,node distance=0cm,minimum height=1cm, minimum width=6cm]
	\node[draw,rectangle,on chain,fill=red!30!white](){4-8kHz};
	\node[draw,rectangle,on chain,fill=yellow!30!white](){8-12kHz};
	\node[draw,rectangle,on chain,fill=green!30!white](){12-16kHz};
	\node[draw,rectangle,on chain,fill=blue!30!white](){16-20kHz};
	\end{tikzpicture}
}
\end{cf}

\begin{cf}{
	1. Partition the Channel \\
	2. Take Turns\\
	3. Randomly Access\\
}
\end{cf}

\begin{cf}{
	Polling
}
\end{cf}

\begin{cf}{
	Token Passing
}
\end{cf}

\begin{cf}{
	Admin Matters
}
\end{cf}

\begin{cf}{
	Midterm and Assignment 1 marks posted on IVLE Gradebook.
	Please check for surprises.
}
\end{cf}

\begin{cf}{
	Assignment 2 due this coming Sunday 23:59.
}
\end{cf}

\begin{cf}{
	1. Partition the Channel \\
	2. Take Turns\\
	3. Randomly Access\\
}
\end{cf}

\begin{cf}{
	Slotted ALOHA\\
	\vspace{1cm}
	\begin{tikzpicture}[start chain=going right,node distance=0cm,minimum height=1cm]
	\node[draw,rectangle,on chain,gray!40!white,minimum width=2cm]{};
	\node[draw,rectangle,on chain,gray!40!white,minimum width=2cm]{};
	\node[draw,rectangle,on chain,gray!40!white,minimum width=2cm]{};
	\node[draw,rectangle,on chain=going below,gray!40!white,minimum width=2cm]{};
	\node[draw,rectangle,on chain=going left,gray!40!white,minimum width=2cm]{};
	\node[draw,rectangle,on chain=going left,gray!40!white,minimum width=2cm]{};
        \node[draw,rectangle,on chain=going below,gray!40!white,minimum width=2cm]{};
	\node[draw,rectangle,on chain,gray!40!white,minimum width=2cm]{};
	\node[draw,rectangle,on chain,gray!40!white,minimum width=2cm]{};
	\end{tikzpicture}
	\\
	\note{
		The discussions about effeciency of ALOHA, Slotted ALOHA, CSMA, and CSMA/CD in the textbook is outside the scope of CS2105.
	}
}
\end{cf}

\begin{cf}{
	ALOHA\\
	\vspace{1cm}
	\begin{tikzpicture}[start chain=going below,node distance=0cm,minimum height=1cm]
	\node[draw,rectangle,on chain,gray!40!white,minimum width=6cm]{};
	\node[draw,rectangle,on chain,gray!40!white,minimum width=6cm]{};
	\node[draw,rectangle,on chain,gray!40!white,minimum width=6cm]{};
	\end{tikzpicture}
}
\end{cf}

%\begin{frame}
%When has a frame to send:
%	\begin{itemize}
%	\item wait until beginning of next slot
%	\item send
%	\item while there is a collision 
%	\begin{itemize}
%	\item retransmit at next slot with probability $p$
%	\end{itemize}
%	\end{itemize}
%\end{frame}

\begin{cf}{
	CSMA (Carrier Sense Multiple Access)\\
	\vspace{1cm}
	\begin{tikzpicture}[start chain=going below,node distance=0cm,minimum height=1cm]
	\node[draw,rectangle,on chain,gray!40!white,minimum width=6cm]{};
	\node[draw,rectangle,on chain,gray!40!white,minimum width=6cm]{};
	\node[draw,rectangle,on chain,gray!40!white,minimum width=6cm]{};
	\end{tikzpicture}
}
\end{cf}

\begin{cf}{
	CSMA/CD (Carrier Sense Multiple Access/Collision Detection)\\
	\vspace{1cm}
	\begin{tikzpicture}[start chain=going below,node distance=0cm,minimum height=1cm]
	\node[draw,rectangle,on chain,gray!40!white,minimum width=6cm]{};
	\node[draw,rectangle,on chain,gray!40!white,minimum width=6cm]{};
	\node[draw,rectangle,on chain,gray!40!white,minimum width=6cm]{};
	\end{tikzpicture}
}
\end{cf}

\begin{cf}{\normalsize
	While channel is not idle, wait\\
	\vspace{1cm}
	\begin{tikzpicture}[scale=2]
		\draw[solid] (0,0) -- (0,3);
		\draw[solid] (2,0) -- (2,3);
	\end{tikzpicture}
}
\end{cf}

\begin{cf}{\normalsize
	if collide during transmission, stop\\
	\vspace{1cm}
	\begin{tikzpicture}[scale=2]
		\draw[solid] (0,0) -- (0,3);
		\draw[solid] (2,0) -- (2,3);
	\end{tikzpicture}
}
\end{cf}

\begin{cf}{\small
	At $n$-th consecutive collision, let $m = \min(n, 10)$.
	Pick $K$ randomly from $\{0, 1, .. 2^{m-1}\}$.  Wait for 512K bit time.\\
	\vspace{1cm}
	\begin{tikzpicture}[scale=2]
		\draw[solid] (0,0) -- (0,2);
		\draw[solid] (2,0) -- (2,2);
	\end{tikzpicture}
}
\end{cf}

\begin{cf}{
	Example: DOCSIS \\
	\vspace{1cm}
	\begin{tikzpicture}[start chain=going below,node distance=0cm,minimum height=1cm, minimum width=6.2cm]
	\node[draw,rectangle,on chain,minimum width=6cm,fill=yellow!30!white]{$\rightarrow$};
	\node[draw,rectangle,on chain,minimum width=6cm,fill=yellow!30!white]{$\rightarrow$};
	\node[draw,rectangle,on chain,minimum width=6cm,fill=yellow!30!white]{$\rightarrow$};
	\end{tikzpicture}\\
	\vspace{1cm}
	\begin{tikzpicture}[start chain=going right,node distance=0cm,minimum height=1cm, minimum width=1.2cm]
	\node[draw,rectangle,on chain,fill=yellow!30!white](){};
	\node[draw,rectangle,on chain,fill=yellow!30!white](){};
	\node[draw,rectangle,on chain,fill=yellow!30!white](){$\leftarrow$};
	\node[draw,rectangle,on chain,fill=yellow!30!white](){};
	\node[draw,rectangle,on chain,fill=yellow!30!white](){};
	\end{tikzpicture}\\
}
\end{cf}

\begin{cf}{
	Ethernet Framing Example\\
	\vspace{1cm}
	\begin{tikzpicture}[start chain=going right,node distance=0cm,minimum height=1cm, draw, rectangle]
	\node[draw,rectangle,on chain,minimum width=2cm,fill=red!80!black]{};
	\node[draw,rectangle,on chain,minimum width=1.5cm,fill=orange!30!red]{};
	\node[draw,rectangle,on chain,minimum width=1.5cm,fill=yellow!50!white]{};
	\node[draw,rectangle,on chain,minimum width=0.5cm,fill=green!50!white]{};
	\node[draw,rectangle,on chain,minimum width=3cm,fill=blue!30!white]{};
	\node[draw,rectangle,on chain,minimum width=1cm,fill=purple!50!white]{};
	\end{tikzpicture}
}
\end{cf}

\begin{cf}{
	\vspace{1cm}
	\begin{tikzpicture}[start chain=going right,node distance=0cm,minimum height=1cm, draw, rectangle]
	\node[draw,rectangle,on chain,minimum width=2cm,fill=red!80!black](P){};
        \node[below=2cm of P, align=center,font=\small](PL){Preamble\\(8 bytes)};
	\draw[black](PL) -- (P);
	\node[draw,rectangle,on chain,minimum width=1.5cm,fill=orange!30!red]{};
	\node[draw,rectangle,on chain,minimum width=1.5cm,fill=yellow!50!white]{};
	\node[draw,rectangle,on chain,minimum width=0.5cm,fill=green!50!white]{};
	\node[draw,rectangle,on chain,minimum width=3cm,fill=blue!30!white]{};
	\node[draw,rectangle,on chain,minimum width=1cm,fill=purple!50!white]{};
	\end{tikzpicture}
}
\end{cf}
\begin{cf}{
	\vspace{1cm}
	\begin{tikzpicture}[start chain=going right,node distance=0cm,minimum height=1cm, draw, rectangle]
	\node[draw,rectangle,on chain,minimum width=2cm,fill=red!80!black](){};
	\node[draw,rectangle,on chain,minimum width=1.5cm,fill=orange!30!red](P){};
        \node[below=2cm of P, align=center,xshift=-1cm,font=\small](PL){Src MAC\\Address\\(6 bytes)};
	\draw[black](PL) -> (P);
	\node[draw,rectangle,on chain,minimum width=1.5cm,fill=yellow!50!white](Q){};
        \node[below=2cm of Q, align=center,xshift=1cm,font=\small](QL){Dest MAC\\Address\\(6 bytes)};
	\draw[black](QL) -> (Q);
	\node[draw,rectangle,on chain,minimum width=0.5cm,fill=green!50!white]{};
	\node[draw,rectangle,on chain,minimum width=3cm,fill=blue!30!white]{};
	\node[draw,rectangle,on chain,minimum width=1cm,fill=purple!50!white]{};
	\end{tikzpicture}
}
\end{cf}

\begin{cf}{
	\vspace{1cm}
	\begin{tikzpicture}[start chain=going right,node distance=0cm,minimum height=1cm, draw, rectangle]
	\node[draw,rectangle,on chain,minimum width=2cm,fill=red!80!black](){};
	\node[draw,rectangle,on chain,minimum width=1.5cm,fill=orange!30!red](P){};
	\node[draw,rectangle,on chain,minimum width=1.5cm,fill=yellow!50!white]{};
	\node[draw,rectangle,on chain,minimum width=0.5cm,fill=green!50!white](Q){};
        \node[below=2cm of Q, align=center,font=\small](QL){Type\\(2 bytes)};
	\draw[black](QL) -> (Q);
	\node[draw,rectangle,on chain,minimum width=3cm,fill=blue!30!white]{};
	\node[draw,rectangle,on chain,minimum width=1cm,fill=purple!50!white]{};
	\end{tikzpicture}
}
\end{cf}
\begin{cf}{
	\vspace{1cm}
	\begin{tikzpicture}[start chain=going right,node distance=0cm,minimum height=1cm, draw, rectangle]
	\node[draw,rectangle,on chain,minimum width=2cm,fill=red!80!black](){};
	\node[draw,rectangle,on chain,minimum width=1.5cm,fill=orange!30!red](P){};
	\node[draw,rectangle,on chain,minimum width=1.5cm,fill=yellow!50!white]{};
	\node[draw,rectangle,on chain,minimum width=0.5cm,fill=green!50!white]{};
	\node[draw,rectangle,on chain,minimum width=3cm,fill=blue!30!white](Q){};
        \node[below=2cm of Q, align=center,font=\small](QL){Data\\(46-1500 bytes)};
	\draw[black](QL) -> (Q);
	\node[draw,rectangle,on chain,minimum width=1cm,fill=purple!50!white]{};
	\end{tikzpicture}
}
\end{cf}
\begin{cf}{
	\vspace{1cm}
	\begin{tikzpicture}[start chain=going right,node distance=0cm,minimum height=1cm, draw, rectangle]
	\node[draw,rectangle,on chain,minimum width=2cm,fill=red!80!black](){};
	\node[draw,rectangle,on chain,minimum width=1.5cm,fill=orange!30!red](P){};
	\node[draw,rectangle,on chain,minimum width=1.5cm,fill=yellow!50!white]{};
	\node[draw,rectangle,on chain,minimum width=0.5cm,fill=green!50!white]{};
	\node[draw,rectangle,on chain,minimum width=3cm,fill=blue!30!white](Q){};
	\node[draw,rectangle,on chain,minimum width=1cm,fill=purple!50!white](Q){};
        \node[below=2cm of Q, align=center,font=\small](QL){CRC\\(4 bytes)};
	\draw[black](QL) -> (Q);
	\end{tikzpicture}
}
\end{cf}
