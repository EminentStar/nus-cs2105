\begin{cf}{
CS2105 Lecture 12\\
\large
\textbf{Putting Everything Together\\[10pt]}
\normalsize
15 April, 2013
}
\end{cf}

\begin{cf}{
	\begin{tikzpicture}[scale=2, start chain=going below]
	\foreach \color / \label in {
		AppBlue/Application,
		TransportBlue/Transport,
		NetworkGreen/Network,
		LinkBrown/Link,
		PhysicalRed/Physical
	}
	\node[draw,rectangle,
	minimum height=1.5 cm, 
	minimum width=5 cm, 
	text=white, 
	node distance=0cm,
	on chain, fill=\color]{\label};
	\end{tikzpicture}
}
\end{cf}



\begin{cf}{
	\begin{tikzpicture}
	\node[draw,circle](R){};
	\node[below=.5cm of R,font=\footnotesize](){router};
	\node[draw,circle,right=of R](M){};
	\node[below=.5cm of M,font=\footnotesize](){modem};
	\node[cloud,draw,cloud puffs=10.4,cloud ignores aspect, minimum width=2.5cm, minimum height=1.6cm,on chain,right=of M](I){\footnotesize Internet};
	\path[draw] (M) -- (R);
	\path[draw] (M) -- (I);

	\node[draw,circle,above=of R](C){};
	\node[above=.5cm of C](){\footnotesize laptop};
	\end{tikzpicture}
}
\end{cf}

\begin{cf}{
	Turn on modem\\
	Turn on router\\
	Turn on laptop\\
	Visit IVLE\\
}
\end{cf}
	
\begin{cf}{
	Turn on Modem
}
\end{cf}

\begin{cf}{
	Modulate between Digital (Manchester Coding) and Analog (QAM/QPSK)
}
\end{cf}

\begin{cf}{
	Turn on Router
}
\end{cf}

\begin{cf}{
	Router broadcasts DHCP Discover
}
\end{cf}

\tikzstyle{layer}=[draw,rectangle,minimum height=2.5 cm, minimum width=9 cm, text=white]
\tikzstyle{interface}=[draw,single arrow,fill=white, minimum width=0.9cm, minimum height=1.2cm]
\tikzstyle{message}=[draw,rectangle,minimum height=.7cm,minimum width=3cm,fill=white!90!yellow]
\tikzstyle{segment payload}=[draw,rectangle,minimum height=.7cm,minimum width=1cm,inner sep=0cm, fill=white!90!yellow]
\tikzstyle{segment header}=[draw,rectangle,minimum height=.7cm,minimum width=.4cm,inner sep=0cm, fill=white!90!red]
\tikzstyle{datagram header}=[draw,rectangle,minimum height=.7cm,minimum width=.4cm,inner sep=0cm, fill=white!50!NetworkGreen]

\newcommand{\segment}[3]{
    \node[segment header](#1)[#3]{};
    \node[segment payload](#2)[right=0cm of #1]{};
}
\newcommand{\datagramwithsegmentheader}[3]{
    \node[datagram header](#1)[#3]{};
    \node[segment header](#1-P)[right=0cm of #1]{};
    \node[segment payload](#2)[minimum width=0.6cm, right=0cm of #1-P]{};
}
\newcommand{\datagramwithoutsegmentheader}[3]{
    \node[datagram header](#1)[#3]{};
    \node[segment payload](#2)[minimum width=0.4cm, right=0cm of #1]{};
}
\begin{cf}{
	\begin{tikzpicture}[scale=2]
	\node[layer](A)[
		fill=AppBlue,
		]{};
	\node[layer](T)[
		below = 0cm of A,
		yshift = -0.1cm, 
		fill=TransportBlue]{};
    \node[message](M)[
		below=0.5cm of A.north,
	]{};
	\node[layer](N)[
		below = 0cm of T,
		yshift = -0.1cm, 
		fill=NetworkGreen]{};
	\node[interface,
		below=of A.south, 
		xshift=-0.5cm, 
		yshift=1cm, 
		rotate=270](FirstArrow){};
	\node[interface,
		below=of A.south, 
		xshift=0.5cm, 
		yshift=1cm, 
		rotate=90]{};
	\segment{D1}{D2}{below=1cm of FirstArrow, xshift=-2cm}
	\segment{D3}{D4}{right=0.5cm of D2}
	\segment{D5}{D6}{right=0.5cm of D4}
	\node[interface,
		below=of T.south, 
		xshift=-0.5cm, 
		yshift=1cm, 
		rotate=270](Arrow2){};
	\node[interface,
		below=of T.south, 
		xshift=0.5cm, 
		yshift=1cm, 
		rotate=90]{};
	\datagramwithsegmentheader{P1}{P2}{below=1cm of Arrow2, xshift=-2.8cm}
	\datagramwithoutsegmentheader{P3}{P4}{right=0.3cm of P2}
	\datagramwithsegmentheader{P5}{P6}{right=0.3cm of P4}
	\datagramwithoutsegmentheader{P7}{P8}{right=0.3cm of P6}
	\datagramwithsegmentheader{P9}{P10}{right=0.3cm of P8}
	\datagramwithoutsegmentheader{P11}{P12}{right=0.3cm of P10}
	\end{tikzpicture}
	}
\end{cf}

% UDP HEADER
\tikzstyle{udpfield}=[draw,rectangle,minimum height=1.5 cm, minimum width=4cm-0.4pt, text=gray!80!black,fill=white!90!blue,font=\small]
\begin{cf}{
\begin{tikzpicture}
\tikzstyle{payload}=[draw,rectangle,minimum height=3 cm, minimum width=8cm, text=gray!80!black,fill=white!90!yellow]
	\node[udpfield](SP){68};
	\node[udpfield](DP)[right=0cm of SP]{67};
	\node[udpfield,fill=white](LN)[below=0cm of SP]{};
	\node[udpfield,fill=white](CS)[right=0cm of LN]{};
	\node[above=.2cm of SP.north west]{\tiny 1};
	\node[above=.2cm of SP.north east]{\tiny 16};
	\node[above=.2cm of DP.north east]{\tiny 32};
\end{tikzpicture}
}
\end{cf}

\tikzstyle{ipfield}=[draw,rectangle,minimum height=1.5 cm, minimum width=4cm-0.4pt, text=gray!80!black,fill=white!90!NetworkGreen,font=\small]
\tikzstyle{ipfield16}=[ipfield,minimum width=4.0cm-0.4pt]
\tikzstyle{ipfield12}=[ipfield,minimum width=3.0cm-0.4pt]
\tikzstyle{ipfield32}=[ipfield,minimum width=8.0cm-0.4pt]
\tikzstyle{ipfield08}=[ipfield,minimum width=2.0cm-0.4pt]
\tikzstyle{ipfield06}=[ipfield,minimum width=1.5cm-0.4pt]
\tikzstyle{ipfield04}=[ipfield,minimum width=1.0cm-0.4pt]
\tikzstyle{ipfield01}=[ipfield,minimum width=0.25cm-0.4pt,inner sep=0cm]
\tikzstyle{ipselected}=[text=white,fill=NetworkGreen!40!black]

% IP HEADER
\begin{cf}{
	\begin{tikzpicture}
		\node[ipfield04](A1){};
		\node[ipfield04](A2)[right=0cm of A1]{};
		\node[ipfield08](A3)[right=0cm of A2]{};
		\node[ipfield16](A4)[right=0cm of A3]{};
		\node[ipfield16](B1)[below=0cm of A2.south east]{};
		\node[ipfield04](B2)[right=0cm of B1]{};
		\node[ipfield12](B3)[right=0cm of B2]{};
		\node[ipfield08,ipselected](C1)[below=0cm of B1.south west,xshift=1cm]{255};
		\node[ipfield08,ipselected](C2)[right=0cm of C1]{UDP};
		\node[ipfield16](C3)[right=0cm of C2]{};
		\node[ipfield32,ipselected](D1)[below=0cm of C3.south west]{0.0.0.0};
		\node[ipfield32,ipselected](D2)[below=0cm of D1]{255.255.255.255};
	\end{tikzpicture}
}
\end{cf}

% ETHERNET HEADER
\begin{cf}{
	\vspace{1cm}
	\begin{tikzpicture}[start chain=going right,node distance=0cm,minimum height=1cm, draw, rectangle]
	\node[draw,rectangle,on chain,minimum width=2cm,fill=red!80!black](){};
	\node[draw,rectangle,on chain,minimum width=1.5cm,fill=orange!30!red](S){};
        \node[below=2cm of S, align=center,xshift=-2cm,font=\small](SL){\texttt{04:42:00:31:98:12}};
	\draw[black](SL) -> (S);
	\node[draw,rectangle,on chain,minimum width=1.5cm,fill=yellow!50!white](D){};
        \node[below=2cm of D, align=center,xshift=-0.5cm,yshift=-1cm,font=\small](DL){\texttt{ff:ff:ff:ff:ff:ff}};
	\draw[black](DL) -> (D);
	\node[draw,rectangle,on chain,minimum width=0.5cm,fill=green!50!white](T){};
        \node[below=2cm of T, align=center,font=\small](TL){IP};
	\draw[black](TL) -> (T);
	\node[draw,rectangle,on chain,minimum width=3cm,fill=blue!30!white](Q){};
	\node[draw,rectangle,on chain,minimum width=1cm,fill=purple!50!white](Q){};
	\end{tikzpicture}
}
\end{cf}

\begin{cf}{
	\begin{tikzpicture}
	\node[draw,circle](R){};
	\node[below=.5cm of R,font=\footnotesize](){router};
	\node[draw,circle,right=of R](M){};
	\node[below=.5cm of M,font=\footnotesize](){modem};
	\node[cloud,draw,cloud puffs=10.4,cloud ignores aspect, minimum width=2.5cm, minimum height=1.6cm,on chain,right=of M](I){\footnotesize Internet};
	\path[draw] (M) -- (R);
	\path[draw] (M) -- (I);

	\node[draw,circle,above=of R](C){};
	\node[above=.5cm of C](){\footnotesize laptop};
	\end{tikzpicture}
}
\end{cf}



\begin{cf}{
	Manchester
	\vspace{1cm}

	\begin{tikzpicture}
	\path[draw,->] (0,0) -- (7, 0);
	\path[draw,->] (0.5,-1) -- (0.5, 1);
	% draw verticle guides
	\foreach \x in {1.5, 2.5, 3.5, 4.5, 5.5, 6.5} 
		\path[draw,-,color=gray!50] (\x,-1) -- (\x, 1);
	\end{tikzpicture}
}
\end{cf}

\begin{cf}{
	DOCSIS \\
	\vspace{1cm}
	\begin{tikzpicture}[start chain=going below,node distance=0cm,minimum height=1cm, minimum width=6.2cm]
	\node[draw,rectangle,on chain,minimum width=6cm,fill=yellow!30!white]{$\leftarrow$};
	\node[draw,rectangle,on chain,minimum width=6cm,fill=yellow!30!white]{$\leftarrow$};
	\node[draw,rectangle,on chain,minimum width=6cm,fill=yellow!30!white]{$\leftarrow$};
	\end{tikzpicture}\\
	\vspace{1cm}
	\begin{tikzpicture}[start chain=going right,node distance=0cm,minimum height=1cm, minimum width=1.2cm]
	\node[draw,rectangle,on chain,fill=yellow!30!white](){};
	\node[draw,rectangle,on chain,fill=yellow!30!white](){};
	\node[draw,rectangle,on chain,fill=yellow!30!white](){$\rightarrow$};
	\node[draw,rectangle,on chain,fill=yellow!30!white](){};
	\node[draw,rectangle,on chain,fill=yellow!30!white](){};
	\end{tikzpicture}\\
}
\end{cf}

\begin{cf}{
	64-QAM
	\vspace{1cm}

	\begin{tikzpicture}
	\path[draw,->] (-2,0) -- (2, 0);
	\path[draw,->] (0,-2) -- (0,2);
	\end{tikzpicture}
}
\end{cf}

\begin{cf}{
	\begin{tikzpicture}
	\node[draw,circle](R){};
	\node[below=.5cm of R,font=\footnotesize](){router};
	\node[draw,circle,right=of R](M){};
	\node[below=.5cm of M,font=\footnotesize](){modem};
	\node[cloud,draw,cloud puffs=10.4,cloud ignores aspect, minimum width=2.5cm, minimum height=1.6cm,on chain,right=of M](I){\footnotesize Internet};
	\path[draw] (M) -- (R);
	\path[draw] (M) -- (I);

	\node[draw,circle,above=of R](C){};
	\node[above=.5cm of C](){\footnotesize laptop};
	\end{tikzpicture}
}
\end{cf}



\begin{cf}{
	Router receives DHCP Offer
}
\end{cf}

\begin{cf}{
	Router broadcasts ARP Request
}
\end{cf}

\begin{cf}{
\tikzstyle{packet}=[draw,rectangle,minimum height=.7cm]
\tikzstyle{layer}=[draw,rectangle,minimum height=2.5 cm, minimum width=9 cm, text=white]
\tikzstyle{interface}=[draw,single arrow,fill=white, minimum width=0.9cm, minimum height=1.2cm]
\tikzstyle{message}=[packet,minimum width=1cm,fill=white!90!yellow]
\tikzstyle{datagram payload}=[packet,minimum width=1cm,fill=white!90!yellow]
\tikzstyle{datagram header}=[packet,minimum width=.5cm,fill=white!90!blue]
	\begin{tikzpicture}[scale=2]
	\node[layer](A)[
		fill=NetworkGreen,
		]{};
	\node[layer](T)[
		below = 0cm of A,
		yshift = -0.1cm, 
		fill=LinkBrown]{};
    \node[message](M1)[below=0.5cm of A.north]{};
    \node[message](M2)[left=0.5cm of M1]{};
    \node[message](M3)[right=0.5cm of M1]{};
	\node[layer](N)[
		below = 0cm of T,
		yshift = -0.1cm, 
		fill=PhysicalRed]{\small 101011001...};
	\node[interface,
		below=of A.south, 
		xshift=-0.5cm, 
		yshift=1cm, 
		rotate=270](FirstArrow){};
	\node[interface,
		below=of A.south, 
		xshift=0.5cm, 
		yshift=1cm, 
		rotate=90]{};
    \node[datagram header](D1)[below=1cm of FirstArrow, xshift=-2cm]{};
    \node[datagram payload](D2)[right=0cm of D1]{}; 
    \node[datagram header](D3)[right=0.5cm of D2]{};
    \node[datagram payload](D4)[right=0cm of D3]{};
    \node[datagram header](D5)[right=0.5cm of D4]{};
    \node[datagram payload](D6)[right=0cm of D5]{};
	\node[interface,
		below=of T.south, 
		xshift=-0.5cm, 
		yshift=1cm, 
		rotate=270](FirstArrow){};
	\node[interface,
		below=of T.south, 
		xshift=0.5cm, 
		yshift=1cm, 
		rotate=90]{};
	\end{tikzpicture}
	}
\end{cf}

% ETHERNET HEADER
\begin{cf}{
	\vspace{1cm}
	\begin{tikzpicture}[start chain=going right,node distance=0cm,minimum height=1cm, draw, rectangle]
	\node[draw,rectangle,on chain,minimum width=2cm,fill=red!80!black](){};
	\node[draw,rectangle,on chain,minimum width=1.5cm,fill=orange!30!red](S){};
        \node[below=2cm of S, align=center,xshift=-2cm,font=\small](SL){\texttt{04:42:00:31:98:12}};
	\draw[black](SL) -> (S);
	\node[draw,rectangle,on chain,minimum width=1.5cm,fill=yellow!50!white](D){};
        \node[below=2cm of D, align=center,xshift=-0.5cm,yshift=-1cm,font=\small](DL){\texttt{ff:ff:ff:ff:ff:ff}};
	\draw[black](DL) -> (D);
	\node[draw,rectangle,on chain,minimum width=0.5cm,fill=green!50!white](T){};
        \node[below=2cm of T, align=center,font=\small](TL){ARP};
	\draw[black](TL) -> (T);
	\node[draw,rectangle,on chain,minimum width=3cm,fill=blue!30!white](Q){};
	\node[draw,rectangle,on chain,minimum width=1cm,fill=purple!50!white](Q){};
	\end{tikzpicture}
}
\end{cf}

\begin{cf}{
	Router receives ARP Respond
}
\end{cf}

\begin{cf}{
	Router sends DHCP Request
}
\end{cf}

% UDP HEADER
\begin{cf}{
\begin{tikzpicture}
	\node[udpfield](SP){68};
	\node[udpfield](DP)[right=0cm of SP]{67};
	\node[udpfield](LN)[below=0cm of SP,fill=white]{};
	\node[udpfield](CS)[right=0cm of LN,fill=white]{};
	\node[above=.2cm of SP.north west]{\tiny 1};
	\node[above=.2cm of SP.north east]{\tiny 16};
	\node[above=.2cm of DP.north east]{\tiny 32};
\end{tikzpicture}
}
\end{cf}

% IP HEADER
\begin{cf}{
	\begin{tikzpicture}
		\node[ipfield04](A1){};
		\node[ipfield04](A2)[right=0cm of A1]{};
		\node[ipfield08](A3)[right=0cm of A2]{};
		\node[ipfield16](A4)[right=0cm of A3]{};
		\node[ipfield16](B1)[below=0cm of A2.south east]{};
		\node[ipfield04](B2)[right=0cm of B1]{};
		\node[ipfield12](B3)[right=0cm of B2]{};
		\node[ipfield08,ipselected](C1)[below=0cm of B1.south west,xshift=1cm]{255};
		\node[ipfield08,ipselected](C2)[right=0cm of C1]{UDP};
		\node[ipfield16](C3)[right=0cm of C2]{};
		\node[ipfield32,ipselected](D1)[below=0cm of C3.south west]{0.0.0.0};
		\node[ipfield32,ipselected](D2)[below=0cm of D1]{218.186.2.1};
	\end{tikzpicture}
}
\end{cf}

% ETHERNET HEADER
\begin{cf}{
	\vspace{1cm}
	\begin{tikzpicture}[start chain=going right,node distance=0cm,minimum height=1cm, draw, rectangle]
	\node[draw,rectangle,on chain,minimum width=2cm,fill=red!80!black](){};
	\node[draw,rectangle,on chain,minimum width=1.5cm,fill=orange!30!red](S){};
        \node[below=2cm of S, align=center,xshift=-2cm,font=\small](SL){\texttt{04:42:00:31:98:12}};
	\draw[black](SL) -> (S);
	\node[draw,rectangle,on chain,minimum width=1.5cm,fill=yellow!50!white](D){};
        \node[below=2cm of D, align=center,xshift=-0.5cm,yshift=-1cm,font=\small](DL){\texttt{00:1c:f6:03:11:94}};
	\draw[black](DL) -> (D);
	\node[draw,rectangle,on chain,minimum width=0.5cm,fill=green!50!white](T){};
        \node[below=2cm of T, align=center,font=\small](TL){IP};
	\draw[black](TL) -> (T);
	\node[draw,rectangle,on chain,minimum width=3cm,fill=blue!30!white](Q){};
	\node[draw,rectangle,on chain,minimum width=1cm,fill=purple!50!white](Q){};
	\end{tikzpicture}
}
\end{cf}

\begin{cf}{
	Router receives DHCP ACK
}
\end{cf}

\begin{cf}{
	\begin{tikzpicture}
	\node[draw,circle](R){};
	\node[below=.5cm of R,font=\footnotesize](){router};
	\node[draw,circle,right=of R](M){};
	\node[below=.5cm of M,font=\footnotesize](){modem};
	\node[cloud,draw,cloud puffs=10.4,cloud ignores aspect, minimum width=2.5cm, minimum height=1.6cm,on chain,right=of M](I){\footnotesize Internet};
	\path[draw] (M) -- (R);
	\path[draw] (M) -- (I);

	\node[draw,circle,above=of R](C){};
	\node[above=.5cm of C](){\footnotesize laptop};
	\end{tikzpicture}
}
\end{cf}



\begin{cf}{
	\includegraphics[scale=0.4]{figures/airport-utility-screenshort.png}
}
\end{cf}

\begin{cf}{
	Turn on Laptop and
	connects to WiFi
	\note{
		There are packet exchanges when the laptop associates itself and authenticate itself with the WiFi AP, which we skip here.
	}
}
\end{cf}

\begin{cf}{
	Laptop runs DHCP client\\
	Router runs DHCP server
}
\end{cf}

\begin{cf}{
	\begin{tikzpicture}[scale=2]
		\draw[solid] (0,0) -- (0,3);
		\draw[solid] (2,0) -- (2,3);
	\end{tikzpicture}
	}
\end{cf}


\begin{cf}{
	\includegraphics[scale=0.4]{figures/dhcp-server-screenshot.png}
}
\end{cf}

\begin{cf}{
	\includegraphics[scale=0.4]{figures/network-preference-screenshot.png}
}
\end{cf}

\begin{cf}{
	Visit https://ivle.nus.edu.sg/
}
\end{cf}

\begin{cf}{
	DNS lookup for ivle.nus.edu.sg
}
\end{cf}

\begin{cf}{
	\includegraphics[scale=0.4]{figures/airport-utility-screenshort.png}
}
\end{cf}

% UDP HEADER
\begin{cf}{
\begin{tikzpicture}
	\node[udpfield](SP){59424};
	\node[udpfield](DP)[right=0cm of SP]{53};
	\node[udpfield,fill=white](LN)[below=0cm of SP]{};
	\node[udpfield,fill=white](CS)[right=0cm of LN]{};
	\node[above=.2cm of SP.north west]{\tiny 1};
	\node[above=.2cm of SP.north east]{\tiny 16};
	\node[above=.2cm of DP.north east]{\tiny 32};
\end{tikzpicture}
}
\end{cf}


% IP HEADER
\begin{cf}{
	\begin{tikzpicture}
		\node[ipfield04](A1){};
		\node[ipfield04](A2)[right=0cm of A1]{};
		\node[ipfield08](A3)[right=0cm of A2]{};
		\node[ipfield16](A4)[right=0cm of A3]{};
		\node[ipfield16](B1)[below=0cm of A2.south east]{};
		\node[ipfield04](B2)[right=0cm of B1]{};
		\node[ipfield12](B3)[right=0cm of B2]{};
		\node[ipfield08](C1)[below=0cm of B1.south west,xshift=1cm]{};
		\node[ipfield08,ipselected](C2)[right=0cm of C1]{UDP};
		\node[ipfield16](C3)[right=0cm of C2]{};
		\node[ipfield32,ipselected](D1)[below=0cm of C3.south west]{10.0.1.7};
		\node[ipfield32,ipselected](D2)[below=0cm of D1]{218.186.2.6};
	\end{tikzpicture}
}
\end{cf}

% ETHERNET HEADER
\begin{cf}{
	\vspace{1cm}
	\begin{tikzpicture}[start chain=going right,node distance=0cm,minimum height=1cm, draw, rectangle]
	\node[draw,rectangle,on chain,minimum width=2cm,fill=red!80!black](){};
	\node[draw,rectangle,on chain,minimum width=1.5cm,fill=orange!30!red](S){};
        \node[below=2cm of S, align=center,xshift=-2cm,font=\small](SL){\texttt{69:1b:04:42:33:12}};
	\draw[black](SL) -> (S);
	\node[draw,rectangle,on chain,minimum width=1.5cm,fill=yellow!50!white](D){};
        \node[below=2cm of D, align=center,xshift=-0.5cm,yshift=-1cm,font=\small](DL){\texttt{70:56:81:c7:11:e2}};
	\draw[black](DL) -> (D);
	\node[draw,rectangle,on chain,minimum width=0.5cm,fill=green!50!white](T){};
        \node[below=2cm of T, align=center,font=\small](TL){IP};
	\draw[black](TL) -> (T);
	\node[draw,rectangle,on chain,minimum width=3cm,fill=blue!30!white](Q){};
	\node[draw,rectangle,on chain,minimum width=1cm,fill=purple!50!white](Q){};
        \node[below=2cm of Q, align=center,font=\small](QL){};
	\end{tikzpicture}
}
\end{cf}

\begin{frame}[fragile]\footnotesize
\begin{verbatim}
laptop:~ ooiwt$ arp -an
? (10.0.1.1) at 70:56:81:c7:11:e2 on en0 
? (10.0.1.8) at d0:23:db:47:64:6f on en0 
? (10.0.1.255) at ff:ff:ff:ff:ff:ff on en0
? (10.0.1.255) at ff:ff:ff:ff:ff:ff on en1
\end{verbatim}

\end{frame}

\begin{cf}{
	\begin{tikzpicture}
	\node[draw,circle](R){};
	\node[below=.5cm of R,font=\footnotesize](){router};
	\node[draw,circle,right=of R](M){};
	\node[below=.5cm of M,font=\footnotesize](){modem};
	\node[cloud,draw,cloud puffs=10.4,cloud ignores aspect, minimum width=2.5cm, minimum height=1.6cm,on chain,right=of M](I){\footnotesize Internet};
	\path[draw] (M) -- (R);
	\path[draw] (M) -- (I);

	\node[draw,circle,above=of R](C){};
	\node[above=.5cm of C](){\footnotesize laptop};
	\end{tikzpicture}
}
\end{cf}



\begin{frame}[fragile]\footnotesize
\begin{verbatim}
traceroute to 218.186.2.6 (218.186.2.6)
 1  10.0.1.1  1.354 ms  0.826 ms  0.913 ms
 2  218.186.176.1  14.015 ms  7.879 ms  12.394 ms
 3  172.20.12.1  11.866 ms  11.813 ms  7.667 ms
 4  172.20.7.82  10.794 ms  15.408 ms  11.775 ms
 5  172.20.7.38  9.464 ms  8.032 ms  11.391 ms
 6  172.20.8.5  12.319.ms 13.101 ms  55.906 ms
 7  218.186.2.6  12.111 ms  11.196 ms  9.436 ms
\end{verbatim}
\end{frame}

\begin{cf}{
	Laptop receives DNS response
	}
\end{cf}

\begin{frame}[fragile]\footnotesize
\begin{verbatim}

;; QUESTION SECTION:
;ivle.nus.edu.sg.		IN	A

;; ANSWER SECTION:
ivle.nus.edu.sg.	294	IN	A	137.132.39.20

\end{verbatim}
\end{frame}

\begin{cf}{
	Laptop initiates HTTPS connection
	(sends TCP SYN to port 443)
	}
\end{cf}

% TCP Header
\begin{cf}{
\begin{tikzpicture}
	\node[ipfield16,ipselected](SP){52123};
	\node[ipfield16,ipselected](DP)[right=0cm of SP]{443};
	\node[ipfield32](SN)[below=0cm of SP.south east]{};
	\node[ipfield32](AN)[below=0cm of SN]{};
	\node[ipfield04](X1)[below=0cm of AN.south west,xshift=0.5cm ]{};
	\node[ipfield06](X2)[right=0cm of X1]{};
	\node[ipfield01](X3)[right=0cm of X2]{};
	\node[ipfield01](X4)[right=0cm of X3]{};
	\node[ipfield01](X5)[right=0cm of X4]{};
	\node[ipfield01](X6)[right=0cm of X5]{};
	\node[ipfield01](X7)[right=0cm of X6]{};
	\node[ipfield01](X8)[right=0cm of X7]{};
	\node[ipfield16](RW)[right=0cm of X8]{};
	\node[ipfield16](CS)[below=0cm of RW]{};
	\node[ipfield16](X10)[left=0cm of CS]{};
\end{tikzpicture}
}
\end{cf}

% IP HEADER
\begin{cf}{
	\begin{tikzpicture}
		\node[ipfield04](A1){};
		\node[ipfield04](A2)[right=0cm of A1]{};
		\node[ipfield08](A3)[right=0cm of A2]{};
		\node[ipfield16](A4)[right=0cm of A3]{};
		\node[ipfield16](B1)[below=0cm of A2.south east]{};
		\node[ipfield04](B2)[right=0cm of B1]{};
		\node[ipfield12](B3)[right=0cm of B2]{};
		\node[ipfield08](C1)[below=0cm of B1.south west,xshift=1cm]{};
		\node[ipfield08,ipselected](C2)[right=0cm of C1]{TCP};
		\node[ipfield16](C3)[right=0cm of C2]{};
		\node[ipfield32,ipselected](D1)[below=0cm of C3.south west]{10.0.1.7};
		\node[ipfield32,ipselected](D2)[below=0cm of D1]{137.132.39.20};
	\end{tikzpicture}
}
\end{cf}

\begin{cf}{
	\begin{tikzpicture}
	\node[draw,circle](R){};
	\node[below=.5cm of R,font=\footnotesize](){router};
	\node[draw,circle,right=of R](M){};
	\node[below=.5cm of M,font=\footnotesize](){modem};
	\node[cloud,draw,cloud puffs=10.4,cloud ignores aspect, minimum width=2.5cm, minimum height=1.6cm,on chain,right=of M](I){\footnotesize Internet};
	\path[draw] (M) -- (R);
	\path[draw] (M) -- (I);

	\node[draw,circle,above=of R](C){};
	\node[above=.5cm of C](){\footnotesize laptop};
	\end{tikzpicture}
}
\end{cf}



\begin{cf}{
	Router creates NAT table entry
	}
\end{cf}

% TCP Header
\begin{cf}{
\begin{tikzpicture}
	\node[ipfield16,ipselected](SP){10000};
	\node[ipfield16,ipselected](DP)[right=0cm of SP]{443};
	\node[ipfield32](SN)[below=0cm of SP.south east]{};
	\node[ipfield32](AN)[below=0cm of SN]{};
	\node[ipfield04](X1)[below=0cm of AN.south west,xshift=0.5cm ]{};
	\node[ipfield06](X2)[right=0cm of X1]{};
	\node[ipfield01](X3)[right=0cm of X2]{};
	\node[ipfield01](X4)[right=0cm of X3]{};
	\node[ipfield01](X5)[right=0cm of X4]{};
	\node[ipfield01](X6)[right=0cm of X5]{};
	\node[ipfield01](X7)[right=0cm of X6]{};
	\node[ipfield01](X8)[right=0cm of X7]{};
	\node[ipfield16](RW)[right=0cm of X8]{};
	\node[ipfield16](CS)[below=0cm of RW]{};
	\node[ipfield16](X10)[left=0cm of CS]{};
\end{tikzpicture}
}
\end{cf}

% IP HEADER
\begin{cf}{
	\begin{tikzpicture}
		\node[ipfield04](A1){};
		\node[ipfield04](A2)[right=0cm of A1]{};
		\node[ipfield08](A3)[right=0cm of A2]{};
		\node[ipfield16](A4)[right=0cm of A3]{};
		\node[ipfield16](B1)[below=0cm of A2.south east]{};
		\node[ipfield04](B2)[right=0cm of B1]{};
		\node[ipfield12](B3)[right=0cm of B2]{};
		\node[ipfield08,](C1)[below=0cm of B1.south west,xshift=1cm]{};
		\node[ipfield08,ipselected](C2)[right=0cm of C1]{TCP};
		\node[ipfield16](C3)[right=0cm of C2]{};
		\node[ipfield32,ipselected](D1)[below=0cm of C3.south west]{218.186.176.42};
		\node[ipfield32,ipselected](D2)[below=0cm of D1]{137.132.39.20};
	\end{tikzpicture}
}
\end{cf}

\begin{cf}{
	\begin{tikzpicture}[scale=2]
		\draw[solid] (0,0) -- (0,3);
		\draw[solid] (2,0) -- (2,3);
	\end{tikzpicture}
	}
\end{cf}


\begin{frame}[fragile]\footnotesize
\begin{verbatim}
GET / HTTP/1.1
Host: ivle.nus.edu.sg
User-Agent: Mozilla/5.0 
Keep-Alive: 300
Connection: keep-alive
Cookie:__utma=145952555.278120873
\end{verbatim}
\end{frame}

\begin{cf}
	Transfer data $m$ with sequence number $s$ as $m'$\\
  \vspace{1cm}
	$h = H(m + M_B + s)$\\
	$m' = E_B (m, h)$\\
\end{cf}


\begin{cf}
	Receives $m'$ with expected sequence number $s'$
  \vspace{1cm}
	$m, h = E_A(m')$\\
	check if $h = H(m + M_A + s') ?$\\
\end{cf}

\begin{frame}[fragile]\footnotesize
\begin{verbatim}
HTTP/1.1 200 OK
Content-Type: text/html; charset=utf-8
Server: Microsoft-IIS/7.5
Set-Cookie: ASP.NET_SessionId=jzhnyhv20h..
Date: Thu, 14 Apr 2011 15:58:41 GMT
Content-Length: 39616
Connection: Keep-Alive
\end{verbatim}
\end{frame}

\begin{cf}{
	\begin{tikzpicture}[scale=2]
		\draw[solid] (0,0) -- (0,3);
		\draw[solid] (2,0) -- (2,3);
	\end{tikzpicture}
	}
\end{cf}


\begin{cf}
	General Lessons from CS2105
\end{cf}

\begin{cf}
	How to build a complex system
\end{cf}

\begin{frame}\begin{center}\normalsize
Many issues to consider, to support different applications running on large number of hosts through different access technology and physical media.
\end{center}\end{frame}

\begin{frame}[plain]
\begin{tikzpicture}[remember picture,overlay]
	\node[at=(current page.center)] {
		\includegraphics[width=\paperwidth]{figures/confused-jackie.jpg}
	}; 
\end{tikzpicture}
\end{frame}

\begin{cf}
	1. Layering
\end{cf}


\begin{cf}{
	\begin{tikzpicture}[scale=2, start chain=going below]
	\foreach \color / \label in {
		AppBlue/Application,
		TransportBlue/Transport,
		NetworkGreen/Network,
		LinkBrown/Link,
		PhysicalRed/Physical
	}
	\node[draw,rectangle,
	minimum height=1.5 cm, 
	minimum width=5 cm, 
	text=white, 
	node distance=0cm,
	on chain, fill=\color]{\label};
	\end{tikzpicture}
}
\end{cf}



\begin{cf}{\small
	\begin{itemize}
	\item Lower layers hide details from upper layer
	\item Upper layers make minimum assumptions about lower layer
	\item Protocols in layers are interchangeable
	\item Keep lower layer simple, more features at higher layer.
	\end{itemize}
}
\end{cf}

\begin{cf}
	2. Decouple name from address\\
	\vspace{1cm}
	hostname, cname, IP addr, MAC addr
\end{cf}

\begin{cf}{\small
	\begin{itemize}
	\item Rotate hosts for load balancing
	\item Change hosts based on geo-location, network conditions
  \item Replace or move hosts without affecting names
	\end{itemize}
}
\end{cf}

\begin{cf}
	3. Hierarchical Naming/Routing\\
	\vspace{1cm}
	www.nus.edu.sg\\
	137.132/16
\end{cf}


\begin{cf}{\small
	\begin{itemize}
	\item More organized
	\item Allow distribution of responsibility
  \item Allow aggregation of addresses for forwarding
	\item Other examples: files, zip code
	\end{itemize}
}
\end{cf}

\begin{cf}
	4. Periodically Forget\\
	\vspace{1cm}
	DNS caches, switching table, DHCP leases, RIP
\end{cf}

\begin{cf}{\small
	\begin{itemize}
	\item Systems adapt quickly
	\item Prevent stale information automatically
	\end{itemize}
}
\end{cf}

\begin{cf}
	5. Randomization\\
	\vspace{1cm}
	nonce, TCP sequence numbers, BEB
\end{cf}

\begin{cf}{\small
	\begin{itemize}
	\item Avoid expensive coordination
	\item Avoid ``coincidence'' (collision, TCP sequence numbers)
	\item Avoid malicious party from guessing
	\end{itemize}
}
\end{cf}

\begin{cf}
	6. Learn by observation\\
	\vspace{1cm}
	NAT, Switch, BEB
\end{cf}

\begin{cf}{\small
	\begin{itemize}
	\item Avoid expensive manual configuration
	\item Adapt to changes automatically
	\end{itemize}
}
\end{cf}

\begin{cf}
What's Next?
\end{cf}

\begin{frame}\small
\textbf{CS2105R} 1MC extension\\
\textbf{CS3103} Computer Networks and Protocols\\
\textbf{CS4222} Wireless and Sensor Networks\\
\textbf{CS4274} Mobile and Multimedia Networking\\
\textbf{CS4344} Networked and Mobile Gaming\\
\textbf{CS5229} Advanced Computer Networks\\
\textbf{CS5248} Systems Support for Continuous Media
\end{frame}

\begin{cf}
Acknowledgement
\end{cf}

\begin{cf}
The End
\end{cf}

