\begin{cf}{
CS2105 Lecture 6\\
\large
\textbf{Network Layer\\[10pt]}
\normalsize
	17 February, 2014
\note{
\tiny
	After this class, you are expected to be able to
	\begin{itemize}
	\item describe the basic services the network layer and IP provides
	\item understand how datagram fragmentation works
	\item understand IP address, subnet, subnet mask, and address allocation. 
	\item understand the purposes of NAT and DHCP and how they works.
	\item understand how ping and traceroute are implemented with ICMP
	\item understand how forwarding is done in a router with longest prefix matching
	\end{itemize}
}
}
\end{cf}

\begin{frame}
	``The Internet has Ran Out of IPv4 Addresses. These Hacks, Used to Keep the Internet Growing, Are Absolutely Brilliant."
\end{frame}

\begin{cf}{
\tikzstyle{layer}=[draw,rectangle,minimum height=1.5 cm, minimum width=5 cm, text=white]
\begin{tikzpicture}[scale=2]
	\node[layer](A)[fill=AppBlue]{Application};
	\node[layer](T)[below = 0cm of A,fill=TransportBlue]{Transport};
	\node[layer](N)[below = 0cm of T,fill=NetworkGreen]{Network};
	\node[layer](L)[below = 0cm of N,fill=LinkBrown]{Link};
	\node[layer](P)[below = 0cm of L,fill=PhysicalRed]{Physical};
\end{tikzpicture}
}
\end{cf}

\tikzstyle{layer}=[draw,rectangle,minimum height=2.5 cm, minimum width=9 cm, text=white]
\tikzstyle{interface}=[draw,single arrow,fill=white, minimum width=0.9cm, minimum height=1.2cm]
\tikzstyle{message}=[draw,rectangle,minimum height=.7cm,minimum width=3cm,fill=white!90!yellow]
\tikzstyle{segment payload}=[draw,rectangle,minimum height=.7cm,minimum width=1cm,inner sep=0cm, fill=white!90!yellow]
\tikzstyle{segment header}=[draw,rectangle,minimum height=.7cm,minimum width=.4cm,inner sep=0cm, fill=white!90!red]
\tikzstyle{datagram header}=[draw,rectangle,minimum height=.7cm,minimum width=.4cm,inner sep=0cm, fill=white!50!NetworkGreen]

\newcommand{\segment}[3]{
    \node[segment header](#1)[#3]{};
    \node[segment payload](#2)[right=0cm of #1]{};
}
\newcommand{\datagramwithsegmentheader}[3]{
    \node[datagram header](#1)[#3]{};
    \node[segment header](#1-P)[right=0cm of #1]{};
    \node[segment payload](#2)[minimum width=0.6cm, right=0cm of #1-P]{};
}
\newcommand{\datagramwithoutsegmentheader}[3]{
    \node[datagram header](#1)[#3]{};
    \node[segment payload](#2)[minimum width=0.4cm, right=0cm of #1]{};
}

\begin{cf}{
	\begin{tikzpicture}[scale=2]
	\node[layer](A)[
		fill=AppBlue,
		]{};
	\node[layer](T)[
		below = 0cm of A,
		yshift = -0.1cm, 
		fill=TransportBlue]{};
    \node[message](M)[
		below=0.5cm of A.north,
	]{};
	\node[layer](N)[
		below = 0cm of T,
		yshift = -0.1cm, 
		fill=NetworkGreen]{Network};
	\node[interface,
		below=of A.south, 
		xshift=-0.5cm, 
		yshift=1cm, 
		rotate=270](FirstArrow){};
	\node[interface,
		below=of A.south, 
		xshift=0.5cm, 
		yshift=1cm, 
		rotate=90]{};
    \segment{D1}{D2}{below=1cm of FirstArrow, xshift=-2cm}
    \segment{D3}{D4}{right=0.5cm of D2}
    \segment{D5}{D6}{right=0.5cm of D4}
	\node[interface,
		below=of T.south, 
		xshift=-0.5cm, 
		yshift=1cm, 
		rotate=270](FirstArrow){};
	\node[interface,
		below=of T.south, 
		xshift=0.5cm, 
		yshift=1cm, 
		rotate=90]{};
	\end{tikzpicture}
	}
\end{cf}

\begin{cf}{
	\begin{tikzpicture}[scale=2]
	\node[layer](A)[
		fill=AppBlue,
		]{};
	\node[layer](T)[
		below = 0cm of A,
		yshift = -0.1cm, 
		fill=TransportBlue]{};
    \node[message](M)[
		below=0.5cm of A.north,
	]{};
	\node[layer](N)[
		below = 0cm of T,
		yshift = -0.1cm, 
		fill=NetworkGreen]{};
	\node[interface,
		below=of A.south, 
		xshift=-0.5cm, 
		yshift=1cm, 
		rotate=270](FirstArrow){};
	\node[interface,
		below=of A.south, 
		xshift=0.5cm, 
		yshift=1cm, 
		rotate=90]{};
	\segment{D1}{D2}{below=1cm of FirstArrow, xshift=-2cm}
	\segment{D3}{D4}{right=0.5cm of D2}
	\segment{D5}{D6}{right=0.5cm of D4}
	\node[interface,
		below=of T.south, 
		xshift=-0.5cm, 
		yshift=1cm, 
		rotate=270](Arrow2){};
	\node[interface,
		below=of T.south, 
		xshift=0.5cm, 
		yshift=1cm, 
		rotate=90]{};
	\datagramwithsegmentheader{P1}{P2}{below=1cm of Arrow2, xshift=-2.8cm}
	\datagramwithoutsegmentheader{P3}{P4}{right=0.3cm of P2}
	\datagramwithsegmentheader{P5}{P6}{right=0.3cm of P4}
	\datagramwithoutsegmentheader{P7}{P8}{right=0.3cm of P6}
	\datagramwithsegmentheader{P9}{P10}{right=0.3cm of P8}
	\datagramwithoutsegmentheader{P11}{P12}{right=0.3cm of P10}
	\end{tikzpicture}
	}
\end{cf}

\begin{cf}
The network layer resides on both end hosts and routers.
\note{
	We focus on datagram networks in CS2105.  You can read about the other alternative, virtual circuit, if you are interested, in Section 4.2.1.
}
\end{cf}

\begin{cf}{
	\begin{tikzpicture}[scale=2]
	\node[layer](T)[
		below = 0cm of A,
		yshift = -0.1cm, 
		fill=TransportBlue]{Transport};
	\node[layer](N)[
		below = 0cm of T,
		yshift = -0.1cm, 
		fill=NetworkGreen]{};
        \node[layer](IP)[
		below = 0.1cm of T,
		yshift = -0.1cm,
		xshift = -3cm,
		fill=NetworkGreen!70!white,
		minimum width=2.8cm,
		minimum height=2.3cm
	]{IP};
        \node[layer](ICMP)[
		right = 0.1cm of IP,
		fill=NetworkGreen!70!white,
		minimum width=2.8cm,
		minimum height=2.3cm
	]{ICMP};
        \node[layer](RP)[
		right = 0.1cm of ICMP,
		fill=NetworkGreen!70!white,
		minimum width=2.7cm,
		text width=2.4cm,
		align=center,
		minimum height=2.3cm
	]{\small Routing Protocols};
	\node[layer](L)[
		below = 0cm of N,
		yshift = -0.1cm, 
		fill=LinkBrown]{Link};
	\end{tikzpicture}
}
\end{cf}

\begin{cf}
IP\\
The Internet Protocol\\
\small version 4
\note{
	We focus on IPv4, which is still the dominant version of IP running today.  IPv6, however, is increasingly being deployed and will replace IPv4 eventually.  Interested students are encourage to read up Section 4.4.4 of the textbook to learn about IPv6.
}
\end{cf}

\tikzstyle{field}=[draw,rectangle,minimum height=1.5 cm, minimum width=4cm-0.4pt, text=gray!80!black,fill=white!90!NetworkGreen,font=\small]
\tikzstyle{field16}=[field,minimum width=4.0cm-0.4pt]
\tikzstyle{field12}=[field,minimum width=3.0cm-0.4pt]
\tikzstyle{field32}=[field,minimum width=8.0cm-0.4pt]
\tikzstyle{field08}=[field,minimum width=2.0cm-0.4pt]
\tikzstyle{field06}=[field,minimum width=1.5cm-0.4pt]
\tikzstyle{field04}=[field,minimum width=1.0cm-0.4pt]
\tikzstyle{field01}=[field,minimum width=0.25cm-0.4pt,inner sep=0cm]
\tikzstyle{selected}=[text=white,fill=NetworkGreen!40!black]

\begin{cf}{
	\begin{tikzpicture}
		\node[field04](A1){};
		\node[field04](A2)[right=0cm of A1]{};
		\node[field08](A3)[right=0cm of A2]{};
		\node[field16](A4)[right=0cm of A3]{};
		\node[field16](B1)[below=0cm of A2.south east]{};
		\node[field04](B2)[right=0cm of B1]{};
		\node[field12](B3)[right=0cm of B2]{};
		\node[field08](C1)[below=0cm of B1.south west,xshift=1cm]{};
		\node[field08](C2)[right=0cm of C1]{};
		\node[field16](C3)[right=0cm of C2]{};
		\node[field32](D1)[below=0cm of C3.south west]{};
		\node[field32](D2)[below=0cm of D1]{};
	\end{tikzpicture}
}
\end{cf}

\begin{cf}{
	\begin{tikzpicture}
		\node[field04](A1){};
		\node[field04](A2)[right=0cm of A1]{};
		\node[field08](A3)[right=0cm of A2]{};
		\node[field16,selected](A4)[right=0cm of A3]{length};
		\node[field16](B1)[below=0cm of A2.south east]{};
		\node[field04](B2)[right=0cm of B1]{};
		\node[field12](B3)[right=0cm of B2]{};
		\node[field08](C1)[below=0cm of B1.south west,xshift=1cm]{};
		\node[field08](C2)[right=0cm of C1]{};
		\node[field16](C3)[right=0cm of C2]{};
		\node[field32](D1)[below=0cm of C3.south west]{};
		\node[field32](D2)[below=0cm of D1]{};
	\end{tikzpicture}
	% length for data + header
}
\end{cf}

\begin{cf}{
	\begin{tikzpicture}
		\node[field04](A1){};
		\node[field04](A2)[right=0cm of A1]{};
		\node[field08](A3)[right=0cm of A2]{};
		\node[field16](A4)[right=0cm of A3]{};
		\node[field16](B1)[below=0cm of A2.south east]{};
		\node[field04](B2)[right=0cm of B1]{};
		\node[field12](B3)[right=0cm of B2]{};
		\node[field08](C1)[below=0cm of B1.south west,xshift=1cm]{};
		\node[field08,selected](C2)[right=0cm of C1]{\scriptsize protocol};
		\node[field16](C3)[right=0cm of C2]{};
		\node[field32](D1)[below=0cm of C3.south west]{};
		\node[field32](D2)[below=0cm of D1]{};
	\end{tikzpicture}
	\note{The list of protocol numbers are managed by IANA.  See \url{http://www.iana.org/assignments/protocol-numbers/protocol-numbers.xml}}
}
\end{cf}

\begin{cf}{
	\begin{tikzpicture}
		\node[field04](A1){};
		\node[field04](A2)[right=0cm of A1]{};
		\node[field08](A3)[right=0cm of A2]{};
		\node[field16](A4)[right=0cm of A3]{};
		\node[field16](B1)[below=0cm of A2.south east]{};
		\node[field04](B2)[right=0cm of B1]{};
		\node[field12](B3)[right=0cm of B2]{};
		\node[field08](C1)[below=0cm of B1.south west,xshift=1cm]{};
		\node[field08](C2)[right=0cm of C1]{};
		\node[field16,selected](C3)[right=0cm of C2]{hdr checksum};
		\node[field32](D1)[below=0cm of C3.south west]{};
		\node[field32](D2)[below=0cm of D1]{};
	\end{tikzpicture}
}
\end{cf}


\begin{cf}{
	\begin{tikzpicture}
		\node[field04](A1){};
		\node[field04](A2)[right=0cm of A1]{};
		\node[field08](A3)[right=0cm of A2]{};
		\node[field16](A4)[right=0cm of A3]{};
		\node[field16](B1)[below=0cm of A2.south east]{};
		\node[field04](B2)[right=0cm of B1]{};
		\node[field12](B3)[right=0cm of B2]{};
		\node[field08,selected](C1)[below=0cm of B1.south west,xshift=1cm]{TTL};
		\node[field08](C2)[right=0cm of C1]{};
		\node[field16](C3)[right=0cm of C2]{};
		\node[field32](D1)[below=0cm of C3.south west]{};
		\node[field32](D2)[below=0cm of D1]{};
	\end{tikzpicture}
}
\end{cf}

\begin{cf}{
	\begin{tikzpicture}
		\node[field04](A1){};
		\node[field04](A2)[right=0cm of A1]{};
		\node[field08](A3)[right=0cm of A2]{};
		\node[field16](A4)[right=0cm of A3]{};
		\node[field16,selected](B1)[below=0cm of A2.south east]{ID};
		\node[field04,selected](B2)[right=0cm of B1]{F};
		\node[field12,selected](B3)[right=0cm of B2]{offset};
		\node[field08](C1)[below=0cm of B1.south west,xshift=1cm]{};
		\node[field08](C2)[right=0cm of C1]{};
		\node[field16](C3)[right=0cm of C2]{};
		\node[field32](D1)[below=0cm of C3.south west]{};
		\node[field32](D2)[below=0cm of D1]{};
	\end{tikzpicture}
	\note{
		Flag is 1 if there is another fragment from the same segment. 0 if this is the last one.
		Offset is in multiple of 8-bytes.
	}
}
\end{cf}

\begin{cf}{
	\begin{tikzpicture}[scale=2]
	\segment{D1}{D2}{below=1cm of FirstArrow, xshift=-2cm}
	\segment{D3}{D4}{right=0.5cm of D2}
	\segment{D5}{D6}{right=0.5cm of D4}
	\datagramwithsegmentheader{P1}{P2}{below=1cm of Arrow2, xshift=-2.8cm}
	\datagramwithoutsegmentheader{P3}{P4}{right=0.3cm of P2}
	\datagramwithsegmentheader{P5}{P6}{right=0.3cm of P4}
	\datagramwithoutsegmentheader{P7}{P8}{right=0.3cm of P6}
	\datagramwithsegmentheader{P9}{P10}{right=0.3cm of P8}
	\datagramwithoutsegmentheader{P11}{P12}{right=0.3cm of P10}
	\end{tikzpicture}
	}
\end{cf}


\begin{cf}{
	\begin{tikzpicture}
		\node[field04](A1){};
		\node[field04](A2)[right=0cm of A1]{};
		\node[field08](A3)[right=0cm of A2]{};
		\node[field16](A4)[right=0cm of A3]{};
		\node[field16](B1)[below=0cm of A2.south east]{};
		\node[field04](B2)[right=0cm of B1]{};
		\node[field12](B3)[right=0cm of B2]{};
		\node[field08](C1)[below=0cm of B1.south west,xshift=1cm]{};
		\node[field08](C2)[right=0cm of C1]{};
		\node[field16](C3)[right=0cm of C2]{};
		\node[field32,selected](D1)[below=0cm of C3.south west]{source IP address};
		\node[field32,selected](D2)[below=0cm of D1]{destination IP address};
	\end{tikzpicture}
}
\end{cf}

\begin{cf}
	An IP address is\\
	associated with \textbf{an interface}.
	% An interface is the boundary between a device and the network.
\end{cf}

\begin{cf}
	4,294,967,296
\end{cf}

\begin{cf}
	ran out in 2012\\
	(but only 14\% utilized)
\end{cf}

\begin{cf}[t]
	network prefix + host identifier
\end{cf}

\begin{cf}[t]
	Classless Inter-Domain Routing (CIDR) notation
	\note{
		\small
		The IPv4 address space registry can be found at:
		\url{http://www.iana.org/assignments/ipv4-address-space/}

		You can query who owns a range of IP addresses using \texttt{whois}. For example: 
		\texttt{whois -A 137.132.0.0/16}
	}
\end{cf}

\begin{cf}
	``special'' IP addresses
	\note{
		The list of special IP addresses can be found in RFC5735.
		\url{http://tools.ietf.org/rfc/rfc5735.txt}
	}
\end{cf}

\begin{cf}
	127.0.0.0/8
\end{cf}

\begin{cf}
	10.0.0.0/8\\
  172.16.0.0/12\\
	192.168.0.0/16
\end{cf}

\begin{cf}
	169.254.0.0/16
\end{cf}

\begin{cf}
	0.0.0.0/32\\
	255.255.255.255/32
\end{cf}

\begin{cf}{
	\textbf{DHCP}\\
	Dynamic Host Configuration Protocol
	% Application layer, run over UDP
	% Besides assigning IP address, also does local DNS server, first-hop router, subnet mask.
}
\end{cf}

\begin{cf}[t]{
	1. DHCP Discover
}
\end{cf}

\begin{cf}[t]{
	2. DHCP Offer
}
\end{cf}

\begin{cf}[t]{
	3. DHCP Request
}
\end{cf}

\begin{cf}[t]{
	4. DHCP ACK
}
\end{cf}

\begin{cf}{
	\textbf{NAT}\\
	Network Address Translation 
}
\end{cf}

\tikzstyle{comp}=[draw,rectangle,minimum height=1.5 cm, minimum width=1.5cm, text=gray!80!black,fill=white!90!blue,font=\small]
\begin{cf}[t]{
\begin{tikzpicture}
	\node[comp](Comp1){};
	\node[comp](Comp2)[below=of Comp1]{};
	\node[comp](Router)[right=of Comp2]{\small Router};
	\node[cloud,draw,cloud puffs=10.4,cloud ignores aspect, minimum width=2.5cm, minimum height=1.6cm](Internet)[right=of Router]{};
	\node[comp](Server)[right=of Internet]{\small Server};
\end{tikzpicture}
}
\end{cf}

\begin{cf}{
	\begin{tikzpicture}[scale=2]
	\node[layer](T)[
		below = 0cm of A,
		yshift = -0.1cm, 
		fill=TransportBlue]{Transport};
	\node[layer](N)[
		below = 0cm of T,
		yshift = -0.1cm, 
		fill=NetworkGreen]{};
        \node[layer](IP)[
		below = 0.1cm of T,
		yshift = -0.1cm,
		xshift = -3cm,
		fill=NetworkGreen!70!white,
		minimum width=2.8cm,
		minimum height=2.3cm
	]{IP};
        \node[layer](ICMP)[
		right = 0.1cm of IP,
		fill=NetworkGreen!70!white,
		minimum width=2.8cm,
		minimum height=2.3cm
	]{ICMP};
        \node[layer](RP)[
		right = 0.1cm of ICMP,
		fill=NetworkGreen!70!white,
		minimum width=2.7cm,
		text width=2.4cm,
		align=center,
		minimum height=2.3cm
	]{\small Routing Protocols};
	\node[layer](L)[
		below = 0cm of N,
		yshift = -0.1cm, 
		fill=LinkBrown]{Link};
	\end{tikzpicture}
}
\end{cf}

\begin{cf}{
	\textbf{ICMP}\\
	Internet Control Message Protocol
}
% above IP layer
\end{cf}


\begin{cf}[t]
	ping
\end{cf}

\begin{cf}[t]
	traceroute
\end{cf}

%\tikzstyle{router}=[draw,cylinder,aspect=1.5,rotate=90,minimum height=2.5 cm, minimum width=3.5cm, text=gray!80!black,fill=white!90!NetworkGreen,font=\small]
%\begin{cf}{
%	\begin{tikzpicture}
%		\node[router](Router){};
%		\node[draw,single arrow](L)[left=1cm of Router.north]{};
%		\node[draw,single arrow](R)[right=1cm of Router.south]{};
%		\node[draw,single arrow, rotate=-45](L1)[above=of L]{};
%		\node[draw,single arrow, rotate=45](R1)[above=of R]{};
%		\node[draw,single arrow, rotate=45](L2)[below=of L]{};
%		\node[draw,single arrow, rotate=-45](R2)[below=of R]{};
%	\end{tikzpicture}
%}
%\end{cf}
%
%\begin{cf}[t]
%	forwarding table
%\end{cf}
%
%\begin{cf}[t]
%	longest prefix matching
%\end{cf}
