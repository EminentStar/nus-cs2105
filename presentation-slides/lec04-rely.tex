\newcommand{\eventaction}[2]{%
$\frac{\mbox{\normalsize\color{blue!50!black}{#1}}}%
      {\mbox{\normalsize\color{red!50!black}{#2}}}$}

\tikzstyle{state}=[draw,circle,text=white,align=center,font=\scriptsize,minimum width=2.5cm]
\tikzstyle{bluestate}=[state,fill=blue!50!black]
\tikzstyle{greenstate}=[state,fill=green!50!black]
\tikzstyle{redstate}=[state,fill=red!50!black]
\tikzstyle{brownstate}=[state,fill=brown!50!black]
\tikzstyle{orangestate}=[state,fill=orange!50!black]
\tikzstyle{blackstate}=[state,fill=gray!50!black]

\begin{cf}{
CS2105 Lecture 4\\
\large
\textbf{Reliable Protocols\\[10pt]}
\normalsize
	3 February, 2014
\note{
\tiny
	After this class, you are expected to:
	\begin{itemize}
	\item understand the interface between the transport layer and the network layer
	\item be able to design your own reliable protocols with ACK, NAK, sequence numbers, timeout, and retransmission.
	\item know how to calculate the utilization of a channel.
	\item understand the working of Go-Back-N and Selective Repeat protocols
	\end{itemize}
}
}
\end{cf}

\begin{frame}
``Sending Data Reliably over the Internet is Harder than You Think.  The Intricacy Involved in Ensuring Reliability Will Make Your Head Explode.''
\end{frame}

\begin{frame}
\begin{center}
\tikzstyle{layer}=[draw,rectangle,minimum height=1.5 cm, minimum width=5 cm, text=white]
	\begin{tikzpicture}[scale=2]
	\node[layer](A)[fill=AppBlue]{Application};
	\node[layer](T)[below = 0cm of A,fill=TransportBlue]{Transport};
	\node[layer](N)[below = 0cm of T,fill=NetworkGreen]{Network};
	\node[layer](L)[below = 0cm of N,fill=LinkBrown]{Link};
	\node[layer](P)[below = 0cm of L,fill=PhysicalRed]{Physical};
	\end{tikzpicture}
\end{center}
\end{frame}

\begin{cf}
Transport layer resides \textbf{on end hosts}
and provides \textbf{process-to-process} communication.
\end{cf}

\begin{cf}
Network layer provides 
\textbf{host-to-host},\\
\textbf{best-effort},\\
\textbf{unreliable}\\
communication.
\end{cf}

\begin{cf}{
\tikzstyle{layer}=[draw,rectangle,minimum height=2.5 cm, minimum width=9 cm, text=white]
\tikzstyle{interface}=[draw,single arrow,fill=white, minimum width=0.9cm, minimum height=1.2cm]
\tikzstyle{message}=[draw,rectangle,minimum height=.7cm,minimum width=3cm,fill=white!90!yellow]
\tikzstyle{datagram payload}=[draw,rectangle,minimum height=.7cm,minimum width=1cm,fill=white!90!yellow]
\tikzstyle{datagram header}=[draw,rectangle,minimum height=.7cm,minimum width=.5cm,fill=white!90!red]
	\begin{tikzpicture}[scale=2]
	\node[layer](A)[
		fill=AppBlue,
		]{};
	\node[layer](T)[
		below = 0cm of A,
		yshift = -0.1cm, 
		fill=TransportBlue]{};
    \node[message](M)[
		below=0.5cm of A.north,
	]{};
	\node[layer](N)[
		below = 0cm of T,
		yshift = -0.1cm, 
		fill=NetworkGreen]{Network};
	\node[interface,
		below=of A.south, 
		xshift=-0.5cm, 
		yshift=1cm, 
		rotate=270](FirstArrow){};
	\node[interface,
		below=of A.south, 
		xshift=0.5cm, 
		yshift=1cm, 
		rotate=90]{};
    \node[datagram header](D1)[
		below=1cm of FirstArrow,
		xshift=-2cm
	]{};
    \node[datagram payload](D2)[
		right=0cm of D1,
	]{};
    \node[datagram header](D3)[
		right=0.5cm of D2,
	]{};
    \node[datagram payload](D4)[
		right=0cm of D3,
	]{};
    \node[datagram header](D5)[
		right=0.5cm of D4,
	]{};
    \node[datagram payload](D6)[
		right=0cm of D5,
	]{};
	\node[interface,
		below=of T.south, 
		xshift=-0.5cm, 
		yshift=1cm, 
		rotate=270](FirstArrow){};
	\node[interface,
		below=of T.south, 
		xshift=0.5cm, 
		yshift=1cm, 
		rotate=90]{};
	\end{tikzpicture}
	}
\end{cf}

\begin{cf}
How to build a \textbf{reliable transfer protocol}
on top of unreliable communication?
\end{cf}

\begin{cf}
Unreliable: may not deliver at all or deliver with error
\end{cf}

\begin{cf}{
\tikzstyle{layer}=[draw,rectangle,minimum height=2.5 cm, minimum width=4 cm, text=white]
\tikzstyle{interface}=[draw,single arrow,fill=white, minimum width=0.9cm, minimum height=1.2cm]
\tikzstyle{message}=[draw,rectangle,minimum height=.7cm,minimum width=1cm,fill=white!90!yellow]
\tikzstyle{datagram payload}=[draw,rectangle,minimum height=.7cm,minimum width=1cm,fill=white!90!yellow]
\tikzstyle{datagram header}=[draw,rectangle,minimum height=.7cm,minimum width=.5cm,fill=white!90!red]
	\begin{tikzpicture}[scale=2]
	\node[layer](A1)[
		fill=AppBlue,
		]{};
	\node[layer](A2)[
		fill=AppBlue,
		right=0.4cm of A1,
		]{};
	\node[layer](T1)[
		below = 0cm of A1,
		yshift = -0.1cm, 
		fill=TransportBlue]{};
	\node[layer](T2)[
		below = 0cm of A2,
		yshift = -0.1cm, 
		fill=TransportBlue]{};
	\node[layer,
		below = 0cm of T1,
		yshift = -0.1cm, 
		xshift = 2.2cm,
		minimum width=8.4cm,
		fill=NetworkGreen]{};
	\node[interface,
		below=of A1.south, 
		xshift=-0.5cm, 
		yshift=1cm, 
		rotate=270](){};
    \node[text=white,
		above=0cm of A1.south,
		xshift=.5cm]{
		\tiny rdt\_send()};
    \node[text=white,
		above=0cm of A2.south,
		xshift=-.7cm]{
		\tiny deliver\_data()};
    \node[message,
		right=of A1.west,
	]{};
    \node[message,
		left=of A2.east,
	]{};
	\node[interface,
		below=of A2.south, 
		xshift=0.5cm, 
		yshift=1cm, 
		rotate=90](ATup){};
    \node[datagram header](D1)[
		right=of T1.west,
	]{};
    \node[datagram payload](D2)[
		right=0cm of D1,
	]{};
    \node[datagram payload](D3)[
		left=of T2.east,
	]{};
    \node[datagram header](D4)[
		left=0cm of D3,
	]{};
	\node[interface,
		below=of T1.south, 
		xshift=-0.5cm, 
		yshift=1cm, 
		rotate=270](){};
    \node[text=white,
		above=0cm of T1.south,
		xshift=.5cm]{
		\tiny udt\_send()};
	\node[interface,
		below=of T2.south, 
		xshift=0.5cm, 
		yshift=1cm, 
		rotate=90](TNup){};
    \node[text=white,
		above=0cm of T2.south,
		xshift=-.5cm]{
		\tiny rdt\_recv()};
	\end{tikzpicture}
	}
\end{cf}

\begin{cf}
Finite State Machine\\[20pt]

\begin{tikzpicture}[scale=2]
  \node[brownstate,initial above,initial distance=0.5cm,initial text=](A){A};
  \node[bluestate](B)[right=of A]{B};
  \path[->] (A) edge [loop left] node {} (A)
        edge [bend left] node {} (B)
    (B) edge [loop right] node {} (B)
        edge [bend left] node {} (A);
\end{tikzpicture}\\[20pt]

\eventaction{Event}{Action}

\end{cf}

\begin{cf}{
		\textbf{rdt 1.0}\\
		Assume underlying channel is reliable
}
\end{cf}

\begin{cf}{
\textbf{rdt 1.0 sender}\\[20pt]
\begin{tikzpicture}[scale=2]
  \node[bluestate,initial above,initial distance=0.5cm,initial text=,font=\scriptsize](A){wait\\for\\call};
	\path[->,align=center] (A) edge [loop right] node {} (A);
\end{tikzpicture}\\[20pt]
}
\end{cf}

\begin{cf}{
\textbf{rdt 1.0 receiver}\\[20pt]
\begin{tikzpicture}[scale=2]
  \node[bluestate,initial above,initial distance=0.5cm,initial text=,](A){wait\\for\\call};
	\path[->,align=center] (A) edge [loop right] node {} (A);
\end{tikzpicture}\\[20pt]
}
\end{cf}

\begin{cf}{
		\textbf{rdt 2.0}\\
		Underlying channel may introduce bit errors
		}
\end{cf}

\begin{cf}{
		\textbf{ARQ}\\
		Automated Repeat reQuest
		}
\end{cf}

\begin{cf}{
		Receiver detects errors\\
		Receiver feeds back to sender\\
		Sender retransmits
		}
\note{We will talk about checksum, the technique to detect errors, in more detail in the subsequent lectures.}
\end{cf}

\begin{cf}{
\textbf{rdt 2.0 sender}\\
\begin{tikzpicture}[scale=2]
  \node[bluestate,initial above,initial distance=0.5cm,initial text=](A){wait\\for\\call};
  \node[greenstate,right=2cm of A](B){wait\\for\\ack/nak};
  \path[->] 
    (A) edge [bend left] node {} (B)
    (B) edge [loop right] node {} (B)
        edge [bend left] node {} (A);
\end{tikzpicture}\\[20pt]
}
\end{cf}

\begin{cf}{
\textbf{rdt 2.0 receiver}\\
\begin{tikzpicture}[scale=2]
  \node[redstate,initial above,initial distance=0.5cm,initial text=,](A){wait\\for\\call};
	\path[->,align=center] (A) edge [loop right] node {} (A);
	\path[->,align=center] (A) edge [loop left] node {} (A);
\end{tikzpicture}\\[20pt]
}
\end{cf}

\begin{cf}{
	\begin{tikzpicture}[scale=2]
		\draw[solid] (0,0) -- (0,3);
		\draw[solid] (2,0) -- (2,3);
	\end{tikzpicture}
	}
\end{cf}


\begin{cf}{
		\Large
		\textbf{Stop-and-Wait Protocol}
	}
\end{cf}

\begin{cf}
		Bug:\\ What if ACK/NAK is corrupted?
\end{cf}

\begin{cf}{
	\begin{tikzpicture}[scale=2]
		\draw[solid] (0,0) -- (0,3);
		\draw[solid] (2,0) -- (2,3);
	\end{tikzpicture}
	}
\end{cf}


\begin{cf}
		Bug Fix:\\ Add a sequence number
\end{cf}

\begin{cf}{
	\begin{tikzpicture}[scale=2]
		\draw[solid] (0,0) -- (0,3);
		\draw[solid] (2,0) -- (2,3);
	\end{tikzpicture}
	}
\end{cf}


\begin{cf}{
\textbf{rdt 2.1 sender}\\
\begin{tikzpicture}[scale=2]
  \node[bluestate,initial above,initial distance=0.5cm,initial text=](A){wait\\for\\call 0};
  \node[greenstate,right=2cm of A](B){wait\\for ack\\/nak 0};
  \node[state,minimum width=0.1cm,draw=none](null1)[below=of A]{};
  \node[state,minimum width=0.1cm,draw=none](null2)[below=of B]{};
  \path[->] 
    (A) edge [bend left] node {} (B)
    (B) edge [loop right] node {} (B)
        edge node {} (null2)
    (null1) edge node {} (A);
\end{tikzpicture}\\[20pt]
}
\end{cf}

\begin{cf}{
\begin{tikzpicture}[scale=2]
  \node[bluestate,initial above,initial distance=0.2cm,initial text=](A){wait\\for\\call 0};
  \node[greenstate,right=2cm of A](B){wait\\for ack\\/nak 0};
  \node[redstate](C)[below=of A]{wait\\for ack\\/nak 1};
  \node[brownstate](D)[below=of B]{wait\\for\\call 1};
  \path[->] 
    (A) edge [bend left] node {} (B)
    (D) edge [bend left] node {} (C)
    (B) edge [loop right] node {} (B)
        edge node {} (D)
    (C) edge [loop left] node {} (C)
	    edge node {} (A);
\end{tikzpicture}\\[20pt]
}
\end{cf}

\begin{cf}{
\textbf{rdt 2.1 receiver}\\
\begin{tikzpicture}[scale=2]
  \node[orangestate,initial left,initial distance=0.5cm,initial text=](A){wait\\for\\0};
  \node[blackstate,right=2cm of A](B){wait\\for\\1};
  \path[->] 
    (A) edge [loop above] node {} (A)
        edge [loop below] node {} (A)
        edge [bend left] node {} (B)
    (B) edge [loop above] node {} (B)
	    edge [loop below] node {} (B)
        edge [bend left] node {} (A);
\end{tikzpicture}\\[20pt]
}
\end{cf}

\begin{cf}{
	\begin{tikzpicture}[scale=2]
		\draw[solid] (0,0) -- (0,3);
		\draw[solid] (2,0) -- (2,3);
	\end{tikzpicture}
	}
\end{cf}


\begin{cf}{
		\textbf{rdt 2.2}\\
		Replace NAK with ACK of the last correctly received packet.
		}
\end{cf}

\begin{cf}{
	\begin{tikzpicture}[scale=2]
		\draw[solid] (0,0) -- (0,3);
		\draw[solid] (2,0) -- (2,3);
	\end{tikzpicture}
	}
\end{cf}


\begin{cf}{
		\textbf{rdt 3.0}\\
		Packet can be loss or corrupted
		}
\end{cf}

\begin{cf}{
		Challenge: If ACK is lost, how to tell if the packet has been received?
		}
\end{cf}

\begin{cf}{
	\begin{tikzpicture}[scale=2]
		\draw[solid] (0,0) -- (0,3);
		\draw[solid] (2,0) -- (2,3);
	\end{tikzpicture}
	}
\end{cf}


\begin{cf}{
		Resend after timeout\\
		(may lead to dups, but OK since we have seq numbers).
		}
\end{cf}

\begin{cf}{
	\begin{tikzpicture}[scale=2]
		\draw[solid] (0,0) -- (0,3);
		\draw[solid] (2,0) -- (2,3);
	\end{tikzpicture}
	}
\end{cf}


\begin{cf}{
\textbf{rdt 3.0 sender}\\
\begin{tikzpicture}[scale=2]
  \node[bluestate,initial above,initial distance=0.5cm,initial text=](A){wait\\for\\call 0};
  \node[greenstate,right=2cm of A](B){wait\\for\\ ack 0};
  \node[state,minimum width=0.1cm,draw=none](null1)[below=of A]{};
  \node[state,minimum width=0.1cm,draw=none](null2)[below=of B]{};
  \path[->] (A) edge [loop left] node {} (A)
        edge [bend left] node {} (B)
    (B) edge [loop right] node {} (B)
        edge [loop above] node {} (B)
        edge node {} (null2)
    (null1) edge node {} (A);
\end{tikzpicture}\\[20pt]
\note{The state diagram for the receiver in rdt 3.0 is given as an exercise.}
}
\end{cf}

\begin{cf}{
	\begin{tikzpicture}[scale=2]
		\draw[solid] (0,0) -- (0,3);
		\draw[solid] (2,0) -- (2,3);
	\end{tikzpicture}
	}
\end{cf}


\begin{cf}{
		\textbf{Alternating Bit Protocol}
	}
\end{cf}

\begin{cf}[t]{
		RTT = 30ms\\
		R = 1 Gbps\\
		L = 1000 bytes\\
	}
\end{cf}

\begin{cf}{
	\begin{tikzpicture}[scale=2]
		\draw[solid] (0,0) -- (0,3);
		\draw[solid] (2,0) -- (2,3);
	\end{tikzpicture}
	}
\end{cf}


\begin{cf}{
\textbf{Pipelining}:\\
	need buffering and larger range of sequence numbers.
}
\end{cf}

\begin{cf}{
	\begin{tikzpicture}[scale=2]
		\draw[solid] (0,0) -- (0,3);
		\draw[solid] (2,0) -- (2,3);
	\end{tikzpicture}
	}
\end{cf}


\begin{cf}{
\textbf{Go-Back-N}
}
\end{cf}

\begin{cf}{
	\begin{tikzpicture}[scale=2]
		\draw[solid] (0,0) -- (0,3);
		\draw[solid] (2,0) -- (2,3);
	\end{tikzpicture}
	}
\end{cf}


\begin{cf}{
	\begin{tikzpicture}[scale=2]
		\draw[solid] (0,0) -- (0,3);
		\draw[solid] (2,0) -- (2,3);
	\end{tikzpicture}
	}
\end{cf}


\tikzstyle{packet}=[draw,rectangle,rounded corners,
	minimum height=2 cm, minimum width=.6 cm,text=white] 
\tikzstyle{window}=[rectangle,minimum height=2.5 cm, minimum width=4.9 cm,fill=gray!25!white] 
\tikzstyle{acked}=[packet,fill=green!30!white]
\tikzstyle{sent}=[packet,fill=yellow!30!white]
\tikzstyle{can-send}=[packet,fill=blue!30!white]
\tikzstyle{cannot-send}=[packet,fill=red!30!white]

\begin{cf}[t]{
\begin{tikzpicture}[scale=2]
	\node[acked](A1){};
	\node[acked](A2)[right=.2cm of A1]{};
	\node[acked](A3)[right=.2cm of A2]{};
	\node[window](W)[right=.1cm of A3]{};
	\node[sent](S1)[right=.2cm of A3]{};
	\node[sent](S2)[right=.2cm of S1]{};
	\node[sent](S3)[right=.2cm of S2]{};
	\node[sent](S4)[right=.2cm of S3]{};
	\node[can-send](C1)[right=.2cm of S4]{};
	\node[can-send](C2)[right=.2cm of C1]{};
	\node[cannot-send](N1)[right=.2cm of C2]{};
	\node[cannot-send](N2)[right=.2cm of N1]{};
	\node[cannot-send](N3)[right=.2cm of N2]{};
\end{tikzpicture}
}
\end{cf}

\begin{cf}[t]{
\begin{tikzpicture}[scale=2]
	\node[acked](A1){};
	\node[acked](A2)[right=.2cm of A1]{};
	\node[acked](A3)[right=.2cm of A2]{};
	\node[window](W)[right=.1cm of A3]{};
	\node[sent](S1)[right=.2cm of A3]{};
	\node[sent](S2)[right=.2cm of S1]{};
	\node[sent](S3)[right=.2cm of S2]{};
	\node[sent](S4)[right=.2cm of S3]{};
	\node[can-send](C1)[right=.2cm of S4]{};
	\node[can-send](C2)[right=.2cm of C1]{};
	\node[cannot-send](N1)[right=.2cm of C2]{};
	\node[cannot-send](N2)[right=.2cm of N1]{};
	\node[cannot-send](N3)[right=.2cm of N2]{};
\end{tikzpicture}\\[7pt]
send a packet\\[10pt]
\begin{tikzpicture}[scale=2]
	\node[acked](A1){};
	\node[acked](A2)[right=.2cm of A1]{};
	\node[acked](A3)[right=.2cm of A2]{};
	\node[window](W)[right=.1cm of A3]{};
	\node[sent](S1)[right=.2cm of A3]{};
	\node[sent](S2)[right=.2cm of S1]{};
	\node[sent](S3)[right=.2cm of S2]{};
	\node[sent](S4)[right=.2cm of S3]{};
	\node[sent](C1)[right=.2cm of S4]{};
	\node[can-send](C2)[right=.2cm of C1]{};
	\node[cannot-send](N1)[right=.2cm of C2]{};
	\node[cannot-send](N2)[right=.2cm of N1]{};
	\node[cannot-send](N3)[right=.2cm of N2]{};
\end{tikzpicture}
}
\end{cf}

\begin{cf}[t]{
\begin{tikzpicture}[scale=2]
	\node[acked](A1){};
	\node[acked](A2)[right=.2cm of A1]{};
	\node[acked](A3)[right=.2cm of A2]{};
	\node[window](W)[right=.1cm of A3]{};
	\node[sent](S1)[right=.2cm of A3]{};
	\node[sent](S2)[right=.2cm of S1]{};
	\node[sent](S3)[right=.2cm of S2]{};
	\node[sent](S4)[right=.2cm of S3]{};
	\node[sent](C1)[right=.2cm of S4]{};
	\node[can-send](C2)[right=.2cm of C1]{};
	\node[cannot-send](N1)[right=.2cm of C2]{};
	\node[cannot-send](N2)[right=.2cm of N1]{};
	\node[cannot-send](N3)[right=.2cm of N2]{};
\end{tikzpicture}\\[7pt]
receive ACK 3\\[10pt]
\begin{tikzpicture}[scale=2]
	\node[acked](A1){};
	\node[acked](A2)[right=.2cm of A1]{};
	\node[acked](A3)[right=.2cm of A2]{};
	\node[acked](S1)[right=.2cm of A3]{};
	\node[window](W)[right=.1cm of S1]{};
	\node[sent](S2)[right=.2cm of S1]{};
	\node[sent](S3)[right=.2cm of S2]{};
	\node[sent](S4)[right=.2cm of S3]{};
	\node[sent](C1)[right=.2cm of S4]{};
	\node[can-send](C2)[right=.2cm of C1]{};
	\node[can-send](N1)[right=.2cm of C2]{};
	\node[cannot-send](N2)[right=.2cm of N1]{};
	\node[cannot-send](N3)[right=.2cm of N2]{};
\end{tikzpicture}
}
\end{cf}

\begin{cf}[t]{
\begin{tikzpicture}[scale=2]
	\node[acked](A1){};
	\node[acked](A2)[right=.2cm of A1]{};
	\node[acked](A3)[right=.2cm of A2]{};
	\node[acked](S1)[right=.2cm of A3]{};
	\node[window](W)[right=.1cm of S1]{};
	\node[sent](S2)[right=.2cm of S1]{};
	\node[sent](S3)[right=.2cm of S2]{};
	\node[sent](S4)[right=.2cm of S3]{};
	\node[sent](C1)[right=.2cm of S4]{};
	\node[can-send](C2)[right=.2cm of C1]{};
	\node[can-send](N1)[right=.2cm of C2]{};
	\node[cannot-send](N2)[right=.2cm of N1]{};
	\node[cannot-send](N3)[right=.2cm of N2]{};
\end{tikzpicture}\\[7pt]
receive ACK 5\\[10pt]
\begin{tikzpicture}[scale=2]
	\node[acked](A1){};
	\node[acked](A2)[right=.2cm of A1]{};
	\node[acked](A3)[right=.2cm of A2]{};
	\node[acked](S1)[right=.2cm of A3]{};
	\node[acked](S2)[right=.2cm of S1]{};
	\node[acked](S3)[right=.2cm of S2]{};
	\node[window](W)[right=.1cm of S3]{};
	\node[sent](S4)[right=.2cm of S3]{};
	\node[sent](C1)[right=.2cm of S4]{};
	\node[can-send](C2)[right=.2cm of C1]{};
	\node[can-send](N1)[right=.2cm of C2]{};
	\node[can-send](N2)[right=.2cm of N1]{};
	\node[can-send](N3)[right=.2cm of N2]{};
\end{tikzpicture}
}
\end{cf}

\begin{cf}[t]{
\begin{tikzpicture}[scale=2]
	\node[acked](A1){};
	\node[acked](A2)[right=.2cm of A1]{};
	\node[acked](A3)[right=.2cm of A2]{};
	\node[window](W)[right=.1cm of A3]{};
	\node[sent](S1)[right=.2cm of A3]{};
	\node[sent](S2)[right=.2cm of S1]{};
	\node[sent](S3)[right=.2cm of S2]{};
	\node[sent](S4)[right=.2cm of S3]{};
	\node[sent](C1)[right=.2cm of S4]{};
	\node[sent](C2)[right=.2cm of C1]{};
	\node[cannot-send](N1)[right=.2cm of C2]{};
	\node[cannot-send](N2)[right=.2cm of N1]{};
	\node[cannot-send](N3)[right=.2cm of N2]{};
\end{tikzpicture}
}\\[7pt]
window is full
\end{cf}

\begin{cf}[t]{
\begin{tikzpicture}[scale=2]
	\node[acked](A1){};
	\node[acked](A2)[right=.2cm of A1]{};
	\node[acked](A3)[right=.2cm of A2]{};
	\node[window](W)[right=.1cm of A3]{};
	\node[can-send](S1)[right=.2cm of A3]{};
	\node[can-send](S2)[right=.2cm of S1]{};
	\node[can-send](S3)[right=.2cm of S2]{};
	\node[can-send](S4)[right=.2cm of S3]{};
	\node[can-send](C1)[right=.2cm of S4]{};
	\node[can-send](C2)[right=.2cm of C1]{};
	\node[cannot-send](N1)[right=.2cm of C2]{};
	\node[cannot-send](N2)[right=.2cm of N1]{};
	\node[cannot-send](N3)[right=.2cm of N2]{};
\end{tikzpicture}
}\\[7pt]
window is empty
\end{cf}

\begin{cf}{
\textbf{Selective Repeat}
}
\end{cf}

\begin{cf}{
	\begin{tikzpicture}[scale=2]
		\draw[solid] (0,0) -- (0,3);
		\draw[solid] (2,0) -- (2,3);
	\end{tikzpicture}
	}
\end{cf}


\begin{cf}{
1. one timer per packet\\
2. receiver needs buffer
}
\end{cf}

\begin{cf}[t]{
at sender:\\[10pt]
\begin{tikzpicture}[scale=2]
	\node[acked](A1){};
	\node[acked](A2)[right=.2cm of A1]{};
	\node[acked](A3)[right=.2cm of A2]{};
	\node[window](W)[right=.1cm of A3]{};
	\node[sent](S1)[right=.2cm of A3]{};
	\node[acked](S2)[right=.2cm of S1]{};
	\node[sent](S3)[right=.2cm of S2]{};
	\node[sent](S4)[right=.2cm of S3]{};
	\node[can-send](C1)[right=.2cm of S4]{};
	\node[can-send](C2)[right=.2cm of C1]{};
	\node[cannot-send](N1)[right=.2cm of C2]{};
	\node[cannot-send](N2)[right=.2cm of N1]{};
	\node[cannot-send](N3)[right=.2cm of N2]{};
\end{tikzpicture}\\[7pt]
receive ACK 5\\[10pt]
\begin{tikzpicture}[scale=2]
	\node[acked](A1){};
	\node[acked](A2)[right=.2cm of A1]{};
	\node[acked](A3)[right=.2cm of A2]{};
	\node[window](W)[right=.1cm of A3]{};
	\node[sent](S1)[right=.2cm of A3]{};
	\node[acked](S2)[right=.2cm of S1]{};
	\node[acked](S3)[right=.2cm of S2]{};
	\node[sent](S4)[right=.2cm of S3]{};
	\node[can-send](C1)[right=.2cm of S4]{};
	\node[can-send](C2)[right=.2cm of C1]{};
	\node[cannot-send](N1)[right=.2cm of C2]{};
	\node[cannot-send](N2)[right=.2cm of N1]{};
	\node[cannot-send](N3)[right=.2cm of N2]{};
\end{tikzpicture}\\[7pt]
}
\end{cf}

\begin{cf}[t]{
at sender:\\[10pt]
\begin{tikzpicture}[scale=2]
	\node[acked](A1){};
	\node[acked](A2)[right=.2cm of A1]{};
	\node[acked](A3)[right=.2cm of A2]{};
	\node[window](W)[right=.1cm of A3]{};
	\node[sent](S1)[right=.2cm of A3]{};
	\node[acked](S2)[right=.2cm of S1]{};
	\node[acked](S3)[right=.2cm of S2]{};
	\node[sent](S4)[right=.2cm of S3]{};
	\node[can-send](C1)[right=.2cm of S4]{};
	\node[can-send](C2)[right=.2cm of C1]{};
	\node[cannot-send](N1)[right=.2cm of C2]{};
	\node[cannot-send](N2)[right=.2cm of N1]{};
	\node[cannot-send](N3)[right=.2cm of N2]{};
\end{tikzpicture}\\[7pt]
receive ACK 3\\[10pt]
\begin{tikzpicture}[scale=2]
	\node[acked](A1){};
	\node[acked](A2)[right=.2cm of A1]{};
	\node[acked](A3)[right=.2cm of A2]{};
	\node[acked](S1)[right=.2cm of A3]{};
	\node[acked](S2)[right=.2cm of S1]{};
	\node[acked](S3)[right=.2cm of S2]{};
	\node[window](W)[right=.1cm of S3]{};
	\node[sent](S4)[right=.2cm of S3]{};
	\node[can-send](C1)[right=.2cm of S4]{};
	\node[can-send](C2)[right=.2cm of C1]{};
	\node[cannot-send](N1)[right=.2cm of C2]{};
	\node[cannot-send](N2)[right=.2cm of N1]{};
	\node[cannot-send](N3)[right=.2cm of N2]{};
\end{tikzpicture}\\[7pt]
}
\end{cf}

\tikzstyle{delivered}=[packet,fill=green!70!white]
\tikzstyle{recved}=[packet,fill=yellow!70!white]
\tikzstyle{can-recv}=[packet,fill=blue!30!white]
\tikzstyle{cannot-recv}=[packet,fill=red!70!white]

\begin{cf}[t]{
at receiver:\\[10pt]
\begin{tikzpicture}[scale=2]
	\node[delivered](A1){};
	\node[delivered](A2)[right=.2cm of A1]{};
	\node[delivered](A3)[right=.2cm of A2]{};
	\node[window](W)[right=.1cm of A3]{};
	\node[can-recv](S1)[right=.2cm of A3]{};
	\node[recved](S2)[right=.2cm of S1]{};
	\node[can-recv](S3)[right=.2cm of S2]{};
	\node[can-recv](S4)[right=.2cm of S3]{};
	\node[can-recv](C1)[right=.2cm of S4]{};
	\node[can-recv](C2)[right=.2cm of C1]{};
	\node[cannot-recv](N1)[right=.2cm of C2]{};
	\node[cannot-recv](N2)[right=.2cm of N1]{};
	\node[cannot-recv](N3)[right=.2cm of N2]{};
\end{tikzpicture}
}
\end{cf}

\begin{cf}[t]{
at receiver:\\[10pt]
\begin{tikzpicture}[scale=2]
	\node[delivered](P1){};
	\node[delivered](P2)[right=.2cm of P1]{};
	\node[delivered](P3)[right=.2cm of P2]{};
	\node[window](W)[right=.1cm of P3]{};
	\node[can-recv](P4)[right=.2cm of P3]{};
	\node[recved](P5)[right=.2cm of P4]{};
	\node[can-recv](P6)[right=.2cm of P5]{};
	\node[can-recv](P7)[right=.2cm of P6]{};
	\node[can-recv](P8)[right=.2cm of P7]{};
	\node[can-recv](P9)[right=.2cm of P8]{};
	\node[cannot-recv](P10)[right=.2cm of P9]{};
	\node[cannot-recv](P11)[right=.2cm of P10]{};
	\node[cannot-recv](P12)[right=.2cm of P11]{};
\end{tikzpicture}\\[7pt]
receive packet 5
\begin{tikzpicture}[scale=2]
	\node[delivered](P1){};
	\node[delivered](P2)[right=.2cm of P1]{};
	\node[delivered](P3)[right=.2cm of P2]{};
	\node[window](W)[right=.1cm of P3]{};
	\node[can-recv](P4)[right=.2cm of P3]{};
	\node[recved](P5)[right=.2cm of P4]{};
	\node[recved](P6)[right=.2cm of P5]{};
	\node[can-recv](P7)[right=.2cm of P6]{};
	\node[can-recv](P8)[right=.2cm of P7]{};
	\node[can-recv](P9)[right=.2cm of P8]{};
	\node[cannot-recv](P10)[right=.2cm of P9]{};
	\node[cannot-recv](P11)[right=.2cm of P10]{};
	\node[cannot-recv](P12)[right=.2cm of P11]{};
\end{tikzpicture}
}
\end{cf}

\begin{cf}[t]{
at receiver:\\[10pt]
\begin{tikzpicture}[scale=2]
	\node[delivered](P1){};
	\node[delivered](P2)[right=.2cm of P1]{};
	\node[delivered](P3)[right=.2cm of P2]{};
	\node[window](W)[right=.1cm of P3]{};
	\node[can-recv](P4)[right=.2cm of P3]{};
	\node[recved](P5)[right=.2cm of P4]{};
	\node[recved](P6)[right=.2cm of P5]{};
	\node[can-recv](P7)[right=.2cm of P6]{};
	\node[can-recv](P8)[right=.2cm of P7]{};
	\node[can-recv](P9)[right=.2cm of P8]{};
	\node[cannot-recv](P10)[right=.2cm of P9]{};
	\node[cannot-recv](P11)[right=.2cm of P10]{};
	\node[cannot-recv](P12)[right=.2cm of P11]{};
\end{tikzpicture}\\[7pt]
receive packet 3
\begin{tikzpicture}[scale=2]
	\node[delivered](P1){};
	\node[delivered](P2)[right=.2cm of P1]{};
	\node[delivered](P3)[right=.2cm of P2]{};
	\node[delivered](P4)[right=.2cm of P3]{};
	\node[delivered](P5)[right=.2cm of P4]{};
	\node[delivered](P6)[right=.2cm of P5]{};
	\node[window](W)[right=.1cm of P6]{};
	\node[can-recv](P7)[right=.2cm of P6]{};
	\node[can-recv](P8)[right=.2cm of P7]{};
	\node[can-recv](P9)[right=.2cm of P8]{};
	\node[can-recv](P10)[right=.2cm of P9]{};
	\node[can-recv](P11)[right=.2cm of P10]{};
	\node[can-recv](P12)[right=.2cm of P11]{};
\end{tikzpicture}
}
\end{cf}

\begin{cf}{
	\begin{tikzpicture}[scale=2]
		\draw[solid] (0,0) -- (0,3);
		\draw[solid] (1.5,0) -- (1.5,3);
		\draw[solid] (2,0) -- (2,3);
		\draw[solid] (3.5,0) -- (3.5,3);
	\end{tikzpicture}
	}
\end{cf}



\begin{cf}{\small
	\begin{tabular}{|r|c|c|}
	\hline
	        & GBN & SR \\
	\hline
	ACK	& cumulative & selective \\
	out-of-order & ignore & keep \\
	retransmit	& all unack & one unack \\
	timer &	earliest unack & one per unack \\
	\hline
	\end{tabular}
}
\end{cf}

\begin{cf}
How to build a \textbf{reliable protocol}
on top of unreliable communication?
\note{The same techniques can be used in other layers, such as in an application layer protocol over UDP, or in a link-layer protocol.}
\end{cf}

\begin{cf}\normalsize
error detection\\
retransmission\\
timers\\
sequence numbers\\
ACK/NAK\\
window and pipelining
\end{cf}
