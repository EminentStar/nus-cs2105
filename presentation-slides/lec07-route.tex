\begin{cf}{
CS2105 Lecture 7\\
\large
\textbf{Routing\\[10pt]}
\normalsize
	4 March, 2013
\note{
\tiny
	After this class, you are expected to understand
	\begin{itemize}
	\item the purpose of routing protocols on the Internet
	\item the differences between inter-domain and intra-domain routing.
	\item the workings of link-state and distance vector routing
	\item the principle of Bellman-Ford equation
	\item how RIP works
	\end{itemize}
}
}
\end{cf}

\begin{cf}{
\tikzstyle{layer}=[draw,rectangle,minimum height=1.5 cm, minimum width=5 cm, text=white]
\begin{tikzpicture}[scale=2]
	\node[layer](A)[fill=AppBlue]{Application};
	\node[layer](T)[below = 0cm of A,fill=TransportBlue]{Transport};
	\node[layer](N)[below = 0cm of T,fill=NetworkGreen]{Network};
	\node[layer](L)[below = 0cm of N,fill=LinkBrown]{Link};
	\node[layer](P)[below = 0cm of L,fill=PhysicalRed]{Physical};
\end{tikzpicture}
}
\end{cf}

\tikzstyle{layer}=[draw,rectangle,minimum height=2.5 cm, minimum width=9 cm, text=white]
\tikzstyle{interface}=[draw,single arrow,fill=white, minimum width=0.9cm, minimum height=1.2cm]
\tikzstyle{message}=[draw,rectangle,minimum height=.7cm,minimum width=3cm,fill=white!90!yellow]
\tikzstyle{segment payload}=[draw,rectangle,minimum height=.7cm,minimum width=1cm,inner sep=0cm, fill=white!90!yellow]
\tikzstyle{segment header}=[draw,rectangle,minimum height=.7cm,minimum width=.4cm,inner sep=0cm, fill=white!90!red]
\tikzstyle{datagram header}=[draw,rectangle,minimum height=.7cm,minimum width=.4cm,inner sep=0cm, fill=white!50!NetworkGreen]

\begin{cf}{
	\begin{tikzpicture}[scale=2]
	\node[layer](T)[
		below = 0cm of A,
		yshift = -0.1cm, 
		fill=TransportBlue]{Transport};
	\node[layer](N)[
		below = 0cm of T,
		yshift = -0.1cm, 
		fill=NetworkGreen]{};
        \node[layer](IP)[
		below = 0.1cm of T,
		yshift = -0.1cm,
		xshift = -3cm,
		fill=NetworkGreen!70!white,
		minimum width=2.8cm,
		minimum height=2.3cm
	]{IP};
        \node[layer](ICMP)[
		right = 0.1cm of IP,
		fill=NetworkGreen!70!white,
		minimum width=2.8cm,
		minimum height=2.3cm
	]{ICMP};
        \node[layer](RP)[
		right = 0.1cm of ICMP,
		fill=NetworkGreen!70!white,
		minimum width=2.7cm,
		text width=2.4cm,
		align=center,
		minimum height=2.3cm
	]{\small Routing Protocols};
	\node[layer](L)[
		below = 0cm of N,
		yshift = -0.1cm, 
		fill=LinkBrown]{Link};
	\end{tikzpicture}
}
\end{cf}


\tikzstyle{router}=[draw,cylinder,aspect=1.5,rotate=90,minimum height=2.5 cm, minimum width=3.5cm, text=gray!80!black,fill=white!90!NetworkGreen,font=\small]
\begin{cf}{
	\begin{tikzpicture}
		\node[router](Router){};
		\node[draw,single arrow](L)[left=1cm of Router.north]{};
		\node[draw,single arrow](R)[right=1cm of Router.south]{};
		\node[draw,single arrow, rotate=-45](L1)[above=of L]{};
		\node[draw,single arrow, rotate=45](R1)[above=of R]{};
		\node[draw,single arrow, rotate=45](L2)[below=of L]{};
		\node[draw,single arrow, rotate=-45](R2)[below=of R]{};
	\end{tikzpicture}
}
\end{cf}

\begin{cf}[t]
	forwarding table
\end{cf}

\begin{cf}[t]
	longest prefix matching
\end{cf}

\begin{cf}
The Internet is a ``network-of-networks'', organized into autonomous systems (AS), each own by an organization.
\end{cf}

\begin{cf}{
\textbf{Inter}-AS routing\\
\textbf{Intra}-AS routing
}
\end{cf}

\tikzstyle{AS}=[cloud,draw,cloud puffs=10.4,cloud ignores aspect, minimum width=2.5cm, minimum height=1.6cm]
\begin{cf}{
\begin{tikzpicture}
	\node[AS](A)[]{\small NUS};
	\node[AS](B)[below=of A]{\small HKIX};
	\node[AS](C)[below=of B]{\small SingNet};
	\node[AS](D)[right=2cm of B]{\small AT\&T};
	\node[AS](E)[below=of D]{\small UUNet};
	\draw[draw](A)--(B);
	\draw[draw](B)--(C);
	\draw[draw](A)--(D);
	\draw[draw](C)--(D);
	\draw[draw](B)--(D);
	\draw[draw](A)--(E);
	\draw[draw](C)--(E);
	\draw[draw](D)--(E);
\end{tikzpicture}
}
\end{cf}

\begin{cf}{
	\textbf{Abstract} view of intra-domain routing
}
\end{cf}

\begin{cf}{
	Routing: finding the \textbf{least cost path} in a graph.
}
\end{cf}

\begin{cf}{
	\textbf{Link-State} Routing
}
\end{cf}

\begin{cf}{\normalsize
	1. Broadcast link cost to each other\\
	2. Compute least cost path locally
	\note{
	We will not cover Step 2 in CS2105.  The algorithm will be covered in CS2010.
	}
}
\end{cf}

\begin{cf}{
	\textbf{Distance Vector} Routing
}
\end{cf}

\begin{cf}{\normalsize
	1. Swap local view with neighbors\\
	2. Update own's local view\\
	3. Repeat
}
\end{cf}

\tikzstyle{rtr}=[draw,cylinder,aspect=1.5,rotate=90,minimum height=1 cm, minimum width=1.5cm, text=gray!80!black,fill=white!90!NetworkGreen,font=\small]

\begin{cf}{\normalsize
	$d_x(y)$: the cost to reach $y$ from $x$'s view\\[1cm]
	\begin{tikzpicture}
			\node[rtr](X)[label=left:x]{};
			\node[rtr](Y)[label=left:y,right=3cm of X.south]{};
			\draw[->,decorate,decoration={snake}](X)--(Y);
	\end{tikzpicture}
}
\end{cf}

\begin{cf}{\normalsize
	$c(x,y)$: the cost between $x$ and $y$.\\[1cm]
	\begin{tikzpicture}
			\node[rtr](X)[label=left:x]{};
			\node[rtr](Y)[label=left:y,right=2cm of X.south]{};
			\draw[](X)--(Y);
	\end{tikzpicture}
}
\end{cf}

\begin{cf}{
	decentralized\\
	self-terminating\\
	iterative\\
	asynchronous\\
}
\end{cf}
\begin{cf}{
	``count-to-infinity''
}
\end{cf}
\begin{cf}{
	``poisoned reverse''
}
\end{cf}
\begin{cf}{
	\textbf{link state vs. distance vector}\\
	message complexity\\
	convergence speed\\
	robustness (error)\\
}
\end{cf}
\begin{cf}{
	\textbf{RIP:} Routing Information Protocol
}
\end{cf}
\begin{cf}{
	Routing Table
	\begin{tabular}{c|c|c}
	dest & next hop & hop count \\
	\hline
	v & z & 3\\
	w & z & 7\\
	x & x & 1\\
	y & z & 5\\
	\end{tabular}
}
\end{cf}
\begin{cf}{
	Route is \textbf{static} and \textbf{load insensitive}.
}
\end{cf}
\begin{cf}{
	Exchange routing table every 30 seconds over UDP port 520
}
\end{cf}
\begin{cf}{
	``Self-repair'': if no update from a neighbor for 3 minutes, assume neighbor has failed
}
\end{cf}

\begin{cf}{
\textbf{Inter}-AS routing\\
\textbf{Intra}-AS routing
}
\end{cf}

\begin{cf}{
	\textbf{BGP:} Border Gateway Protocol
}
\end{cf}

\begin{cf}{
	Hot Potato Routing: route to AS whose gateway has the least cost.
}
\end{cf}

\begin{cf}{
\begin{tikzpicture}
	\node[AS](A)[]{\small NUS};
	\node[AS](B)[below=of A]{\small HKIX};
	\node[AS](C)[below=of B]{\small SingNet};
	\node[AS](D)[right=of B]{\small AT\&T};
	\node[AS](E)[below=of D]{\small UUNet};
	\draw[draw](A)--(B);
	\draw[draw](B)--(C);
	\draw[draw](A)--(D);
	\draw[draw](C)--(D);
	\draw[draw](B)--(D);
	\draw[draw](A)--(E);
	\draw[draw](C)--(E);
	\draw[draw](D)--(E);
\end{tikzpicture}
}
\end{cf}
