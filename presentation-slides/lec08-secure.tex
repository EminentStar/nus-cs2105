\begin{cf}{
CS2105 Lecture 8\\
\large
\textbf{Network Security\\[10pt]}
\normalsize
	18 March, 2013
\note{
\tiny
	After this class, you are expected to be able to: 
	\begin{itemize}
	\item describe how network security can be compromised.
	\item analyze simple security protocols and identify flaws.
	\item understand (i) how symmetric key cryptography and public key cryptograph can be used to ensure confidentiality; (ii) how MAC (with authentication key and cryptographic hash function) ensure message integrity; (iii) how digital signature (with nonce) ensure source authenticity.
	\item understand the role of certificate authority and digital certificates.
	\item understand how the different techniques are applied in secure email, VPN, IPSec, SSL.
	\end{itemize}
}
}
\end{cf}

\begin{frame}
\begin{center}
\tikzstyle{layer}=[draw,rectangle,minimum height=1.5 cm, minimum width=5 cm, text=white]
	\begin{tikzpicture}[scale=2]
	\node[layer](A)[fill=AppBlue]{Application};
	\node[layer](T)[below = 0cm of A,fill=TransportBlue]{Transport};
	\node[layer](N)[below = 0cm of T,fill=NetworkGreen]{Network};
	\node[layer](L)[below = 0cm of N,fill=LinkBrown]{Link};
	\node[layer](P)[below = 0cm of L,fill=PhysicalRed]{Physical};
	\end{tikzpicture}
\end{center}
\end{frame}

\tikzstyle{user}=[draw,circle,text=white,align=center,minimum width=1.5cm,font =\small]
\tikzstyle{userA}=[user,fill=red!40!black]
\tikzstyle{userB}=[user,fill=blue!40!black]
\tikzstyle{userT}=[user,fill=green!40!black]
\tikzstyle{hidden}=[minimum width=0,inner sep=0]
\begin{cf}{
	\begin{tikzpicture}
	\node[userA,](A){A};
	\node[userB,right=5cm of A](B){B};
	\node[hidden,right=2.5cm of A](x){};
	\node[userT,below=1.5cm of x](T){T};
	\path[<->, line width =3pt] (A) edge node {} (B);
	\path[->, line width =3pt] (x) edge node {} (T);
	\end{tikzpicture}
}
\end{cf}
\begin{cf}{\normalsize
	$T$ can \textbf{eavesdrop} on $A$ and $B$.\\
	Need to ensure \textbf{confidentiality}.\\
	\vspace{1cm}
	\begin{tikzpicture}
	\node[userA,](A){A};
	\node[userB,right=5cm of A](B){B};
	\node[hidden,right=2.5cm of A](x){};
	\node[userT,below=1.5cm of x](T){T};
	\path[<->, line width =3pt] (A) edge node {} (B);
	\path[->, line width =3pt] (x) edge node {} (T);
	\end{tikzpicture}
}
\end{cf}
\begin{cf}{\normalsize
	$T$ can \textbf{modify, insert, delete messages} between $A$ and $B$.\\
	Need to ensure \textbf{message integrity}.\\
	\vspace{1cm}
	\begin{tikzpicture}
	\node[userA,](A){A};
	\node[userB,right=5cm of A](B){B};
	\node[hidden,right=2.5cm of A](x){};
	\node[userT,below=1.5cm of x](T){T};
	\path[<->, line width =3pt] (A) edge node {} (B);
	\path[->, line width =3pt] (x) edge node {} (T);
	\end{tikzpicture}
}
\end{cf}
\begin{cf}{\normalsize
	$T$ can \textbf{pretend} to be $A$ or $B$.\\
	Need to ensure \textbf{end-point authenticity}.\\
	\vspace{1cm}
	\begin{tikzpicture}
	\node[userA,](A){A};
	\node[userB,right=5cm of A](B){B};
	\node[hidden,right=2.5cm of A](x){};
	\node[userT,below=1.5cm of x](T){T};
	\path[<->, line width =3pt] (A) edge node {} (B);
	\path[->, line width =3pt] (x) edge node {} (T);
	\end{tikzpicture}
}
\end{cf}
\begin{cf}{\normalsize
	$T$ can \textbf{attack the communication channel} between $A$ or $B$.\\
	Need to ensure \textbf{operational security}.\\
	\vspace{1cm}
	\begin{tikzpicture}
	\node[userA,](A){A};
	\node[userB,right=5cm of A](B){B};
	\node[hidden,right=2.5cm of A](x){};
	\node[userT,below=1.5cm of x](T){T};
	\path[<->, line width =3pt] (A) edge node {} (B);
	\path[->, line width =3pt] (x) edge node {} (T);
	\end{tikzpicture}
	\note{We skip two important practical topics in CS2105, firewalls and intrusion detection system, since they are rather straightforward.  You can read all about them yourself in Section 8.9. 
	}
}
\end{cf}
\begin{cf}{
	Ensuring \textbf{confidentiality} with \textbf{cryptography}.
}
\end{cf}
\begin{cf}{\normalsize
	\textbf{Abstraction}\\[20pt]

	ciphertext = encrypt(plaintext, key)\\
	plaintext = decrypt(ciphertext, key)
}
\end{cf}
\begin{cf}{
	\textbf{Abstraction}\\[20pt]

	$m' = K_A(m)$\\
	$m = K_B(K_A(m))$\\
	\vspace{1cm}
	\begin{tikzpicture}
	\node[userA,](A){A};
	\node[userB,right=5cm of A](B){B};
	\node[hidden,right=2.5cm of A](x){};
	\node[userT,below=1.5cm of x](T){T};
	\path[<->, line width =3pt] (A) edge node {} (B);
	\path[->, line width =3pt] (x) edge node {} (T);
	\end{tikzpicture}
}
\end{cf}
\begin{cf}{
	\textbf{Property of $K_A$ and $K_B$}\\[20pt]

	Given $K_A(m)$, it should be computationally hard to compute $m$ without $K_B$.
	\note{We skip the mathematical details on how $K_A$ and $K_B$ are implemented.  Interested students can read Section 8.2 in details and take CS4236.
	}
}
\end{cf}
\begin{cf}{
	\textbf{Symmetric Key Cyptography}\\
	$K_A = K_B$\\
	\vspace{1cm}
	\begin{tikzpicture}
	\node[userA,](A){A};
	\node[userB,right=5cm of A](B){B};
	\node[hidden,right=2.5cm of A](x){};
	\node[userT,below=1.5cm of x](T){T};
	\path[<->, line width =3pt] (A) edge node {} (B);
	\path[->, line width =3pt] (x) edge node {} (T);
	\end{tikzpicture}
}
\end{cf}
\begin{cf}{
	\textbf{Public Key Cyptography}\\[20pt]

	$K_A \not = K_B$\\[20pt]
	$m' = K_B^+(m)$\\
	$m = K_B^-(K_B^+(m))$\\[20pt]

}
\end{cf}
\begin{cf}{
	\begin{tikzpicture}
	\node[userA,](A){A};
	\node[userB,right=5cm of A](B){B};
	\node[hidden,right=2.5cm of A](x){};
	\node[userT,below=1.5cm of x](T){T};
	\path[<->, line width =3pt] (A) edge node {} (B);
	\path[->, line width =3pt] (x) edge node {} (T);
	\end{tikzpicture}
}
\end{cf}
%\begin{cf}{
	\begin{tikzpicture}[scale=2]
		\draw[solid] (0,0) -- (0,3);
		\draw[solid] (2,0) -- (2,3);
	\end{tikzpicture}
	}
\end{cf}

%\begin{cf}{
	\begin{tikzpicture}[scale=2]
		\draw[solid] (0,0) -- (0,3);
		\draw[solid] (1.5,0) -- (1.5,3);
		\draw[solid] (3,0) -- (3,3);
	\end{tikzpicture}
	}
\end{cf}

\begin{cf}{
	Ensuring \textbf{message integrity} with \textbf{message authentication code (MAC)}.
}
\end{cf}
\begin{cf}{
	\textbf{Cryptographic Hash Function}\\[20pt]
	hash = $H$(message)
}
\end{cf}
\begin{cf}{
	\textbf{Property of $H$}\\[20pt]
	Given $H$(m), it should be computationally hard to construct $m'$ such that $H(m') = H(m)$.
}
\end{cf}
\begin{cf}{
	\textbf{Message Authentication Code}\\
	\small{(flawed)}\\
	\vspace{1cm}
	\begin{tikzpicture}
	\node[userA,](A){A};
	\node[userB,right=5cm of A](B){B};
	\node[hidden,right=2.5cm of A](x){};
	\node[userT,below=1.5cm of x](T){T};
	\path[<->, line width =3pt] (A) edge node[auto] {$\leftarrow$ \small{$m$, $H(m)$}} (B);
	\path[->, line width =3pt] (x) edge node {} (T);
	\end{tikzpicture}
}
\end{cf}
\begin{cf}{
	\textbf{Message Authentication Code}\\
	\small{(fixed with authentication key $s$, a shared secret between $A$ and $B$)}\\
	\vspace{1cm}
	\begin{tikzpicture}
	\node[userA,](A){A};
	\node[userB,right=5cm of A](B){B};
	\node[hidden,right=2.5cm of A](x){};
	\node[userT,below=1.5cm of x](T){T};
	\path[<->, line width =3pt] (A) edge node[auto] {$\leftarrow$ \small{$m$, $H(m+s)$}} (B);
	\path[->, line width =3pt] (x) edge node {} (T);
	\end{tikzpicture}
}
\end{cf}
\begin{cf}{
	Verifying \textbf{message source} with\\ \textbf{digital signature}.
}
\end{cf}
\begin{cf}{
	\textbf{Digital Signature}\\
	\small{(flawed)}\\
	\vspace{1cm}
	\begin{tikzpicture}
	\node[userA,](A){A};
	\node[userB,right=5cm of A](B){B};
	\node[hidden,right=2.5cm of A](x){};
	\node[userT,below=1.5cm of x](T){T};
	\path[<->, line width =3pt] (A) edge node[auto] {$\leftarrow$ \small{$m$, $H(m+s)$}} (B);
	\path[->, line width =3pt] (x) edge node {} (T);
	\end{tikzpicture}
}
\end{cf}
\begin{cf}{
	\textbf{Digital Signature}\\
	\small{(fixed with public/private key)}\\
	\vspace{1cm}
	\begin{tikzpicture}
	\node[userA,](A){A};
	\node[userB,right=5cm of A](B){B};
	\node[hidden,right=2.5cm of A](x){};
	\node[userT,below=1.5cm of x](T){T};
	\path[<->, line width =3pt] (A) edge node[auto] {$\leftarrow$ \small{$m$, $K_B^-(m)$}} (B);
	\path[->, line width =3pt] (x) edge node {} (T);
	\end{tikzpicture}
}
\end{cf}
\begin{cf}{
	\textbf{Digital Signature}\\
	\small{(flawed with untrusted public key)}\\
	\vspace{1cm}
	\begin{tikzpicture}
	\node[userA,](A){A};
	\node[userB,right=5cm of A](B){B};
	\node[hidden,right=2.5cm of A](x){};
	\node[userT,below=1.5cm of x](T){T};
	\path[<->, line width =3pt] (A) edge node[auto] {} (B);
	\path[->, line width =3pt] (x) edge node[auto] {$\uparrow$ \small{m, $K_T^-(m)$}} (T);
	\end{tikzpicture}
}
\end{cf}
\begin{cf}{
	\textbf{Digital Certificate}\\[20pt]

	(public key, name)
}
\end{cf}
\begin{cf}{
	\textbf{Certificate Authority (CA)}\\[20pt]

	issue digitally signed digital certificates
	\note{
	Can you trust the digital signature (and therefore the public key) of a CA?
		We reserve the discussion on this for Problem Set 6.}
}
\end{cf}
\begin{cf}{
	\textbf{Digital Signature}\\
	\small{(with trusted public key)}\\
	\vspace{1cm}
	\begin{tikzpicture}
	\node[userA](A){A};
	\node[userB,right=5cm of A](B){B};
	\node[hidden,right=2.5cm of A](x){};
	\node[user,fill=purple!40!black,above=1cm of x](CA){CA};
	\node[userT,below=1.5cm of x](T){T};
	\path[<->, line width =3pt] (A) edge node[auto] {} (B);
	\path[->, line width =3pt] (x) edge node[auto] {} (T);
	\end{tikzpicture}
}
\end{cf}
\begin{cf}{
	\textbf{End Point Authentication}
}
\end{cf}
\begin{cf}{
	\textbf{Replay Attack}\\
	\small{(with digital signature)}\\
	\vspace{1cm}
	\begin{tikzpicture}
	\node[userA,](A){A};
	\node[userB,right=5cm of A](B){B};
	\node[hidden,right=2.5cm of A](x){};
	\node[userT,below=1.5cm of x](T){T};
	\path[<->, line width =3pt] (A) edge node[auto] {$\leftarrow$ \small{$m$, $K_B^-(m)$}} (B);
	\path[->, line width =3pt] (x) edge node {} (T);
	\end{tikzpicture}
}
\end{cf}
\begin{cf}{
	\textbf{Replay Attack}\\
	\small{(with MAC)}\\
	\vspace{1cm}
	\begin{tikzpicture}
	\node[userA](A){A};
	\node[userB,right=5cm of A](B){B};
	\node[hidden,right=2.5cm of A](x){};
	\node[userT,below=1.5cm of x](T){T};
	\path[<->, line width =3pt] (A) edge node[auto] {$\leftarrow$ \small{$m$, $H(m+s)$}} (B);
	\path[->, line width =3pt] (x) edge node {} (T);
	\end{tikzpicture}
}
\end{cf}
\begin{cf}{
	\textbf{nonce}: a number used only once
}
\end{cf}
\begin{cf}{
	\textbf{End Point Authentication}\\
	\small{(with MAC and nonce)}\\
	\vspace{1cm}
	\begin{tikzpicture}
	\node[userA](A){A};
	\node[userB,right=6cm of A](B){B};
	\node[hidden,right=3cm of A](x){};
	\node[userT,below=1.5cm of x](T){T};
	\path[<->, line width =3pt] (A) edge node[auto] {\small{$R$ $\rightarrow$ \vspace{1cm} $\leftarrow$ $m$, $H(m+s+R)$}} (B);
	\path[->, line width =3pt] (x) edge node {} (T);
	\end{tikzpicture}
}
\end{cf}
\begin{cf}{
	\textbf{End Point Authentication}\\
	\small{(with public key and nonce)}\\
	\vspace{1cm}
	\begin{tikzpicture}
	\node[userA](A){A};
	\node[userB,right=6cm of A](B){B};
	\node[hidden,right=3cm of A](x){};
	\node[userT,below=1.5cm of x](T){T};
	\path[<->, line width =3pt] (A) edge node[auto] {\small{$R$ $\rightarrow$ \vspace{1cm} $\leftarrow$ $m$, $K_B^-(m+R)$}} (B);
	\path[->, line width =3pt] (x) edge node {} (T);
	\end{tikzpicture}
}
\end{cf}

\begin{cf}{
	Putting everything together: \textbf{Securing E-mail}
	\note{Read Section 8.5 in the textbook for how these techniques are used together to send secure e-mails.}
}
\end{cf}

\tikzstyle{layer}=[draw,rectangle,minimum height=1.5 cm, minimum width=5 cm, text=white]

\begin{cf}{
	\begin{tikzpicture}[scale=2]
	\node[layer](A)[fill=AppBlue]{Application};
	\node[layer](T)[below = 1.1cm of A,fill=TransportBlue]{Transport};
	\node[layer](N)[below = 0cm of T,fill=NetworkGreen]{Network};
	\node[layer](L)[below = 0cm of N,fill=LinkBrown]{Link};

	\node[layer,rounded corners](SSL)[below=0cm of A,fill=AppBlue!50!TransportBlue,rounded corners,font=\small,minimum width=3cm,yshift=.2cm]{SSL};
	\end{tikzpicture}
}
\end{cf}

\begin{cf}[t]
	1. Establishing a shared master key between $A$ and $B$
\end{cf}
\begin{cf}[t]
	2. Derive encryption keys $E_A$, $E_B$ and MAC keys $M_A$, $M_B$ from the master key.
\end{cf}
\begin{cf}[t]
	3. Transfer data $m$ with sequence number $s$ as $m'$\\
	$h = H(m + M_B + s)$\\
	$m' = E_B (m, h)$\\
	\vspace{1cm}
	\begin{tikzpicture}
	\node[draw,rectangle split, rectangle split parts=5, 
	rectangle split horizontal,
	rectangle split part fill={red!50,blue!50,green!50!black,black!50,brown!50},
	text=white,font=\footnotesize]{
		type 
		\nodepart{two} version 
		\nodepart{three} length
		\nodepart{four} data
		\nodepart{five} MAC
	};
	\end{tikzpicture}
\end{cf}

\begin{cf}{
	\begin{tikzpicture}[scale=2]
	\node[layer](A)[fill=AppBlue]{Application};
	\node[layer](T)[below = 0cm of A,fill=TransportBlue]{Transport};
	\node[layer](N)[below = 1.1cm of T,fill=NetworkGreen]{Network};
	\node[layer](L)[below = 0cm of N,fill=LinkBrown]{Link};
	\node[layer,rounded corners, minimum width=3cm,font=\small](IPSec)[below=0cm of T,fill=TransportBlue!50!NetworkGreen,yshift=.2cm]{IPSec};
	\end{tikzpicture}
}
\end{cf}

\begin{cf}
	IPSec is a suite of protocols 
\end{cf}

\begin{cf}{
	\textbf{Authentication Header (AH)} protocol: source authentication and data integrity.
}
\end{cf}

\begin{cf}{
	\textbf{Encapsulation Security Payload (ESP)} protocol: source authentication, data integrity and confidentiality.
}
\end{cf}

\begin{frame}\normalsize
	\textbf{Security Association (SA)}: a one-way logical connection, consisting of:
	\begin{itemize}
	\item an SA ID (Security Parameter Index)
	\item src and dst IP addresses
	\item type and key for encryption
	\item type and key for integrity check
	\end{itemize}
\end{frame}

\begin{cf}{\normalsize
	$m = E_B$(IP datagram + padding)\\
	$h = H(i + n + m, M_B)$\\
	\vspace{1cm}
	\begin{tikzpicture}
	\node[draw,rectangle split, 
	rectangle split parts=6, 
	rectangle split horizontal,
	rectangle split part fill={red!50,blue!50,green!50!black,black!50,black!50,brown!50},
	text=white,font=\footnotesize]{
		IP hdr 
		\nodepart{two} SPI $i$
		\nodepart{three} Seq\# $n$
		\nodepart{four} IP datagram
		\nodepart{five} pad
		\nodepart{six} hash $h$
	};
	\end{tikzpicture}
}
\end{cf}

\begin{cf}{\textbf{VPN}: Virtual Private Network\\
	\vspace{1cm}
	\begin{tikzpicture}[start chain=going right]
	\node[userA,on chain](A){A};
	\node[draw,rectangle,on chain](VPN1){};
	\node[cloud,draw,cloud puffs=10.4,cloud ignores aspect, minimum width=2.5cm, minimum height=1.6cm,on chain,line width=2pt](Internet){};
	\node[draw,rectangle,on chain](VPN2){};
	\node[userB,on chain](B){B};
	\draw[style=<->] (A) -- (VPN1);
	\draw[style=<->,line width=2pt] (VPN1) -- (Internet);
	\draw[style=<->,line width=2pt] (Internet) -- (VPN2);
	\draw[style=<->] (VPN2) -- (B);
	\end{tikzpicture}
	\note{VPN can run over either IPSec or SSL.  We will discuss the pros and cons of these approaches in Problem Set 6.}
}
\end{cf}
